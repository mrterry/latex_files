% usage.tex
%
% This file explains how to use the withesis style
%   it is heavily modelled after a similar chapter by McCauley
%   for the Purdue Thesis style
%
% Eric Benedict, May 2000
%
% It is provided without warranty on an AS IS basis.


\chapter{Using the {\tt withesis} Style}

You can get a copy of the \LaTeX{} style for creating a University
of Wisconsin--Madison thesis or dissertation from:

{\tt http://www.cae.wisc.edu/\verb+~+benedict/LaTeX.html}

After somehow unpacking it, you will have the style files ({\tt withesis.sty}
{\tt withe10.sty}, and {\tt withe12.sty}) as well the files used to create
this document.  The files used for this document can be copied and used as a
template for your own thesis or dissertation.

The final printed form of this document is useful, but the
combination of the source code and final copy form a much more valuable
reference.  Keeping a working copy of the this document can be helpful
when you are later working on your thesis or disseration and want to know
how to do something.  If you find a similar example in this document,
then you can simply look at the corresponding source code and add it to
your document.    Because many parts of this document were written by
different people, the styles and techniques are also different and provide
different ways of achieving the same or similar results.

Because of the typical size of theses, it makes sense to break the document
up into several smaller files.  Usually this is done at the chapter level.
These files can then be {\tt \verb|\include|}d in a {\em root} file.  It is
the {\em root} file that you will run \LaTeX{} on.  For this manual, the
root file is called {\tt main.tex}.

\section{The Root File and the Preamble}
The {\tt \verb|\documentclass|} command is used to tell \LaTeX{} that you will
be using the {\tt withesis} document class and it is the first command in your
root file.  Class options such as {\tt 10pt}, {\tt 12pt}, {\tt msthesis} or
{\tt margincheck} are specified here:

{\tt \verb|\documentclass[12pt,msthesis]{withesis}|}

The class option {\tt msthesis} sets the margins to be appropriate for depositing
with the UW library, namely a 1.25 inch left margin with the remaining margins 1 inch.
The defaults for the title page are also defined for a thesis and for a Master of
Science degree.

The class option {\tt margincheck} will place a small black square at the end of
each line which exceeds the margins.\footnote{In reality, the square is
placed at the end of lines which exceed their {\tt \char92hbox}.  This usually
(but not always) indicates a  margin violation on the right margin.  Left
margin violations aren't indicated and if the margin violation is large enough,
there isn't room for the black box to be visiable.}  This is visible both in the {\tt .dvi} file
as well as in the {\tt .ps} file.

The area immediately following this command is called the {\em preamble} and is
used for things like including different style packages,
defining new macros and declaring the page style.

The style packages can be used to easily change the thesis font.  For example,
this document is set in Times Roman instead of the \LaTeX default of Computer
Modern.  This change was performed by including the {\tt times} package:

{\tt\verb|\usepackage{times}|}\footnote{In this document, the typewriter font
{\tt $\backslash$tt} was redefined to use the Computer Modern font with the command
{\tt $\backslash$renewcommand\{$\backslash$ttdefault\}\{cmtt\}}.  
For more information, see~\cite{goossens}.}

Remember that if you change the fonts from the default Computer Modern to
PostScript ({\em e.g.} Times Roman) then in order to correctly see the
document, you will need to convert the {\tt *.dvi} output into a {\tt *.ps}
file and view the document with a PostScript viewer. This is required since 
most {\tt *.dvi} previewer programs cannot 
display PostScript fonts.  Usually, the previewer will substitute
default fonts so the document may be viewed; however, since the alternate
fonts may not be the same size, the formatting of the document may appear
to be incorrect.

The style package for including Postscript figures, {\tt epsfig}, is included with

{\tt\verb|\usepackage{epsfig}|}

If multiple style packages are required, then they can be combined into one statement
as follows:

{\tt\verb|\usepackage{epsfig,times}|}

Many different style packages are available.  For more information, see~\cite{goossens}.

The page styles are defined using a similar method.
A special style is defined for the {\tt withesis} style:

{\tt\verb|\pagestyle{thesisdraft}|}

This style causes the footer text to become:

{\verb| DRAFT: Do Not Distribute        <time><Date>        <input file name>|}

This appears at the bottom of every page.

In addition to the page style command, the {\tt withesis} has defined several useful
commands which are specified in the preamble.  They include {\tt \verb| \draftmargin|},
{\tt \verb|\draftscreen|}, {\tt \verb|\noappendixtables|}, and
{\tt \verb|\noappendixfigures|}.

The command  {\tt \verb|\draftmargin|} draws a PostScript box with the dimensions of
the margins.  This makes it easy to check that the margins are correct and to see if
any of the text or figures are outside of the required margins.  This box is only visible
in the {\tt .ps} file since it is a PostScript special.


The command  {\tt \verb|\draftscreen|} draws a PostScript screen with the word {\em DRAFT}
in light grey and diagonally across the page.  This screen is only visible
in the {\tt .ps} file since it is a PostScript special.

The commands {\tt \verb|\noappendixtables|} and/or {\tt \verb|\noappendixfigures|} should
be used if the appendix does not have either tables or figures respectively.  These commands
inhibit the Appendix Table or Appendix Figure titles in the List of Tables or List of
Figures.\label{usage:noapp}


If you have specified the {\tt psfig} or {\tt epsfig} document style package, then a useful
command is {\tt \verb|\psdraft|}.  This command will show the bounding box that the figure
would occupy (instead of actually including the figure).  This speeds up the draft copy
printing, reduces toner usage and the drawn box is visible in the {\tt .dvi} file.

The next usual command is {\tt \verb|\begin{document}|}.  The following example is part
of the root file used for this manual.

\begin{quote} \singlespace\footnotesize\tt
\begin{verbatim}
\bibliographystyle{plain}
% prelude.tex
%   - titlepage
%   - dedication
%   - acknowledgments
%   - table of contents, list of tables and list of figures
%   - nomenclature
%   - abstract
%============================================================================


\clearpage\pagenumbering{roman}  % This makes the page numbers Roman (i, ii, etc)



% TITLE PAGE
%   - define \title{} \author{} \date{}
\title{How to \LaTeX\ a Thesis}
\author{Eric R. L. Benedict}
\date{2000}
%   - The default degree is ``Doctor of Philosophy''
%     (unless the document style msthesis is specified
%      and then the default degree is ``Master of Science'')
%     Degree can be changed using the command \degree{}
\degree{Master \TeX nician}
%   - The default is dissertation, unless the document style
%     msthesis was specified in which case it becomes thesis.
%     If msthesis is specified for the MS margins, you can
%     still have a dissertation if you specify \disseration
%\disseration
%   - for a masters project report, specify \project
%\project
%   - for a preliminary report, specify \prelim
\prelim
%   - for a masters thesis, specify \thesis
%\thesis
%   - The default department is ``Electrical Engineering''
%     The department can be changed using the command \department{}
%\department{New Department}
%   - once the above are defined, use \maketitle to generate the titlepage
\maketitle

% COPYRIGHT PAGE
%   - To include a copyright page use \copyrightpage
\copyrightpage

% DEDICATION
\begin{dedication}
To my pet rock, Skippy.
\end{dedication}

% ACKNOWLEDGMENTS
\begin{acknowledgments}
I thank the many people who have done lots of nice things for me.
\end{acknowledgments}

% CONTENTS, TABLES, FIGURES
\tableofcontents
\listoftables
\listoffigures

% NOMENCLATURE
\begin{nomenclature}
\begin{description}
\item{\makebox[0.75in][l]{\TeX}}
       \parbox[t]{5in}{a typesetting system by Donald Knuth~\cite{knuth}.  It
       also refers to the ``plain'' format.  The proper pronounciation
       rhymes with ``heck'' and ``peck'' and does not sound like
       ``hex'' or ``Rex.''\\}

\item{\makebox[0.75in][l]{\LaTeX}}  
        \parbox[t]{5in}{a set of \TeX{} macros originally written by Leslie 
        Lamport~\cite{lamport}.  The proper pronunciation is 
        {\tt l\={a}$\cdot$tek'} and not {\tt l\={a}'$\cdot$teks} (see above).\\}

\item{\makebox[0.75in][l]{{\sc Bib}\TeX}} 
         \parbox[t]{5in}{a bibliography generation program by Oren 
                Patashnik~\cite{lamport}
                that can be used with either plain \TeX{} or \LaTeX{}.\\}

\item{\makebox[0.75in][l]{$C_1$}} Constant 1

\item{\makebox[0.75in][l]{$V$}}    Voltage 

\item{\makebox[0.75in][l]{\$}}     US Dollars
\end{description}
\end{nomenclature}


\advisorname{Bucky J. Badger}
\advisortitle{Assistant Professor}
% ABSTRACT
\begin{umiabstract}
  % abstract.tex
%
% This file has the abstract for the withesis style documentation
%
% Eric Benedict, Aug 2000
%
% It is provided without warranty on an AS IS basis.

\noindent       % Don't indent this paragraph.
This is not a thesis or dissertation and Master \TeX nician is not a
degree granted at the University of Wisconsin-Madison.

\vspace*{0.5em}
\noindent       % Don't indent this paragraph.
This explains the basics for using \LaTeX\ to typeset a dissertation,
thesis or masters project or preliminary report for the University of 
Wisconsin-Madison. Chapter
1 talks briefly about the thesis formatting at UW-Madison.  Chapter 2 gives
an overview of the ``essentials'' of \LaTeX{} and was written by Jon Warbrick.
Chapter 3 talks about figures and tables and what a {\em float} is.  Chapter 4
briefly introduces the {\sc Bib}\TeX{} program.  And finally, Chapter 5 discusses
some of the details for using the {\tt withesis} style file.  The material in
Chapters 2-4 basically are a review of fundamental \LaTeX{} usage and form
a reasonable basic tutorial.

\vspace*{0.5em}
\noindent       % Don't indent this paragraph.
The style discussed in this manual was originally written by Dave Kraynie and
edited by James Darrell McCauley as the {\tt puthesis} style for Purdue
University's theses.  This style was modified to form the {\tt withesis} style. This
manual is largely based on a similar manual by James Darrell McCauley and Scott Hucker.
Permission to use, copy, modify and distribute this software and its documentation
for any purpose and without fee is here by granted.  This software and its documentation
is provided ``as is'' without any express or implied warranty.

\end{umiabstract}

\begin{abstract}
  % abstract.tex
%
% This file has the abstract for the withesis style documentation
%
% Eric Benedict, Aug 2000
%
% It is provided without warranty on an AS IS basis.

\noindent       % Don't indent this paragraph.
This is not a thesis or dissertation and Master \TeX nician is not a
degree granted at the University of Wisconsin-Madison.

\vspace*{0.5em}
\noindent       % Don't indent this paragraph.
This explains the basics for using \LaTeX\ to typeset a dissertation,
thesis or masters project or preliminary report for the University of 
Wisconsin-Madison. Chapter
1 talks briefly about the thesis formatting at UW-Madison.  Chapter 2 gives
an overview of the ``essentials'' of \LaTeX{} and was written by Jon Warbrick.
Chapter 3 talks about figures and tables and what a {\em float} is.  Chapter 4
briefly introduces the {\sc Bib}\TeX{} program.  And finally, Chapter 5 discusses
some of the details for using the {\tt withesis} style file.  The material in
Chapters 2-4 basically are a review of fundamental \LaTeX{} usage and form
a reasonable basic tutorial.

\vspace*{0.5em}
\noindent       % Don't indent this paragraph.
The style discussed in this manual was originally written by Dave Kraynie and
edited by James Darrell McCauley as the {\tt puthesis} style for Purdue
University's theses.  This style was modified to form the {\tt withesis} style. This
manual is largely based on a similar manual by James Darrell McCauley and Scott Hucker.
Permission to use, copy, modify and distribute this software and its documentation
for any purpose and without fee is here by granted.  This software and its documentation
is provided ``as is'' without any express or implied warranty.

\end{abstract}


\clearpage\pagenumbering{arabic} % This makes the page numbers Arabic (1, 2, etc)
        % Title page, abstract, table of contents, etc
% Pre-lim
% by Eric Benedict


\chapter{Introducing the {\tt withesis} \LaTeX{} Style Guide}
This manual is was written to test the {\tt withesis} style
file and to provide documentation for this style file.  

\section{History}
The
idea for this came from a similar manual written by James Darrell
McCauley and Scott Hucker in 1993 for the Purdue University thesis
style file.  Content ideas were liberally borrowed from this document.
The {\tt withesis} style file is based on the Purdue thesis file
written by Dave Kraynie and edited by Darrell McCauley.  This base was
edited to meet the format requirements of the University of 
Wisconsin--Madison and several additional new commands were created.
In addition, environments from the UW Mathematics Department were also
incorporated.

\section{Producing Your Thesis or Dissertation}
The {\tt withesis} style file will take care of most of the formatting
requirements for submitting your thesis or dissertation at the University
of Wisconsin-Madison.  There are some requirements on the printing of your
document.  From the Graduate School's {\em UW-Madison Guide To Preparing 
Your Doctoral Dissertation},
\begin{quote}\singlespace
Print your dissertation on a laser printer. (Some high quality dot-matrix
printers may be acceptable.) The printer must produce output that
meets all format and legibility requirements. A professional copy shop
can produce an acceptable copy to be submitted to the Graduate School.
Some copiers enlarge the original between one and two percent. To avoid
problems with margins, produce the original copy with margins larger than
the required minimum. Look carefully at the copy before paying for the
services and ask for pages to be recopied if necessary. Common flaws are:
smudges, copy lines, specks, missing pages, margin shifts, slanting of
the printed image on the page, and poor paper quality.
\end{quote}

\subsection{Required Paper}
The paper which is used for PhD Dissertations should be:
\begin{itemize}
\item 8-1/2 x 11 inches
\item High-quality, white
\item 20 pound weight, bond
\end{itemize}
 
While for Masters Theses, the paper should be:

\begin{itemize}
\item 8-1/2 x 11 inches
\item White
\item Acid-free or pH neutral
\item 20 pound weight
\item 25\% cotton bond minimum
\end{itemize}

Paper that meets these requirements can be purchased at book and stationery
stores.

\subsection{Copyright Page}
\label{copyright}
If you choose to retain and register copyright of the dissertation, prepare
a copyright page using the {\tt withesis} {\tt \verb|\copyrightpage|} command. 
Center the text in the bottom third of the page within the dissertation
margins. This page is not numbered. There is an additional fee for copyrighting
your dissertation which is payable at the bursars office along with the
microfilming and binding fee.

\subsection{Prechecks}
The Graduate School has reserved 9:00-9:30 each morning to answer specific formatting questions
(for example: use of tables, graphs and charts). You may bring in 8-10
pages to be reviewed. No appointment is necessary.

\subsection{Final Checks}
\sloppypar
For information about the final Graduate School review and about depositing
your dissertation in the library, see {\em The Three D's: Deadlines, Defending, 
Depositing Your Doctoral Dissertation} or look
at the web site 
\begin{quote}
{\tt http://www.wisc.edu/grad/gs/degrees/ddd.html}
\end{quote}

\section{Disclaimer}
This software and documentation is provided ``as is'' without any
express or implied warranty.
While care has been taken by the authors of this style file such that the
final product will probably meet the University of Wisconsin's formatting 
requirements this is not guaranteed. 
          % Chapter 1
\include{essentials}     % Edited ``Essential LaTeX'' by Jon Warbrick
\chapter{Figures and Tables}\label{quad}
This chapter\footnote{Most of the text in this chapter's introduction is from {\em How to
\TeX{} a Thesis: The Purdue Thesis Styles}} shows some example ways of incorporating tables and figures into \LaTeX{}.
Special environments exist for tables and figures and are special because they are
allowed to {\em float}---that is, \LaTeX{} doesn't always put them in the exact place
that they occur in your input file.  An algorithm is used to place the floating environments,
or floats, at locations which are typographically correct.  This may cause endless frustration
if you want to have a figure or table occur at a specific location.  There are a few
methods for solving this.

You can exert some influence on \LaTeX{}'s float placement algorithm by using
{\em float position specifiers}.  These specifiers, listed below, tell \LaTeX{}
what you prefer.
\begin{tabbing}
{\tt hhhhhh} \= ``bottom'' \=  \kill
{\tt h}\> ``here'' \> do not move this object \\
{\tt p}\> ``page'' \> put this object on a page of floats \\
{\tt b}\> ``bottom'' \> put this object at the bottom of a page\\
{\tt t}\> ``top'' \> put this object at the top of a page\\
\end{tabbing}

Any combination of these can be used:
\begin{quote}\tt\singlespace\begin{verbatim}
\begin{figure}[htbp]
 ...
\caption{A Figure!}
\end{figure}
\end{verbatim}\end{quote}

In this example, we asked \LaTeX{} to ``put the figure `here' if possible.  If it
is not possible (according to the rule encoded in the float algorithm), put it on the
next float page.  A float page is a page which contains nothing but floating objects,
{\em e.g.} a page of nothing but figures or tables.  If this isn't possible, try to put it
at the `top' of a page.  The last thing to try is to put the figure at the `bottom' of
a page.''

The remainder of this chapter deals with some examples of what to put into the figure,
the ellipsis (\ldots ) in the example above.

\section{Tables}
Table~\ref{pde.tab1} is an example table from the UW Math Department.
\begin{table}[htbp]
\centering
\caption{PDE solve times, $15^3+1$
equations.\label{pde.tab1}}
\begin{tabular}{||l|l|l|l|l|l||}\hline
Precond. & Time & Nonlinear & Krylov
& Function & Precond. \\
 & & Iterations & Iterations & calls & solves \\ \hline
None & 1260.9u & 3 & 26 & 30 & 0  \\
 &(21:09) & & & &  \\ \hline
FFT  & 983.4u & 2  & 5  & 8  & 7 \\
&(16:31) & & & & \\ \hline
\end{tabular}
\end{table}
The code to generate it is as follows:
\begin{quote}\tt\singlespace\begin{verbatim}
\begin{table}[htbp]
\centering
\caption{PDE solve times, $15^3+1$
equations.\label{pde.tab1}}
\begin{tabular}{||l|l|l|l|l|l||}\hline
Precond. & Time & Nonlinear & Krylov
& Function & Precond. \\
 & & Iterations & Iterations & calls & solves \\ \hline
None & 1260.9u & 3 & 26 & 30 & 0  \\
 &(21:09) & & & &  \\ \hline
FFT  & 983.4u & 2  & 5  & 8  & 7 \\
&(16:31) & & & & \\ \hline
\end{tabular}
\end{table}
\end{verbatim}\end{quote}

\section{Figures}
There are many different ways to incorporate figures into a \LaTeX{}
document.  \LaTeX{} has an internal {\tt picture} environment and
some programs will generate files which are in this format and can
be simply {\tt include}d.  In addition to \LaTeX{} native {\tt picture}
format, additional packages can be loaded in the {\tt\verb|\documentstyle|}
command (or using the {\tt input} command) to allow \LaTeX{} to process
non-native formats such as PostScript.

\subsection{\tt gnuplot}
The graph of Figure~\ref{gelfand.fig2}
 was created by gnuplot. For simple graphs this is a
 great utility.  For example, if you want a sin curve in your thesis
 try the following:
\begin{quote}\tt\singlespace\begin{verbatim}
 (terminal window): gnuplot
 (in gnuplot):
                 set terminal latex
                 set output "foo.tex"
                 plot sin(x)
                 quit
\end{verbatim}\end{quote}
This will generate a file called {\tt foo.tex} which can be read in
with the following statements.
\begin{figure}[htbp]
\centering
% GNUPLOT: LaTeX picture
\setlength{\unitlength}{0.240900pt}
\ifx\plotpoint\undefined\newsavebox{\plotpoint}\fi
\sbox{\plotpoint}{\rule[-0.175pt]{0.350pt}{0.350pt}}%
\begin{picture}(1500,900)(0,0)
%\tenrm
\sbox{\plotpoint}{\rule[-0.175pt]{0.350pt}{0.350pt}}%
\put(264,158){\rule[-0.175pt]{282.335pt}{0.350pt}}
\put(264,158){\rule[-0.175pt]{0.350pt}{151.526pt}}
\put(264,158){\rule[-0.175pt]{4.818pt}{0.350pt}}
%\put(242,158){\makebox(0,0)[r]{0}}
\put(1416,158){\rule[-0.175pt]{4.818pt}{0.350pt}}
\put(264,284){\rule[-0.175pt]{4.818pt}{0.350pt}}
%\put(242,284){\makebox(0,0)[r]{2}}
\put(1416,284){\rule[-0.175pt]{4.818pt}{0.350pt}}
\put(264,410){\rule[-0.175pt]{4.818pt}{0.350pt}}
%\put(242,410){\makebox(0,0)[r]{4}}
\put(1416,410){\rule[-0.175pt]{4.818pt}{0.350pt}}
\put(264,535){\rule[-0.175pt]{4.818pt}{0.350pt}}
%\put(242,535){\makebox(0,0)[r]{6}}
\put(1416,535){\rule[-0.175pt]{4.818pt}{0.350pt}}
\put(264,661){\rule[-0.175pt]{4.818pt}{0.350pt}}
%\put(242,661){\makebox(0,0)[r]{8}}
\put(1416,661){\rule[-0.175pt]{4.818pt}{0.350pt}}
\put(264,787){\rule[-0.175pt]{4.818pt}{0.350pt}}
%\put(242,787){\makebox(0,0)[r]{10}}
\put(1416,787){\rule[-0.175pt]{4.818pt}{0.350pt}}
\put(264,158){\rule[-0.175pt]{0.350pt}{4.818pt}}
%\put(264,113){\makebox(0,0){0}}
\put(264,767){\rule[-0.175pt]{0.350pt}{4.818pt}}
\put(411,158){\rule[-0.175pt]{0.350pt}{4.818pt}}
%\put(411,113){\makebox(0,0){0.5}}
\put(411,767){\rule[-0.175pt]{0.350pt}{4.818pt}}
\put(557,158){\rule[-0.175pt]{0.350pt}{4.818pt}}
%\put(557,113){\makebox(0,0){1}}
\put(557,767){\rule[-0.175pt]{0.350pt}{4.818pt}}
\put(704,158){\rule[-0.175pt]{0.350pt}{4.818pt}}
%\put(704,113){\makebox(0,0){1.5}}
\put(704,767){\rule[-0.175pt]{0.350pt}{4.818pt}}
\put(850,158){\rule[-0.175pt]{0.350pt}{4.818pt}}
%\put(850,113){\makebox(0,0){2}}
\put(850,767){\rule[-0.175pt]{0.350pt}{4.818pt}}
\put(997,158){\rule[-0.175pt]{0.350pt}{4.818pt}}
%\put(997,113){\makebox(0,0){2.5}}
\put(997,767){\rule[-0.175pt]{0.350pt}{4.818pt}}
\put(1143,158){\rule[-0.175pt]{0.350pt}{4.818pt}}
%\put(1143,113){\makebox(0,0){3}}
\put(1143,767){\rule[-0.175pt]{0.350pt}{4.818pt}}
\put(1290,158){\rule[-0.175pt]{0.350pt}{4.818pt}}
%\put(1290,113){\makebox(0,0){3.5}}
\put(1290,767){\rule[-0.175pt]{0.350pt}{4.818pt}}
\put(1436,158){\rule[-0.175pt]{0.350pt}{4.818pt}}
%\put(1436,113){\makebox(0,0){4}}
\put(1436,767){\rule[-0.175pt]{0.350pt}{4.818pt}}
\put(264,158){\rule[-0.175pt]{282.335pt}{0.350pt}}
\put(1436,158){\rule[-0.175pt]{0.350pt}{151.526pt}}
\put(264,787){\rule[-0.175pt]{282.335pt}{0.350pt}}
\put(100,472){\makebox(0,0)[l]{\shortstack{$\| u\|$}}}
\put(850,68){\makebox(0,0){$\lambda$}}
%\put(850,832){\makebox(0,0){plot}}
\put(264,158){\rule[-0.175pt]{0.350pt}{151.526pt}}
%\put(1306,722){\makebox(0,0)[r]{}}
%\put(1328,722){\rule[-0.175pt]{15.899pt}{0.350pt}}
\put(264,158){\usebox{\plotpoint}}
\put(264,158){\rule[-0.175pt]{6.304pt}{0.350pt}}
\put(290,159){\rule[-0.175pt]{6.304pt}{0.350pt}}
\put(316,160){\rule[-0.175pt]{6.304pt}{0.350pt}}
\put(342,161){\rule[-0.175pt]{6.304pt}{0.350pt}}
\put(368,162){\rule[-0.175pt]{6.304pt}{0.350pt}}
\put(394,163){\rule[-0.175pt]{6.304pt}{0.350pt}}
\put(420,164){\rule[-0.175pt]{5.644pt}{0.350pt}}
\put(444,165){\rule[-0.175pt]{5.644pt}{0.350pt}}
\put(467,166){\rule[-0.175pt]{5.644pt}{0.350pt}}
\put(491,167){\rule[-0.175pt]{5.644pt}{0.350pt}}
\put(514,168){\rule[-0.175pt]{5.644pt}{0.350pt}}
\put(538,169){\rule[-0.175pt]{5.644pt}{0.350pt}}
\put(561,170){\rule[-0.175pt]{5.644pt}{0.350pt}}
\put(585,171){\rule[-0.175pt]{6.384pt}{0.350pt}}
\put(611,172){\rule[-0.175pt]{6.384pt}{0.350pt}}
\put(638,173){\rule[-0.175pt]{6.384pt}{0.350pt}}
\put(664,174){\rule[-0.175pt]{6.384pt}{0.350pt}}
\put(691,175){\rule[-0.175pt]{6.384pt}{0.350pt}}
\put(717,176){\rule[-0.175pt]{6.384pt}{0.350pt}}
\put(744,177){\rule[-0.175pt]{5.862pt}{0.350pt}}
\put(768,178){\rule[-0.175pt]{5.862pt}{0.350pt}}
\put(792,179){\rule[-0.175pt]{5.862pt}{0.350pt}}
\put(816,180){\rule[-0.175pt]{5.862pt}{0.350pt}}
\put(841,181){\rule[-0.175pt]{5.862pt}{0.350pt}}
\put(865,182){\rule[-0.175pt]{5.862pt}{0.350pt}}
\put(889,183){\rule[-0.175pt]{4.371pt}{0.350pt}}
\put(908,184){\rule[-0.175pt]{4.371pt}{0.350pt}}
\put(926,185){\rule[-0.175pt]{4.371pt}{0.350pt}}
\put(944,186){\rule[-0.175pt]{4.371pt}{0.350pt}}
\put(962,187){\rule[-0.175pt]{4.371pt}{0.350pt}}
\put(980,188){\rule[-0.175pt]{4.371pt}{0.350pt}}
\put(998,189){\rule[-0.175pt]{4.371pt}{0.350pt}}
\put(1017,190){\rule[-0.175pt]{4.216pt}{0.350pt}}
\put(1034,191){\rule[-0.175pt]{4.216pt}{0.350pt}}
\put(1052,192){\rule[-0.175pt]{4.216pt}{0.350pt}}
\put(1069,193){\rule[-0.175pt]{4.216pt}{0.350pt}}
\put(1087,194){\rule[-0.175pt]{4.216pt}{0.350pt}}
\put(1104,195){\rule[-0.175pt]{4.216pt}{0.350pt}}
\put(1122,196){\rule[-0.175pt]{3.172pt}{0.350pt}}
\put(1135,197){\rule[-0.175pt]{3.172pt}{0.350pt}}
\put(1148,198){\rule[-0.175pt]{3.172pt}{0.350pt}}
\put(1161,199){\rule[-0.175pt]{3.172pt}{0.350pt}}
\put(1174,200){\rule[-0.175pt]{3.172pt}{0.350pt}}
\put(1187,201){\rule[-0.175pt]{3.172pt}{0.350pt}}
\put(1200,202){\rule[-0.175pt]{1.893pt}{0.350pt}}
\put(1208,203){\rule[-0.175pt]{1.893pt}{0.350pt}}
\put(1216,204){\rule[-0.175pt]{1.893pt}{0.350pt}}
\put(1224,205){\rule[-0.175pt]{1.893pt}{0.350pt}}
\put(1232,206){\rule[-0.175pt]{1.893pt}{0.350pt}}
\put(1240,207){\rule[-0.175pt]{1.893pt}{0.350pt}}
\put(1248,208){\rule[-0.175pt]{1.893pt}{0.350pt}}
\put(1256,209){\rule[-0.175pt]{1.245pt}{0.350pt}}
\put(1261,210){\rule[-0.175pt]{1.245pt}{0.350pt}}
\put(1266,211){\rule[-0.175pt]{1.245pt}{0.350pt}}
\put(1271,212){\rule[-0.175pt]{1.245pt}{0.350pt}}
\put(1276,213){\rule[-0.175pt]{1.245pt}{0.350pt}}
\put(1281,214){\rule[-0.175pt]{1.245pt}{0.350pt}}
\put(1286,215){\usebox{\plotpoint}}
\put(1288,216){\usebox{\plotpoint}}
\put(1289,217){\usebox{\plotpoint}}
\put(1291,218){\usebox{\plotpoint}}
\put(1292,219){\usebox{\plotpoint}}
\put(1294,220){\usebox{\plotpoint}}
\put(1295,221){\usebox{\plotpoint}}
\put(1295,222){\rule[-0.175pt]{0.361pt}{0.350pt}}
\put(1294,223){\rule[-0.175pt]{0.361pt}{0.350pt}}
\put(1292,224){\rule[-0.175pt]{0.361pt}{0.350pt}}
\put(1291,225){\rule[-0.175pt]{0.361pt}{0.350pt}}
\put(1289,226){\rule[-0.175pt]{0.361pt}{0.350pt}}
\put(1288,227){\rule[-0.175pt]{0.361pt}{0.350pt}}
\put(1284,228){\rule[-0.175pt]{0.964pt}{0.350pt}}
\put(1280,229){\rule[-0.175pt]{0.964pt}{0.350pt}}
\put(1276,230){\rule[-0.175pt]{0.964pt}{0.350pt}}
\put(1272,231){\rule[-0.175pt]{0.964pt}{0.350pt}}
\put(1268,232){\rule[-0.175pt]{0.964pt}{0.350pt}}
\put(1264,233){\rule[-0.175pt]{0.964pt}{0.350pt}}
\put(1258,234){\rule[-0.175pt]{1.273pt}{0.350pt}}
\put(1253,235){\rule[-0.175pt]{1.273pt}{0.350pt}}
\put(1248,236){\rule[-0.175pt]{1.273pt}{0.350pt}}
\put(1242,237){\rule[-0.175pt]{1.273pt}{0.350pt}}
\put(1237,238){\rule[-0.175pt]{1.273pt}{0.350pt}}
\put(1232,239){\rule[-0.175pt]{1.273pt}{0.350pt}}
\put(1227,240){\rule[-0.175pt]{1.273pt}{0.350pt}}
\put(1219,241){\rule[-0.175pt]{1.847pt}{0.350pt}}
\put(1211,242){\rule[-0.175pt]{1.847pt}{0.350pt}}
\put(1204,243){\rule[-0.175pt]{1.847pt}{0.350pt}}
\put(1196,244){\rule[-0.175pt]{1.847pt}{0.350pt}}
\put(1188,245){\rule[-0.175pt]{1.847pt}{0.350pt}}
\put(1181,246){\rule[-0.175pt]{1.847pt}{0.350pt}}
\put(1172,247){\rule[-0.175pt]{2.128pt}{0.350pt}}
\put(1163,248){\rule[-0.175pt]{2.128pt}{0.350pt}}
\put(1154,249){\rule[-0.175pt]{2.128pt}{0.350pt}}
\put(1145,250){\rule[-0.175pt]{2.128pt}{0.350pt}}
\put(1136,251){\rule[-0.175pt]{2.128pt}{0.350pt}}
\put(1128,252){\rule[-0.175pt]{2.128pt}{0.350pt}}
\put(1120,253){\rule[-0.175pt]{1.893pt}{0.350pt}}
\put(1112,254){\rule[-0.175pt]{1.893pt}{0.350pt}}
\put(1104,255){\rule[-0.175pt]{1.893pt}{0.350pt}}
\put(1096,256){\rule[-0.175pt]{1.893pt}{0.350pt}}
\put(1088,257){\rule[-0.175pt]{1.893pt}{0.350pt}}
\put(1080,258){\rule[-0.175pt]{1.893pt}{0.350pt}}
\put(1073,259){\rule[-0.175pt]{1.893pt}{0.350pt}}
\put(1063,260){\rule[-0.175pt]{2.208pt}{0.350pt}}
\put(1054,261){\rule[-0.175pt]{2.208pt}{0.350pt}}
\put(1045,262){\rule[-0.175pt]{2.208pt}{0.350pt}}
\put(1036,263){\rule[-0.175pt]{2.208pt}{0.350pt}}
\put(1027,264){\rule[-0.175pt]{2.208pt}{0.350pt}}
\put(1018,265){\rule[-0.175pt]{2.208pt}{0.350pt}}
\put(1009,266){\rule[-0.175pt]{2.168pt}{0.350pt}}
\put(1000,267){\rule[-0.175pt]{2.168pt}{0.350pt}}
\put(991,268){\rule[-0.175pt]{2.168pt}{0.350pt}}
\put(982,269){\rule[-0.175pt]{2.168pt}{0.350pt}}
\put(973,270){\rule[-0.175pt]{2.168pt}{0.350pt}}
\put(964,271){\rule[-0.175pt]{2.168pt}{0.350pt}}
\put(957,272){\rule[-0.175pt]{1.686pt}{0.350pt}}
\put(950,273){\rule[-0.175pt]{1.686pt}{0.350pt}}
\put(943,274){\rule[-0.175pt]{1.686pt}{0.350pt}}
\put(936,275){\rule[-0.175pt]{1.686pt}{0.350pt}}
\put(929,276){\rule[-0.175pt]{1.686pt}{0.350pt}}
\put(922,277){\rule[-0.175pt]{1.686pt}{0.350pt}}
\put(915,278){\rule[-0.175pt]{1.686pt}{0.350pt}}
\put(907,279){\rule[-0.175pt]{1.767pt}{0.350pt}}
\put(900,280){\rule[-0.175pt]{1.767pt}{0.350pt}}
\put(893,281){\rule[-0.175pt]{1.767pt}{0.350pt}}
\put(885,282){\rule[-0.175pt]{1.767pt}{0.350pt}}
\put(878,283){\rule[-0.175pt]{1.767pt}{0.350pt}}
\put(871,284){\rule[-0.175pt]{1.767pt}{0.350pt}}
\put(864,285){\rule[-0.175pt]{1.486pt}{0.350pt}}
\put(858,286){\rule[-0.175pt]{1.486pt}{0.350pt}}
\put(852,287){\rule[-0.175pt]{1.486pt}{0.350pt}}
\put(846,288){\rule[-0.175pt]{1.486pt}{0.350pt}}
\put(840,289){\rule[-0.175pt]{1.486pt}{0.350pt}}
\put(834,290){\rule[-0.175pt]{1.486pt}{0.350pt}}
\put(829,291){\rule[-0.175pt]{0.998pt}{0.350pt}}
\put(825,292){\rule[-0.175pt]{0.998pt}{0.350pt}}
\put(821,293){\rule[-0.175pt]{0.998pt}{0.350pt}}
\put(817,294){\rule[-0.175pt]{0.998pt}{0.350pt}}
\put(813,295){\rule[-0.175pt]{0.998pt}{0.350pt}}
\put(809,296){\rule[-0.175pt]{0.998pt}{0.350pt}}
\put(805,297){\rule[-0.175pt]{0.998pt}{0.350pt}}
\put(801,298){\rule[-0.175pt]{0.883pt}{0.350pt}}
\put(797,299){\rule[-0.175pt]{0.883pt}{0.350pt}}
\put(793,300){\rule[-0.175pt]{0.883pt}{0.350pt}}
\put(790,301){\rule[-0.175pt]{0.883pt}{0.350pt}}
\put(786,302){\rule[-0.175pt]{0.883pt}{0.350pt}}
\put(783,303){\rule[-0.175pt]{0.883pt}{0.350pt}}
\put(780,304){\rule[-0.175pt]{0.522pt}{0.350pt}}
\put(778,305){\rule[-0.175pt]{0.522pt}{0.350pt}}
\put(776,306){\rule[-0.175pt]{0.522pt}{0.350pt}}
\put(774,307){\rule[-0.175pt]{0.522pt}{0.350pt}}
\put(772,308){\rule[-0.175pt]{0.522pt}{0.350pt}}
\put(770,309){\rule[-0.175pt]{0.522pt}{0.350pt}}
\put(770,310){\usebox{\plotpoint}}
\put(769,311){\usebox{\plotpoint}}
\put(768,312){\usebox{\plotpoint}}
\put(767,314){\usebox{\plotpoint}}
\put(766,315){\usebox{\plotpoint}}
\put(765,316){\rule[-0.175pt]{0.350pt}{0.723pt}}
\put(766,320){\rule[-0.175pt]{0.350pt}{0.723pt}}
\put(767,323){\usebox{\plotpoint}}
\put(768,324){\usebox{\plotpoint}}
\put(769,325){\usebox{\plotpoint}}
\put(771,326){\usebox{\plotpoint}}
\put(772,327){\usebox{\plotpoint}}
\put(774,328){\usebox{\plotpoint}}
\put(775,329){\usebox{\plotpoint}}
\put(777,330){\rule[-0.175pt]{0.602pt}{0.350pt}}
\put(779,331){\rule[-0.175pt]{0.602pt}{0.350pt}}
\put(782,332){\rule[-0.175pt]{0.602pt}{0.350pt}}
\put(784,333){\rule[-0.175pt]{0.602pt}{0.350pt}}
\put(787,334){\rule[-0.175pt]{0.602pt}{0.350pt}}
\put(789,335){\rule[-0.175pt]{0.602pt}{0.350pt}}
\put(792,336){\rule[-0.175pt]{0.843pt}{0.350pt}}
\put(795,337){\rule[-0.175pt]{0.843pt}{0.350pt}}
\put(799,338){\rule[-0.175pt]{0.843pt}{0.350pt}}
\put(802,339){\rule[-0.175pt]{0.843pt}{0.350pt}}
\put(806,340){\rule[-0.175pt]{0.843pt}{0.350pt}}
\put(809,341){\rule[-0.175pt]{0.843pt}{0.350pt}}
\put(813,342){\rule[-0.175pt]{0.826pt}{0.350pt}}
\put(816,343){\rule[-0.175pt]{0.826pt}{0.350pt}}
\put(819,344){\rule[-0.175pt]{0.826pt}{0.350pt}}
\put(823,345){\rule[-0.175pt]{0.826pt}{0.350pt}}
\put(826,346){\rule[-0.175pt]{0.826pt}{0.350pt}}
\put(830,347){\rule[-0.175pt]{0.826pt}{0.350pt}}
\put(833,348){\rule[-0.175pt]{0.826pt}{0.350pt}}
\put(837,349){\rule[-0.175pt]{1.084pt}{0.350pt}}
\put(841,350){\rule[-0.175pt]{1.084pt}{0.350pt}}
\put(846,351){\rule[-0.175pt]{1.084pt}{0.350pt}}
\put(850,352){\rule[-0.175pt]{1.084pt}{0.350pt}}
\put(855,353){\rule[-0.175pt]{1.084pt}{0.350pt}}
\put(859,354){\rule[-0.175pt]{1.084pt}{0.350pt}}
\put(864,355){\rule[-0.175pt]{1.164pt}{0.350pt}}
\put(868,356){\rule[-0.175pt]{1.164pt}{0.350pt}}
\put(873,357){\rule[-0.175pt]{1.164pt}{0.350pt}}
\put(878,358){\rule[-0.175pt]{1.164pt}{0.350pt}}
\put(883,359){\rule[-0.175pt]{1.164pt}{0.350pt}}
\put(888,360){\rule[-0.175pt]{1.164pt}{0.350pt}}
\put(892,361){\rule[-0.175pt]{1.032pt}{0.350pt}}
\put(897,362){\rule[-0.175pt]{1.032pt}{0.350pt}}
\put(901,363){\rule[-0.175pt]{1.032pt}{0.350pt}}
\put(905,364){\rule[-0.175pt]{1.032pt}{0.350pt}}
\put(910,365){\rule[-0.175pt]{1.032pt}{0.350pt}}
\put(914,366){\rule[-0.175pt]{1.032pt}{0.350pt}}
\put(918,367){\rule[-0.175pt]{1.032pt}{0.350pt}}
\put(922,368){\rule[-0.175pt]{1.205pt}{0.350pt}}
\put(928,369){\rule[-0.175pt]{1.204pt}{0.350pt}}
\put(933,370){\rule[-0.175pt]{1.204pt}{0.350pt}}
\put(938,371){\rule[-0.175pt]{1.204pt}{0.350pt}}
\put(943,372){\rule[-0.175pt]{1.204pt}{0.350pt}}
\put(948,373){\rule[-0.175pt]{1.204pt}{0.350pt}}
\put(953,374){\rule[-0.175pt]{1.124pt}{0.350pt}}
\put(957,375){\rule[-0.175pt]{1.124pt}{0.350pt}}
\put(962,376){\rule[-0.175pt]{1.124pt}{0.350pt}}
\put(967,377){\rule[-0.175pt]{1.124pt}{0.350pt}}
\put(971,378){\rule[-0.175pt]{1.124pt}{0.350pt}}
\put(976,379){\rule[-0.175pt]{1.124pt}{0.350pt}}
\put(981,380){\rule[-0.175pt]{0.929pt}{0.350pt}}
\put(984,381){\rule[-0.175pt]{0.929pt}{0.350pt}}
\put(988,382){\rule[-0.175pt]{0.929pt}{0.350pt}}
\put(992,383){\rule[-0.175pt]{0.929pt}{0.350pt}}
\put(996,384){\rule[-0.175pt]{0.929pt}{0.350pt}}
\put(1000,385){\rule[-0.175pt]{0.929pt}{0.350pt}}
\put(1004,386){\rule[-0.175pt]{0.929pt}{0.350pt}}
\put(1007,387){\rule[-0.175pt]{0.923pt}{0.350pt}}
\put(1011,388){\rule[-0.175pt]{0.923pt}{0.350pt}}
\put(1015,389){\rule[-0.175pt]{0.923pt}{0.350pt}}
\put(1019,390){\rule[-0.175pt]{0.923pt}{0.350pt}}
\put(1023,391){\rule[-0.175pt]{0.923pt}{0.350pt}}
\put(1027,392){\rule[-0.175pt]{0.923pt}{0.350pt}}
\put(1031,393){\rule[-0.175pt]{0.843pt}{0.350pt}}
\put(1034,394){\rule[-0.175pt]{0.843pt}{0.350pt}}
\put(1038,395){\rule[-0.175pt]{0.843pt}{0.350pt}}
\put(1041,396){\rule[-0.175pt]{0.843pt}{0.350pt}}
\put(1045,397){\rule[-0.175pt]{0.843pt}{0.350pt}}
\put(1048,398){\rule[-0.175pt]{0.843pt}{0.350pt}}
\put(1052,399){\rule[-0.175pt]{0.585pt}{0.350pt}}
\put(1054,400){\rule[-0.175pt]{0.585pt}{0.350pt}}
\put(1056,401){\rule[-0.175pt]{0.585pt}{0.350pt}}
\put(1059,402){\rule[-0.175pt]{0.585pt}{0.350pt}}
\put(1061,403){\rule[-0.175pt]{0.585pt}{0.350pt}}
\put(1064,404){\rule[-0.175pt]{0.585pt}{0.350pt}}
\put(1066,405){\rule[-0.175pt]{0.585pt}{0.350pt}}
\put(1069,406){\rule[-0.175pt]{0.522pt}{0.350pt}}
\put(1071,407){\rule[-0.175pt]{0.522pt}{0.350pt}}
\put(1073,408){\rule[-0.175pt]{0.522pt}{0.350pt}}
\put(1075,409){\rule[-0.175pt]{0.522pt}{0.350pt}}
\put(1077,410){\rule[-0.175pt]{0.522pt}{0.350pt}}
\put(1079,411){\rule[-0.175pt]{0.522pt}{0.350pt}}
\put(1081,412){\rule[-0.175pt]{0.402pt}{0.350pt}}
\put(1083,413){\rule[-0.175pt]{0.401pt}{0.350pt}}
\put(1085,414){\rule[-0.175pt]{0.401pt}{0.350pt}}
\put(1086,415){\rule[-0.175pt]{0.401pt}{0.350pt}}
\put(1088,416){\rule[-0.175pt]{0.401pt}{0.350pt}}
\put(1090,417){\rule[-0.175pt]{0.401pt}{0.350pt}}
\put(1091,418){\usebox{\plotpoint}}
\put(1092,418){\usebox{\plotpoint}}
\put(1093,419){\usebox{\plotpoint}}
\put(1094,420){\usebox{\plotpoint}}
\put(1095,422){\usebox{\plotpoint}}
\put(1096,423){\usebox{\plotpoint}}
\put(1097,424){\rule[-0.175pt]{0.350pt}{0.723pt}}
\put(1098,428){\rule[-0.175pt]{0.350pt}{0.723pt}}
\put(1099,431){\rule[-0.175pt]{0.350pt}{1.686pt}}
\put(1098,438){\usebox{\plotpoint}}
\put(1097,439){\usebox{\plotpoint}}
\put(1096,440){\usebox{\plotpoint}}
\put(1095,441){\usebox{\plotpoint}}
\put(1094,442){\usebox{\plotpoint}}
\put(1091,444){\usebox{\plotpoint}}
\put(1090,445){\usebox{\plotpoint}}
\put(1089,446){\usebox{\plotpoint}}
\put(1088,447){\usebox{\plotpoint}}
\put(1087,448){\usebox{\plotpoint}}
\put(1086,449){\usebox{\plotpoint}}
\put(1084,450){\usebox{\plotpoint}}
\put(1083,451){\usebox{\plotpoint}}
\put(1081,452){\usebox{\plotpoint}}
\put(1080,453){\usebox{\plotpoint}}
\put(1078,454){\usebox{\plotpoint}}
\put(1077,455){\usebox{\plotpoint}}
\put(1076,456){\usebox{\plotpoint}}
\put(1074,457){\rule[-0.175pt]{0.442pt}{0.350pt}}
\put(1072,458){\rule[-0.175pt]{0.442pt}{0.350pt}}
\put(1070,459){\rule[-0.175pt]{0.442pt}{0.350pt}}
\put(1068,460){\rule[-0.175pt]{0.442pt}{0.350pt}}
\put(1066,461){\rule[-0.175pt]{0.442pt}{0.350pt}}
\put(1065,462){\rule[-0.175pt]{0.442pt}{0.350pt}}
\put(1063,463){\rule[-0.175pt]{0.482pt}{0.350pt}}
\put(1061,464){\rule[-0.175pt]{0.482pt}{0.350pt}}
\put(1059,465){\rule[-0.175pt]{0.482pt}{0.350pt}}
\put(1057,466){\rule[-0.175pt]{0.482pt}{0.350pt}}
\put(1055,467){\rule[-0.175pt]{0.482pt}{0.350pt}}
\put(1053,468){\rule[-0.175pt]{0.482pt}{0.350pt}}
\put(1051,469){\rule[-0.175pt]{0.482pt}{0.350pt}}
\put(1049,470){\rule[-0.175pt]{0.482pt}{0.350pt}}
\put(1047,471){\rule[-0.175pt]{0.482pt}{0.350pt}}
\put(1045,472){\rule[-0.175pt]{0.482pt}{0.350pt}}
\put(1043,473){\rule[-0.175pt]{0.482pt}{0.350pt}}
\put(1041,474){\rule[-0.175pt]{0.482pt}{0.350pt}}
\put(1039,475){\rule[-0.175pt]{0.482pt}{0.350pt}}
\put(1036,476){\rule[-0.175pt]{0.522pt}{0.350pt}}
\put(1034,477){\rule[-0.175pt]{0.522pt}{0.350pt}}
\put(1032,478){\rule[-0.175pt]{0.522pt}{0.350pt}}
\put(1030,479){\rule[-0.175pt]{0.522pt}{0.350pt}}
\put(1028,480){\rule[-0.175pt]{0.522pt}{0.350pt}}
\put(1026,481){\rule[-0.175pt]{0.522pt}{0.350pt}}
\put(1023,482){\rule[-0.175pt]{0.522pt}{0.350pt}}
\put(1021,483){\rule[-0.175pt]{0.522pt}{0.350pt}}
\put(1019,484){\rule[-0.175pt]{0.522pt}{0.350pt}}
\put(1017,485){\rule[-0.175pt]{0.522pt}{0.350pt}}
\put(1015,486){\rule[-0.175pt]{0.522pt}{0.350pt}}
\put(1013,487){\rule[-0.175pt]{0.522pt}{0.350pt}}
\put(1011,488){\rule[-0.175pt]{0.447pt}{0.350pt}}
\put(1009,489){\rule[-0.175pt]{0.447pt}{0.350pt}}
\put(1007,490){\rule[-0.175pt]{0.447pt}{0.350pt}}
\put(1005,491){\rule[-0.175pt]{0.447pt}{0.350pt}}
\put(1003,492){\rule[-0.175pt]{0.447pt}{0.350pt}}
\put(1001,493){\rule[-0.175pt]{0.447pt}{0.350pt}}
\put(1000,494){\rule[-0.175pt]{0.447pt}{0.350pt}}
\put(998,495){\rule[-0.175pt]{0.442pt}{0.350pt}}
\put(996,496){\rule[-0.175pt]{0.442pt}{0.350pt}}
\put(994,497){\rule[-0.175pt]{0.442pt}{0.350pt}}
\put(992,498){\rule[-0.175pt]{0.442pt}{0.350pt}}
\put(990,499){\rule[-0.175pt]{0.442pt}{0.350pt}}
\put(989,500){\rule[-0.175pt]{0.442pt}{0.350pt}}
\put(987,501){\rule[-0.175pt]{0.442pt}{0.350pt}}
\put(985,502){\rule[-0.175pt]{0.442pt}{0.350pt}}
\put(983,503){\rule[-0.175pt]{0.442pt}{0.350pt}}
\put(981,504){\rule[-0.175pt]{0.442pt}{0.350pt}}
\put(979,505){\rule[-0.175pt]{0.442pt}{0.350pt}}
\put(978,506){\rule[-0.175pt]{0.442pt}{0.350pt}}
\put(976,507){\usebox{\plotpoint}}
\put(975,508){\usebox{\plotpoint}}
\put(974,509){\usebox{\plotpoint}}
\put(972,510){\usebox{\plotpoint}}
\put(971,511){\usebox{\plotpoint}}
\put(970,512){\usebox{\plotpoint}}
\put(969,513){\usebox{\plotpoint}}
\put(967,514){\usebox{\plotpoint}}
\put(966,515){\usebox{\plotpoint}}
\put(965,516){\usebox{\plotpoint}}
\put(964,517){\usebox{\plotpoint}}
\put(963,518){\usebox{\plotpoint}}
\put(962,519){\usebox{\plotpoint}}
\put(962,520){\usebox{\plotpoint}}
\put(961,521){\usebox{\plotpoint}}
\put(960,522){\usebox{\plotpoint}}
\put(959,524){\usebox{\plotpoint}}
\put(958,525){\usebox{\plotpoint}}
\put(957,527){\rule[-0.175pt]{0.350pt}{0.361pt}}
\put(956,528){\rule[-0.175pt]{0.350pt}{0.361pt}}
\put(955,530){\rule[-0.175pt]{0.350pt}{0.361pt}}
\put(954,531){\rule[-0.175pt]{0.350pt}{0.361pt}}
\put(953,533){\rule[-0.175pt]{0.350pt}{0.723pt}}
\put(952,536){\rule[-0.175pt]{0.350pt}{0.723pt}}
\put(951,539){\rule[-0.175pt]{0.350pt}{1.686pt}}
\put(950,546){\rule[-0.175pt]{0.350pt}{1.445pt}}
\put(951,552){\rule[-0.175pt]{0.350pt}{0.482pt}}
\put(952,554){\rule[-0.175pt]{0.350pt}{0.482pt}}
\put(953,556){\rule[-0.175pt]{0.350pt}{0.482pt}}
\put(954,558){\rule[-0.175pt]{0.350pt}{0.562pt}}
\put(955,560){\rule[-0.175pt]{0.350pt}{0.562pt}}
\put(956,562){\rule[-0.175pt]{0.350pt}{0.562pt}}
\put(957,564){\usebox{\plotpoint}}
\put(958,566){\usebox{\plotpoint}}
\put(959,567){\usebox{\plotpoint}}
\put(960,568){\usebox{\plotpoint}}
\put(961,569){\usebox{\plotpoint}}
\put(962,571){\usebox{\plotpoint}}
\put(963,572){\usebox{\plotpoint}}
\put(964,573){\usebox{\plotpoint}}
\put(965,574){\usebox{\plotpoint}}
\put(966,575){\usebox{\plotpoint}}
\put(967,577){\usebox{\plotpoint}}
\put(968,578){\usebox{\plotpoint}}
\put(969,579){\usebox{\plotpoint}}
\put(970,580){\usebox{\plotpoint}}
\put(971,581){\usebox{\plotpoint}}
\put(972,582){\usebox{\plotpoint}}
\put(973,584){\usebox{\plotpoint}}
\put(974,585){\usebox{\plotpoint}}
\put(975,586){\usebox{\plotpoint}}
\put(976,587){\usebox{\plotpoint}}
\put(977,588){\usebox{\plotpoint}}
\put(978,589){\usebox{\plotpoint}}
\put(979,590){\usebox{\plotpoint}}
\put(980,591){\usebox{\plotpoint}}
\put(981,592){\usebox{\plotpoint}}
\put(982,593){\usebox{\plotpoint}}
\put(983,594){\usebox{\plotpoint}}
\put(984,595){\usebox{\plotpoint}}
\put(985,596){\usebox{\plotpoint}}
\put(986,597){\usebox{\plotpoint}}
\put(987,598){\usebox{\plotpoint}}
\put(988,600){\usebox{\plotpoint}}
\put(989,601){\usebox{\plotpoint}}
\put(990,603){\usebox{\plotpoint}}
\put(991,604){\usebox{\plotpoint}}
\put(992,605){\usebox{\plotpoint}}
\put(993,606){\usebox{\plotpoint}}
\put(994,607){\usebox{\plotpoint}}
\put(995,608){\usebox{\plotpoint}}
\put(996,609){\usebox{\plotpoint}}
\put(997,610){\usebox{\plotpoint}}
\put(998,611){\usebox{\plotpoint}}
\put(999,612){\usebox{\plotpoint}}
\put(1000,613){\usebox{\plotpoint}}
\put(1001,615){\usebox{\plotpoint}}
\put(1002,616){\usebox{\plotpoint}}
\put(1003,617){\usebox{\plotpoint}}
\put(1004,619){\usebox{\plotpoint}}
\put(1005,620){\usebox{\plotpoint}}
\put(1006,622){\rule[-0.175pt]{0.350pt}{0.361pt}}
\put(1007,623){\rule[-0.175pt]{0.350pt}{0.361pt}}
\put(1008,625){\rule[-0.175pt]{0.350pt}{0.361pt}}
\put(1009,626){\rule[-0.175pt]{0.350pt}{0.361pt}}
\put(1010,628){\rule[-0.175pt]{0.350pt}{0.562pt}}
\put(1011,630){\rule[-0.175pt]{0.350pt}{0.562pt}}
\put(1012,632){\rule[-0.175pt]{0.350pt}{0.562pt}}
\put(1013,634){\rule[-0.175pt]{0.350pt}{0.723pt}}
\put(1014,638){\rule[-0.175pt]{0.350pt}{0.723pt}}
\put(1015,641){\rule[-0.175pt]{0.350pt}{1.445pt}}
\put(1016,647){\rule[-0.175pt]{0.350pt}{1.686pt}}
\put(1017,654){\rule[-0.175pt]{0.350pt}{3.734pt}}
\put(1016,669){\rule[-0.175pt]{0.350pt}{0.843pt}}
\put(1015,673){\rule[-0.175pt]{0.350pt}{1.445pt}}
\put(1014,679){\rule[-0.175pt]{0.350pt}{0.723pt}}
\put(1013,682){\rule[-0.175pt]{0.350pt}{0.723pt}}
\put(1012,685){\rule[-0.175pt]{0.350pt}{0.562pt}}
\put(1011,687){\rule[-0.175pt]{0.350pt}{0.562pt}}
\put(1010,689){\rule[-0.175pt]{0.350pt}{0.562pt}}
\put(1009,691){\rule[-0.175pt]{0.350pt}{0.723pt}}
\put(1008,695){\rule[-0.175pt]{0.350pt}{0.723pt}}
\put(1007,698){\rule[-0.175pt]{0.350pt}{0.482pt}}
\put(1006,700){\rule[-0.175pt]{0.350pt}{0.482pt}}
\put(1005,702){\rule[-0.175pt]{0.350pt}{0.482pt}}
\put(1004,704){\rule[-0.175pt]{0.350pt}{0.562pt}}
\put(1003,706){\rule[-0.175pt]{0.350pt}{0.562pt}}
\put(1002,708){\rule[-0.175pt]{0.350pt}{0.562pt}}
\put(1001,710){\rule[-0.175pt]{0.350pt}{0.723pt}}
\put(1000,714){\rule[-0.175pt]{0.350pt}{0.723pt}}
\put(999,717){\rule[-0.175pt]{0.350pt}{0.482pt}}
\put(998,719){\rule[-0.175pt]{0.350pt}{0.482pt}}
\put(997,721){\rule[-0.175pt]{0.350pt}{0.482pt}}
\put(996,723){\rule[-0.175pt]{0.350pt}{0.843pt}}
\put(995,726){\rule[-0.175pt]{0.350pt}{0.843pt}}
\put(994,730){\rule[-0.175pt]{0.350pt}{0.723pt}}
\put(993,733){\rule[-0.175pt]{0.350pt}{0.723pt}}
\put(992,736){\rule[-0.175pt]{0.350pt}{0.843pt}}
\put(991,739){\rule[-0.175pt]{0.350pt}{0.843pt}}
\put(990,743){\rule[-0.175pt]{0.350pt}{1.445pt}}
\put(989,749){\rule[-0.175pt]{0.350pt}{1.445pt}}
\put(988,755){\rule[-0.175pt]{0.350pt}{1.686pt}}
\put(987,762){\rule[-0.175pt]{0.350pt}{4.577pt}}
\put(988,781){\rule[-0.175pt]{0.350pt}{1.445pt}}
\end{picture}

\caption{Gelfand equation on the ball, $3\leq n \leq 9$.
\label{gelfand.fig2}}
\end{figure}
\begin{quote}\tt\singlespace\begin{verbatim}
\begin{figure}[htbp]
\centering
% GNUPLOT: LaTeX picture
\setlength{\unitlength}{0.240900pt}
\ifx\plotpoint\undefined\newsavebox{\plotpoint}\fi
\sbox{\plotpoint}{\rule[-0.175pt]{0.350pt}{0.350pt}}%
\begin{picture}(1500,900)(0,0)
%\tenrm
\sbox{\plotpoint}{\rule[-0.175pt]{0.350pt}{0.350pt}}%
\put(264,158){\rule[-0.175pt]{282.335pt}{0.350pt}}
\put(264,158){\rule[-0.175pt]{0.350pt}{151.526pt}}
\put(264,158){\rule[-0.175pt]{4.818pt}{0.350pt}}
%\put(242,158){\makebox(0,0)[r]{0}}
\put(1416,158){\rule[-0.175pt]{4.818pt}{0.350pt}}
\put(264,284){\rule[-0.175pt]{4.818pt}{0.350pt}}
%\put(242,284){\makebox(0,0)[r]{2}}
\put(1416,284){\rule[-0.175pt]{4.818pt}{0.350pt}}
\put(264,410){\rule[-0.175pt]{4.818pt}{0.350pt}}
%\put(242,410){\makebox(0,0)[r]{4}}
\put(1416,410){\rule[-0.175pt]{4.818pt}{0.350pt}}
\put(264,535){\rule[-0.175pt]{4.818pt}{0.350pt}}
%\put(242,535){\makebox(0,0)[r]{6}}
\put(1416,535){\rule[-0.175pt]{4.818pt}{0.350pt}}
\put(264,661){\rule[-0.175pt]{4.818pt}{0.350pt}}
%\put(242,661){\makebox(0,0)[r]{8}}
\put(1416,661){\rule[-0.175pt]{4.818pt}{0.350pt}}
\put(264,787){\rule[-0.175pt]{4.818pt}{0.350pt}}
%\put(242,787){\makebox(0,0)[r]{10}}
\put(1416,787){\rule[-0.175pt]{4.818pt}{0.350pt}}
\put(264,158){\rule[-0.175pt]{0.350pt}{4.818pt}}
%\put(264,113){\makebox(0,0){0}}
\put(264,767){\rule[-0.175pt]{0.350pt}{4.818pt}}
\put(411,158){\rule[-0.175pt]{0.350pt}{4.818pt}}
%\put(411,113){\makebox(0,0){0.5}}
\put(411,767){\rule[-0.175pt]{0.350pt}{4.818pt}}
\put(557,158){\rule[-0.175pt]{0.350pt}{4.818pt}}
%\put(557,113){\makebox(0,0){1}}
\put(557,767){\rule[-0.175pt]{0.350pt}{4.818pt}}
\put(704,158){\rule[-0.175pt]{0.350pt}{4.818pt}}
%\put(704,113){\makebox(0,0){1.5}}
\put(704,767){\rule[-0.175pt]{0.350pt}{4.818pt}}
\put(850,158){\rule[-0.175pt]{0.350pt}{4.818pt}}
%\put(850,113){\makebox(0,0){2}}
\put(850,767){\rule[-0.175pt]{0.350pt}{4.818pt}}
\put(997,158){\rule[-0.175pt]{0.350pt}{4.818pt}}
%\put(997,113){\makebox(0,0){2.5}}
\put(997,767){\rule[-0.175pt]{0.350pt}{4.818pt}}
\put(1143,158){\rule[-0.175pt]{0.350pt}{4.818pt}}
%\put(1143,113){\makebox(0,0){3}}
\put(1143,767){\rule[-0.175pt]{0.350pt}{4.818pt}}
\put(1290,158){\rule[-0.175pt]{0.350pt}{4.818pt}}
%\put(1290,113){\makebox(0,0){3.5}}
\put(1290,767){\rule[-0.175pt]{0.350pt}{4.818pt}}
\put(1436,158){\rule[-0.175pt]{0.350pt}{4.818pt}}
%\put(1436,113){\makebox(0,0){4}}
\put(1436,767){\rule[-0.175pt]{0.350pt}{4.818pt}}
\put(264,158){\rule[-0.175pt]{282.335pt}{0.350pt}}
\put(1436,158){\rule[-0.175pt]{0.350pt}{151.526pt}}
\put(264,787){\rule[-0.175pt]{282.335pt}{0.350pt}}
\put(100,472){\makebox(0,0)[l]{\shortstack{$\| u\|$}}}
\put(850,68){\makebox(0,0){$\lambda$}}
%\put(850,832){\makebox(0,0){plot}}
\put(264,158){\rule[-0.175pt]{0.350pt}{151.526pt}}
%\put(1306,722){\makebox(0,0)[r]{}}
%\put(1328,722){\rule[-0.175pt]{15.899pt}{0.350pt}}
\put(264,158){\usebox{\plotpoint}}
\put(264,158){\rule[-0.175pt]{6.304pt}{0.350pt}}
\put(290,159){\rule[-0.175pt]{6.304pt}{0.350pt}}
\put(316,160){\rule[-0.175pt]{6.304pt}{0.350pt}}
\put(342,161){\rule[-0.175pt]{6.304pt}{0.350pt}}
\put(368,162){\rule[-0.175pt]{6.304pt}{0.350pt}}
\put(394,163){\rule[-0.175pt]{6.304pt}{0.350pt}}
\put(420,164){\rule[-0.175pt]{5.644pt}{0.350pt}}
\put(444,165){\rule[-0.175pt]{5.644pt}{0.350pt}}
\put(467,166){\rule[-0.175pt]{5.644pt}{0.350pt}}
\put(491,167){\rule[-0.175pt]{5.644pt}{0.350pt}}
\put(514,168){\rule[-0.175pt]{5.644pt}{0.350pt}}
\put(538,169){\rule[-0.175pt]{5.644pt}{0.350pt}}
\put(561,170){\rule[-0.175pt]{5.644pt}{0.350pt}}
\put(585,171){\rule[-0.175pt]{6.384pt}{0.350pt}}
\put(611,172){\rule[-0.175pt]{6.384pt}{0.350pt}}
\put(638,173){\rule[-0.175pt]{6.384pt}{0.350pt}}
\put(664,174){\rule[-0.175pt]{6.384pt}{0.350pt}}
\put(691,175){\rule[-0.175pt]{6.384pt}{0.350pt}}
\put(717,176){\rule[-0.175pt]{6.384pt}{0.350pt}}
\put(744,177){\rule[-0.175pt]{5.862pt}{0.350pt}}
\put(768,178){\rule[-0.175pt]{5.862pt}{0.350pt}}
\put(792,179){\rule[-0.175pt]{5.862pt}{0.350pt}}
\put(816,180){\rule[-0.175pt]{5.862pt}{0.350pt}}
\put(841,181){\rule[-0.175pt]{5.862pt}{0.350pt}}
\put(865,182){\rule[-0.175pt]{5.862pt}{0.350pt}}
\put(889,183){\rule[-0.175pt]{4.371pt}{0.350pt}}
\put(908,184){\rule[-0.175pt]{4.371pt}{0.350pt}}
\put(926,185){\rule[-0.175pt]{4.371pt}{0.350pt}}
\put(944,186){\rule[-0.175pt]{4.371pt}{0.350pt}}
\put(962,187){\rule[-0.175pt]{4.371pt}{0.350pt}}
\put(980,188){\rule[-0.175pt]{4.371pt}{0.350pt}}
\put(998,189){\rule[-0.175pt]{4.371pt}{0.350pt}}
\put(1017,190){\rule[-0.175pt]{4.216pt}{0.350pt}}
\put(1034,191){\rule[-0.175pt]{4.216pt}{0.350pt}}
\put(1052,192){\rule[-0.175pt]{4.216pt}{0.350pt}}
\put(1069,193){\rule[-0.175pt]{4.216pt}{0.350pt}}
\put(1087,194){\rule[-0.175pt]{4.216pt}{0.350pt}}
\put(1104,195){\rule[-0.175pt]{4.216pt}{0.350pt}}
\put(1122,196){\rule[-0.175pt]{3.172pt}{0.350pt}}
\put(1135,197){\rule[-0.175pt]{3.172pt}{0.350pt}}
\put(1148,198){\rule[-0.175pt]{3.172pt}{0.350pt}}
\put(1161,199){\rule[-0.175pt]{3.172pt}{0.350pt}}
\put(1174,200){\rule[-0.175pt]{3.172pt}{0.350pt}}
\put(1187,201){\rule[-0.175pt]{3.172pt}{0.350pt}}
\put(1200,202){\rule[-0.175pt]{1.893pt}{0.350pt}}
\put(1208,203){\rule[-0.175pt]{1.893pt}{0.350pt}}
\put(1216,204){\rule[-0.175pt]{1.893pt}{0.350pt}}
\put(1224,205){\rule[-0.175pt]{1.893pt}{0.350pt}}
\put(1232,206){\rule[-0.175pt]{1.893pt}{0.350pt}}
\put(1240,207){\rule[-0.175pt]{1.893pt}{0.350pt}}
\put(1248,208){\rule[-0.175pt]{1.893pt}{0.350pt}}
\put(1256,209){\rule[-0.175pt]{1.245pt}{0.350pt}}
\put(1261,210){\rule[-0.175pt]{1.245pt}{0.350pt}}
\put(1266,211){\rule[-0.175pt]{1.245pt}{0.350pt}}
\put(1271,212){\rule[-0.175pt]{1.245pt}{0.350pt}}
\put(1276,213){\rule[-0.175pt]{1.245pt}{0.350pt}}
\put(1281,214){\rule[-0.175pt]{1.245pt}{0.350pt}}
\put(1286,215){\usebox{\plotpoint}}
\put(1288,216){\usebox{\plotpoint}}
\put(1289,217){\usebox{\plotpoint}}
\put(1291,218){\usebox{\plotpoint}}
\put(1292,219){\usebox{\plotpoint}}
\put(1294,220){\usebox{\plotpoint}}
\put(1295,221){\usebox{\plotpoint}}
\put(1295,222){\rule[-0.175pt]{0.361pt}{0.350pt}}
\put(1294,223){\rule[-0.175pt]{0.361pt}{0.350pt}}
\put(1292,224){\rule[-0.175pt]{0.361pt}{0.350pt}}
\put(1291,225){\rule[-0.175pt]{0.361pt}{0.350pt}}
\put(1289,226){\rule[-0.175pt]{0.361pt}{0.350pt}}
\put(1288,227){\rule[-0.175pt]{0.361pt}{0.350pt}}
\put(1284,228){\rule[-0.175pt]{0.964pt}{0.350pt}}
\put(1280,229){\rule[-0.175pt]{0.964pt}{0.350pt}}
\put(1276,230){\rule[-0.175pt]{0.964pt}{0.350pt}}
\put(1272,231){\rule[-0.175pt]{0.964pt}{0.350pt}}
\put(1268,232){\rule[-0.175pt]{0.964pt}{0.350pt}}
\put(1264,233){\rule[-0.175pt]{0.964pt}{0.350pt}}
\put(1258,234){\rule[-0.175pt]{1.273pt}{0.350pt}}
\put(1253,235){\rule[-0.175pt]{1.273pt}{0.350pt}}
\put(1248,236){\rule[-0.175pt]{1.273pt}{0.350pt}}
\put(1242,237){\rule[-0.175pt]{1.273pt}{0.350pt}}
\put(1237,238){\rule[-0.175pt]{1.273pt}{0.350pt}}
\put(1232,239){\rule[-0.175pt]{1.273pt}{0.350pt}}
\put(1227,240){\rule[-0.175pt]{1.273pt}{0.350pt}}
\put(1219,241){\rule[-0.175pt]{1.847pt}{0.350pt}}
\put(1211,242){\rule[-0.175pt]{1.847pt}{0.350pt}}
\put(1204,243){\rule[-0.175pt]{1.847pt}{0.350pt}}
\put(1196,244){\rule[-0.175pt]{1.847pt}{0.350pt}}
\put(1188,245){\rule[-0.175pt]{1.847pt}{0.350pt}}
\put(1181,246){\rule[-0.175pt]{1.847pt}{0.350pt}}
\put(1172,247){\rule[-0.175pt]{2.128pt}{0.350pt}}
\put(1163,248){\rule[-0.175pt]{2.128pt}{0.350pt}}
\put(1154,249){\rule[-0.175pt]{2.128pt}{0.350pt}}
\put(1145,250){\rule[-0.175pt]{2.128pt}{0.350pt}}
\put(1136,251){\rule[-0.175pt]{2.128pt}{0.350pt}}
\put(1128,252){\rule[-0.175pt]{2.128pt}{0.350pt}}
\put(1120,253){\rule[-0.175pt]{1.893pt}{0.350pt}}
\put(1112,254){\rule[-0.175pt]{1.893pt}{0.350pt}}
\put(1104,255){\rule[-0.175pt]{1.893pt}{0.350pt}}
\put(1096,256){\rule[-0.175pt]{1.893pt}{0.350pt}}
\put(1088,257){\rule[-0.175pt]{1.893pt}{0.350pt}}
\put(1080,258){\rule[-0.175pt]{1.893pt}{0.350pt}}
\put(1073,259){\rule[-0.175pt]{1.893pt}{0.350pt}}
\put(1063,260){\rule[-0.175pt]{2.208pt}{0.350pt}}
\put(1054,261){\rule[-0.175pt]{2.208pt}{0.350pt}}
\put(1045,262){\rule[-0.175pt]{2.208pt}{0.350pt}}
\put(1036,263){\rule[-0.175pt]{2.208pt}{0.350pt}}
\put(1027,264){\rule[-0.175pt]{2.208pt}{0.350pt}}
\put(1018,265){\rule[-0.175pt]{2.208pt}{0.350pt}}
\put(1009,266){\rule[-0.175pt]{2.168pt}{0.350pt}}
\put(1000,267){\rule[-0.175pt]{2.168pt}{0.350pt}}
\put(991,268){\rule[-0.175pt]{2.168pt}{0.350pt}}
\put(982,269){\rule[-0.175pt]{2.168pt}{0.350pt}}
\put(973,270){\rule[-0.175pt]{2.168pt}{0.350pt}}
\put(964,271){\rule[-0.175pt]{2.168pt}{0.350pt}}
\put(957,272){\rule[-0.175pt]{1.686pt}{0.350pt}}
\put(950,273){\rule[-0.175pt]{1.686pt}{0.350pt}}
\put(943,274){\rule[-0.175pt]{1.686pt}{0.350pt}}
\put(936,275){\rule[-0.175pt]{1.686pt}{0.350pt}}
\put(929,276){\rule[-0.175pt]{1.686pt}{0.350pt}}
\put(922,277){\rule[-0.175pt]{1.686pt}{0.350pt}}
\put(915,278){\rule[-0.175pt]{1.686pt}{0.350pt}}
\put(907,279){\rule[-0.175pt]{1.767pt}{0.350pt}}
\put(900,280){\rule[-0.175pt]{1.767pt}{0.350pt}}
\put(893,281){\rule[-0.175pt]{1.767pt}{0.350pt}}
\put(885,282){\rule[-0.175pt]{1.767pt}{0.350pt}}
\put(878,283){\rule[-0.175pt]{1.767pt}{0.350pt}}
\put(871,284){\rule[-0.175pt]{1.767pt}{0.350pt}}
\put(864,285){\rule[-0.175pt]{1.486pt}{0.350pt}}
\put(858,286){\rule[-0.175pt]{1.486pt}{0.350pt}}
\put(852,287){\rule[-0.175pt]{1.486pt}{0.350pt}}
\put(846,288){\rule[-0.175pt]{1.486pt}{0.350pt}}
\put(840,289){\rule[-0.175pt]{1.486pt}{0.350pt}}
\put(834,290){\rule[-0.175pt]{1.486pt}{0.350pt}}
\put(829,291){\rule[-0.175pt]{0.998pt}{0.350pt}}
\put(825,292){\rule[-0.175pt]{0.998pt}{0.350pt}}
\put(821,293){\rule[-0.175pt]{0.998pt}{0.350pt}}
\put(817,294){\rule[-0.175pt]{0.998pt}{0.350pt}}
\put(813,295){\rule[-0.175pt]{0.998pt}{0.350pt}}
\put(809,296){\rule[-0.175pt]{0.998pt}{0.350pt}}
\put(805,297){\rule[-0.175pt]{0.998pt}{0.350pt}}
\put(801,298){\rule[-0.175pt]{0.883pt}{0.350pt}}
\put(797,299){\rule[-0.175pt]{0.883pt}{0.350pt}}
\put(793,300){\rule[-0.175pt]{0.883pt}{0.350pt}}
\put(790,301){\rule[-0.175pt]{0.883pt}{0.350pt}}
\put(786,302){\rule[-0.175pt]{0.883pt}{0.350pt}}
\put(783,303){\rule[-0.175pt]{0.883pt}{0.350pt}}
\put(780,304){\rule[-0.175pt]{0.522pt}{0.350pt}}
\put(778,305){\rule[-0.175pt]{0.522pt}{0.350pt}}
\put(776,306){\rule[-0.175pt]{0.522pt}{0.350pt}}
\put(774,307){\rule[-0.175pt]{0.522pt}{0.350pt}}
\put(772,308){\rule[-0.175pt]{0.522pt}{0.350pt}}
\put(770,309){\rule[-0.175pt]{0.522pt}{0.350pt}}
\put(770,310){\usebox{\plotpoint}}
\put(769,311){\usebox{\plotpoint}}
\put(768,312){\usebox{\plotpoint}}
\put(767,314){\usebox{\plotpoint}}
\put(766,315){\usebox{\plotpoint}}
\put(765,316){\rule[-0.175pt]{0.350pt}{0.723pt}}
\put(766,320){\rule[-0.175pt]{0.350pt}{0.723pt}}
\put(767,323){\usebox{\plotpoint}}
\put(768,324){\usebox{\plotpoint}}
\put(769,325){\usebox{\plotpoint}}
\put(771,326){\usebox{\plotpoint}}
\put(772,327){\usebox{\plotpoint}}
\put(774,328){\usebox{\plotpoint}}
\put(775,329){\usebox{\plotpoint}}
\put(777,330){\rule[-0.175pt]{0.602pt}{0.350pt}}
\put(779,331){\rule[-0.175pt]{0.602pt}{0.350pt}}
\put(782,332){\rule[-0.175pt]{0.602pt}{0.350pt}}
\put(784,333){\rule[-0.175pt]{0.602pt}{0.350pt}}
\put(787,334){\rule[-0.175pt]{0.602pt}{0.350pt}}
\put(789,335){\rule[-0.175pt]{0.602pt}{0.350pt}}
\put(792,336){\rule[-0.175pt]{0.843pt}{0.350pt}}
\put(795,337){\rule[-0.175pt]{0.843pt}{0.350pt}}
\put(799,338){\rule[-0.175pt]{0.843pt}{0.350pt}}
\put(802,339){\rule[-0.175pt]{0.843pt}{0.350pt}}
\put(806,340){\rule[-0.175pt]{0.843pt}{0.350pt}}
\put(809,341){\rule[-0.175pt]{0.843pt}{0.350pt}}
\put(813,342){\rule[-0.175pt]{0.826pt}{0.350pt}}
\put(816,343){\rule[-0.175pt]{0.826pt}{0.350pt}}
\put(819,344){\rule[-0.175pt]{0.826pt}{0.350pt}}
\put(823,345){\rule[-0.175pt]{0.826pt}{0.350pt}}
\put(826,346){\rule[-0.175pt]{0.826pt}{0.350pt}}
\put(830,347){\rule[-0.175pt]{0.826pt}{0.350pt}}
\put(833,348){\rule[-0.175pt]{0.826pt}{0.350pt}}
\put(837,349){\rule[-0.175pt]{1.084pt}{0.350pt}}
\put(841,350){\rule[-0.175pt]{1.084pt}{0.350pt}}
\put(846,351){\rule[-0.175pt]{1.084pt}{0.350pt}}
\put(850,352){\rule[-0.175pt]{1.084pt}{0.350pt}}
\put(855,353){\rule[-0.175pt]{1.084pt}{0.350pt}}
\put(859,354){\rule[-0.175pt]{1.084pt}{0.350pt}}
\put(864,355){\rule[-0.175pt]{1.164pt}{0.350pt}}
\put(868,356){\rule[-0.175pt]{1.164pt}{0.350pt}}
\put(873,357){\rule[-0.175pt]{1.164pt}{0.350pt}}
\put(878,358){\rule[-0.175pt]{1.164pt}{0.350pt}}
\put(883,359){\rule[-0.175pt]{1.164pt}{0.350pt}}
\put(888,360){\rule[-0.175pt]{1.164pt}{0.350pt}}
\put(892,361){\rule[-0.175pt]{1.032pt}{0.350pt}}
\put(897,362){\rule[-0.175pt]{1.032pt}{0.350pt}}
\put(901,363){\rule[-0.175pt]{1.032pt}{0.350pt}}
\put(905,364){\rule[-0.175pt]{1.032pt}{0.350pt}}
\put(910,365){\rule[-0.175pt]{1.032pt}{0.350pt}}
\put(914,366){\rule[-0.175pt]{1.032pt}{0.350pt}}
\put(918,367){\rule[-0.175pt]{1.032pt}{0.350pt}}
\put(922,368){\rule[-0.175pt]{1.205pt}{0.350pt}}
\put(928,369){\rule[-0.175pt]{1.204pt}{0.350pt}}
\put(933,370){\rule[-0.175pt]{1.204pt}{0.350pt}}
\put(938,371){\rule[-0.175pt]{1.204pt}{0.350pt}}
\put(943,372){\rule[-0.175pt]{1.204pt}{0.350pt}}
\put(948,373){\rule[-0.175pt]{1.204pt}{0.350pt}}
\put(953,374){\rule[-0.175pt]{1.124pt}{0.350pt}}
\put(957,375){\rule[-0.175pt]{1.124pt}{0.350pt}}
\put(962,376){\rule[-0.175pt]{1.124pt}{0.350pt}}
\put(967,377){\rule[-0.175pt]{1.124pt}{0.350pt}}
\put(971,378){\rule[-0.175pt]{1.124pt}{0.350pt}}
\put(976,379){\rule[-0.175pt]{1.124pt}{0.350pt}}
\put(981,380){\rule[-0.175pt]{0.929pt}{0.350pt}}
\put(984,381){\rule[-0.175pt]{0.929pt}{0.350pt}}
\put(988,382){\rule[-0.175pt]{0.929pt}{0.350pt}}
\put(992,383){\rule[-0.175pt]{0.929pt}{0.350pt}}
\put(996,384){\rule[-0.175pt]{0.929pt}{0.350pt}}
\put(1000,385){\rule[-0.175pt]{0.929pt}{0.350pt}}
\put(1004,386){\rule[-0.175pt]{0.929pt}{0.350pt}}
\put(1007,387){\rule[-0.175pt]{0.923pt}{0.350pt}}
\put(1011,388){\rule[-0.175pt]{0.923pt}{0.350pt}}
\put(1015,389){\rule[-0.175pt]{0.923pt}{0.350pt}}
\put(1019,390){\rule[-0.175pt]{0.923pt}{0.350pt}}
\put(1023,391){\rule[-0.175pt]{0.923pt}{0.350pt}}
\put(1027,392){\rule[-0.175pt]{0.923pt}{0.350pt}}
\put(1031,393){\rule[-0.175pt]{0.843pt}{0.350pt}}
\put(1034,394){\rule[-0.175pt]{0.843pt}{0.350pt}}
\put(1038,395){\rule[-0.175pt]{0.843pt}{0.350pt}}
\put(1041,396){\rule[-0.175pt]{0.843pt}{0.350pt}}
\put(1045,397){\rule[-0.175pt]{0.843pt}{0.350pt}}
\put(1048,398){\rule[-0.175pt]{0.843pt}{0.350pt}}
\put(1052,399){\rule[-0.175pt]{0.585pt}{0.350pt}}
\put(1054,400){\rule[-0.175pt]{0.585pt}{0.350pt}}
\put(1056,401){\rule[-0.175pt]{0.585pt}{0.350pt}}
\put(1059,402){\rule[-0.175pt]{0.585pt}{0.350pt}}
\put(1061,403){\rule[-0.175pt]{0.585pt}{0.350pt}}
\put(1064,404){\rule[-0.175pt]{0.585pt}{0.350pt}}
\put(1066,405){\rule[-0.175pt]{0.585pt}{0.350pt}}
\put(1069,406){\rule[-0.175pt]{0.522pt}{0.350pt}}
\put(1071,407){\rule[-0.175pt]{0.522pt}{0.350pt}}
\put(1073,408){\rule[-0.175pt]{0.522pt}{0.350pt}}
\put(1075,409){\rule[-0.175pt]{0.522pt}{0.350pt}}
\put(1077,410){\rule[-0.175pt]{0.522pt}{0.350pt}}
\put(1079,411){\rule[-0.175pt]{0.522pt}{0.350pt}}
\put(1081,412){\rule[-0.175pt]{0.402pt}{0.350pt}}
\put(1083,413){\rule[-0.175pt]{0.401pt}{0.350pt}}
\put(1085,414){\rule[-0.175pt]{0.401pt}{0.350pt}}
\put(1086,415){\rule[-0.175pt]{0.401pt}{0.350pt}}
\put(1088,416){\rule[-0.175pt]{0.401pt}{0.350pt}}
\put(1090,417){\rule[-0.175pt]{0.401pt}{0.350pt}}
\put(1091,418){\usebox{\plotpoint}}
\put(1092,418){\usebox{\plotpoint}}
\put(1093,419){\usebox{\plotpoint}}
\put(1094,420){\usebox{\plotpoint}}
\put(1095,422){\usebox{\plotpoint}}
\put(1096,423){\usebox{\plotpoint}}
\put(1097,424){\rule[-0.175pt]{0.350pt}{0.723pt}}
\put(1098,428){\rule[-0.175pt]{0.350pt}{0.723pt}}
\put(1099,431){\rule[-0.175pt]{0.350pt}{1.686pt}}
\put(1098,438){\usebox{\plotpoint}}
\put(1097,439){\usebox{\plotpoint}}
\put(1096,440){\usebox{\plotpoint}}
\put(1095,441){\usebox{\plotpoint}}
\put(1094,442){\usebox{\plotpoint}}
\put(1091,444){\usebox{\plotpoint}}
\put(1090,445){\usebox{\plotpoint}}
\put(1089,446){\usebox{\plotpoint}}
\put(1088,447){\usebox{\plotpoint}}
\put(1087,448){\usebox{\plotpoint}}
\put(1086,449){\usebox{\plotpoint}}
\put(1084,450){\usebox{\plotpoint}}
\put(1083,451){\usebox{\plotpoint}}
\put(1081,452){\usebox{\plotpoint}}
\put(1080,453){\usebox{\plotpoint}}
\put(1078,454){\usebox{\plotpoint}}
\put(1077,455){\usebox{\plotpoint}}
\put(1076,456){\usebox{\plotpoint}}
\put(1074,457){\rule[-0.175pt]{0.442pt}{0.350pt}}
\put(1072,458){\rule[-0.175pt]{0.442pt}{0.350pt}}
\put(1070,459){\rule[-0.175pt]{0.442pt}{0.350pt}}
\put(1068,460){\rule[-0.175pt]{0.442pt}{0.350pt}}
\put(1066,461){\rule[-0.175pt]{0.442pt}{0.350pt}}
\put(1065,462){\rule[-0.175pt]{0.442pt}{0.350pt}}
\put(1063,463){\rule[-0.175pt]{0.482pt}{0.350pt}}
\put(1061,464){\rule[-0.175pt]{0.482pt}{0.350pt}}
\put(1059,465){\rule[-0.175pt]{0.482pt}{0.350pt}}
\put(1057,466){\rule[-0.175pt]{0.482pt}{0.350pt}}
\put(1055,467){\rule[-0.175pt]{0.482pt}{0.350pt}}
\put(1053,468){\rule[-0.175pt]{0.482pt}{0.350pt}}
\put(1051,469){\rule[-0.175pt]{0.482pt}{0.350pt}}
\put(1049,470){\rule[-0.175pt]{0.482pt}{0.350pt}}
\put(1047,471){\rule[-0.175pt]{0.482pt}{0.350pt}}
\put(1045,472){\rule[-0.175pt]{0.482pt}{0.350pt}}
\put(1043,473){\rule[-0.175pt]{0.482pt}{0.350pt}}
\put(1041,474){\rule[-0.175pt]{0.482pt}{0.350pt}}
\put(1039,475){\rule[-0.175pt]{0.482pt}{0.350pt}}
\put(1036,476){\rule[-0.175pt]{0.522pt}{0.350pt}}
\put(1034,477){\rule[-0.175pt]{0.522pt}{0.350pt}}
\put(1032,478){\rule[-0.175pt]{0.522pt}{0.350pt}}
\put(1030,479){\rule[-0.175pt]{0.522pt}{0.350pt}}
\put(1028,480){\rule[-0.175pt]{0.522pt}{0.350pt}}
\put(1026,481){\rule[-0.175pt]{0.522pt}{0.350pt}}
\put(1023,482){\rule[-0.175pt]{0.522pt}{0.350pt}}
\put(1021,483){\rule[-0.175pt]{0.522pt}{0.350pt}}
\put(1019,484){\rule[-0.175pt]{0.522pt}{0.350pt}}
\put(1017,485){\rule[-0.175pt]{0.522pt}{0.350pt}}
\put(1015,486){\rule[-0.175pt]{0.522pt}{0.350pt}}
\put(1013,487){\rule[-0.175pt]{0.522pt}{0.350pt}}
\put(1011,488){\rule[-0.175pt]{0.447pt}{0.350pt}}
\put(1009,489){\rule[-0.175pt]{0.447pt}{0.350pt}}
\put(1007,490){\rule[-0.175pt]{0.447pt}{0.350pt}}
\put(1005,491){\rule[-0.175pt]{0.447pt}{0.350pt}}
\put(1003,492){\rule[-0.175pt]{0.447pt}{0.350pt}}
\put(1001,493){\rule[-0.175pt]{0.447pt}{0.350pt}}
\put(1000,494){\rule[-0.175pt]{0.447pt}{0.350pt}}
\put(998,495){\rule[-0.175pt]{0.442pt}{0.350pt}}
\put(996,496){\rule[-0.175pt]{0.442pt}{0.350pt}}
\put(994,497){\rule[-0.175pt]{0.442pt}{0.350pt}}
\put(992,498){\rule[-0.175pt]{0.442pt}{0.350pt}}
\put(990,499){\rule[-0.175pt]{0.442pt}{0.350pt}}
\put(989,500){\rule[-0.175pt]{0.442pt}{0.350pt}}
\put(987,501){\rule[-0.175pt]{0.442pt}{0.350pt}}
\put(985,502){\rule[-0.175pt]{0.442pt}{0.350pt}}
\put(983,503){\rule[-0.175pt]{0.442pt}{0.350pt}}
\put(981,504){\rule[-0.175pt]{0.442pt}{0.350pt}}
\put(979,505){\rule[-0.175pt]{0.442pt}{0.350pt}}
\put(978,506){\rule[-0.175pt]{0.442pt}{0.350pt}}
\put(976,507){\usebox{\plotpoint}}
\put(975,508){\usebox{\plotpoint}}
\put(974,509){\usebox{\plotpoint}}
\put(972,510){\usebox{\plotpoint}}
\put(971,511){\usebox{\plotpoint}}
\put(970,512){\usebox{\plotpoint}}
\put(969,513){\usebox{\plotpoint}}
\put(967,514){\usebox{\plotpoint}}
\put(966,515){\usebox{\plotpoint}}
\put(965,516){\usebox{\plotpoint}}
\put(964,517){\usebox{\plotpoint}}
\put(963,518){\usebox{\plotpoint}}
\put(962,519){\usebox{\plotpoint}}
\put(962,520){\usebox{\plotpoint}}
\put(961,521){\usebox{\plotpoint}}
\put(960,522){\usebox{\plotpoint}}
\put(959,524){\usebox{\plotpoint}}
\put(958,525){\usebox{\plotpoint}}
\put(957,527){\rule[-0.175pt]{0.350pt}{0.361pt}}
\put(956,528){\rule[-0.175pt]{0.350pt}{0.361pt}}
\put(955,530){\rule[-0.175pt]{0.350pt}{0.361pt}}
\put(954,531){\rule[-0.175pt]{0.350pt}{0.361pt}}
\put(953,533){\rule[-0.175pt]{0.350pt}{0.723pt}}
\put(952,536){\rule[-0.175pt]{0.350pt}{0.723pt}}
\put(951,539){\rule[-0.175pt]{0.350pt}{1.686pt}}
\put(950,546){\rule[-0.175pt]{0.350pt}{1.445pt}}
\put(951,552){\rule[-0.175pt]{0.350pt}{0.482pt}}
\put(952,554){\rule[-0.175pt]{0.350pt}{0.482pt}}
\put(953,556){\rule[-0.175pt]{0.350pt}{0.482pt}}
\put(954,558){\rule[-0.175pt]{0.350pt}{0.562pt}}
\put(955,560){\rule[-0.175pt]{0.350pt}{0.562pt}}
\put(956,562){\rule[-0.175pt]{0.350pt}{0.562pt}}
\put(957,564){\usebox{\plotpoint}}
\put(958,566){\usebox{\plotpoint}}
\put(959,567){\usebox{\plotpoint}}
\put(960,568){\usebox{\plotpoint}}
\put(961,569){\usebox{\plotpoint}}
\put(962,571){\usebox{\plotpoint}}
\put(963,572){\usebox{\plotpoint}}
\put(964,573){\usebox{\plotpoint}}
\put(965,574){\usebox{\plotpoint}}
\put(966,575){\usebox{\plotpoint}}
\put(967,577){\usebox{\plotpoint}}
\put(968,578){\usebox{\plotpoint}}
\put(969,579){\usebox{\plotpoint}}
\put(970,580){\usebox{\plotpoint}}
\put(971,581){\usebox{\plotpoint}}
\put(972,582){\usebox{\plotpoint}}
\put(973,584){\usebox{\plotpoint}}
\put(974,585){\usebox{\plotpoint}}
\put(975,586){\usebox{\plotpoint}}
\put(976,587){\usebox{\plotpoint}}
\put(977,588){\usebox{\plotpoint}}
\put(978,589){\usebox{\plotpoint}}
\put(979,590){\usebox{\plotpoint}}
\put(980,591){\usebox{\plotpoint}}
\put(981,592){\usebox{\plotpoint}}
\put(982,593){\usebox{\plotpoint}}
\put(983,594){\usebox{\plotpoint}}
\put(984,595){\usebox{\plotpoint}}
\put(985,596){\usebox{\plotpoint}}
\put(986,597){\usebox{\plotpoint}}
\put(987,598){\usebox{\plotpoint}}
\put(988,600){\usebox{\plotpoint}}
\put(989,601){\usebox{\plotpoint}}
\put(990,603){\usebox{\plotpoint}}
\put(991,604){\usebox{\plotpoint}}
\put(992,605){\usebox{\plotpoint}}
\put(993,606){\usebox{\plotpoint}}
\put(994,607){\usebox{\plotpoint}}
\put(995,608){\usebox{\plotpoint}}
\put(996,609){\usebox{\plotpoint}}
\put(997,610){\usebox{\plotpoint}}
\put(998,611){\usebox{\plotpoint}}
\put(999,612){\usebox{\plotpoint}}
\put(1000,613){\usebox{\plotpoint}}
\put(1001,615){\usebox{\plotpoint}}
\put(1002,616){\usebox{\plotpoint}}
\put(1003,617){\usebox{\plotpoint}}
\put(1004,619){\usebox{\plotpoint}}
\put(1005,620){\usebox{\plotpoint}}
\put(1006,622){\rule[-0.175pt]{0.350pt}{0.361pt}}
\put(1007,623){\rule[-0.175pt]{0.350pt}{0.361pt}}
\put(1008,625){\rule[-0.175pt]{0.350pt}{0.361pt}}
\put(1009,626){\rule[-0.175pt]{0.350pt}{0.361pt}}
\put(1010,628){\rule[-0.175pt]{0.350pt}{0.562pt}}
\put(1011,630){\rule[-0.175pt]{0.350pt}{0.562pt}}
\put(1012,632){\rule[-0.175pt]{0.350pt}{0.562pt}}
\put(1013,634){\rule[-0.175pt]{0.350pt}{0.723pt}}
\put(1014,638){\rule[-0.175pt]{0.350pt}{0.723pt}}
\put(1015,641){\rule[-0.175pt]{0.350pt}{1.445pt}}
\put(1016,647){\rule[-0.175pt]{0.350pt}{1.686pt}}
\put(1017,654){\rule[-0.175pt]{0.350pt}{3.734pt}}
\put(1016,669){\rule[-0.175pt]{0.350pt}{0.843pt}}
\put(1015,673){\rule[-0.175pt]{0.350pt}{1.445pt}}
\put(1014,679){\rule[-0.175pt]{0.350pt}{0.723pt}}
\put(1013,682){\rule[-0.175pt]{0.350pt}{0.723pt}}
\put(1012,685){\rule[-0.175pt]{0.350pt}{0.562pt}}
\put(1011,687){\rule[-0.175pt]{0.350pt}{0.562pt}}
\put(1010,689){\rule[-0.175pt]{0.350pt}{0.562pt}}
\put(1009,691){\rule[-0.175pt]{0.350pt}{0.723pt}}
\put(1008,695){\rule[-0.175pt]{0.350pt}{0.723pt}}
\put(1007,698){\rule[-0.175pt]{0.350pt}{0.482pt}}
\put(1006,700){\rule[-0.175pt]{0.350pt}{0.482pt}}
\put(1005,702){\rule[-0.175pt]{0.350pt}{0.482pt}}
\put(1004,704){\rule[-0.175pt]{0.350pt}{0.562pt}}
\put(1003,706){\rule[-0.175pt]{0.350pt}{0.562pt}}
\put(1002,708){\rule[-0.175pt]{0.350pt}{0.562pt}}
\put(1001,710){\rule[-0.175pt]{0.350pt}{0.723pt}}
\put(1000,714){\rule[-0.175pt]{0.350pt}{0.723pt}}
\put(999,717){\rule[-0.175pt]{0.350pt}{0.482pt}}
\put(998,719){\rule[-0.175pt]{0.350pt}{0.482pt}}
\put(997,721){\rule[-0.175pt]{0.350pt}{0.482pt}}
\put(996,723){\rule[-0.175pt]{0.350pt}{0.843pt}}
\put(995,726){\rule[-0.175pt]{0.350pt}{0.843pt}}
\put(994,730){\rule[-0.175pt]{0.350pt}{0.723pt}}
\put(993,733){\rule[-0.175pt]{0.350pt}{0.723pt}}
\put(992,736){\rule[-0.175pt]{0.350pt}{0.843pt}}
\put(991,739){\rule[-0.175pt]{0.350pt}{0.843pt}}
\put(990,743){\rule[-0.175pt]{0.350pt}{1.445pt}}
\put(989,749){\rule[-0.175pt]{0.350pt}{1.445pt}}
\put(988,755){\rule[-0.175pt]{0.350pt}{1.686pt}}
\put(987,762){\rule[-0.175pt]{0.350pt}{4.577pt}}
\put(988,781){\rule[-0.175pt]{0.350pt}{1.445pt}}
\end{picture}

\caption{Gelfand equation on the ball, $3\leq n \leq 9$.
\label{gelfand.fig2}}
\end{figure}
\end{verbatim}\end{quote}
One advantage to using the native \LaTeX{} {\tt picture} environment
is that the fonts will be assured to agree and the pictures can be viewed
in the {\tt .dvi} viewer.

\subsection{PostScript}
Many drawing applications now allow the export of a graphic to the
{\em Encapsulated PostScript} format.  These files have a suffix of
{\tt .EPS} or {\tt .EPSF} and are similar to a regular PostScript
file except that they contain a {\em bounding box} which describes
the dimensions of the figure.

In order to include PostScript figures, the {\tt epsfig} (or {\tt psfig}
depending on the system you are using) style file must be included in either
the {\tt\verb|\documentstyle|} command or the preamble using the {\tt input} command.

Figure~\ref{vwcontr} is a plot from Matlab.
\begin{figure}[htbp]
\centerline{
\psfig{figure=vwcontr.eps,width=5in,angle=0}
           }
\caption{$\sigma$ as a Function of Voltage and Speed, $\alpha = 20$}
\label{vwcontr}
\end{figure}
The commands to include this figure are
\begin{quote}\tt\singlespace\begin{verbatim}
\begin{figure}[htbp]
\centerline{
\psfig{figure=vwcontr.ps,width=5in,angle=0}
           }
\caption{$\sigma$ as a Function of Voltage and Speed, $\alpha = 20$}
\label{vwcontr}
\end{figure}
\end{verbatim}\end{quote}

Observe that the {\tt \verb|\psfig|} command allows the scaling of the figure
by setting either the {\tt width} or {\tt height} of the figure.  If only one
dimension is specified, the other is computed to keep the same aspect ratio.
The figure can also be rotated by setting {\tt angle} to the desired value in
degrees.
           % Chapter 3 Edited from UW Math Dept's Sample Thesis
% bibs.tex
%
% This chapter briefly talks about BibTex and is mostly
% copied from a similar chapter from "How to TeX a Thesis:
% The Purdue Thesis Styles" by James Darrell McCauley and
% Scott Hucker
%

\newcommand{\BibTeX}{{\sc Bib}\TeX}

\chapter{Citations and Bibliographies}
This chapter is an edited form of the same chapter from {\em How to 
\TeX{} a Thesis: The Purdue Thesis Styles} by James Darrell McCauley and
Scott Hucker.

The task of compiling and formatting the sources cited in papers can
be quite tedious, especially for large documents like theses.  A program
separate from \LaTeX{}, called ``\BibTeX{},''can be used to automate this task~\cite{lamport}.

\section{The Citation Command}
When referring to the work of someone else, the {\tt \verb|\cite|} command is used.
This generates the citation in the text for you.  In the above paragraph, the command
{\tt \verb|\cite{lamport}|} was used after the word ``task.''  The formatting of your
citation is handled by either the document style or a style option.  The default citation
style uses the number system (a number in square brackets).  Other citation styles
may use the author-date system, (Lamport, 1986) or the superscript$^3$ system.

\section{Bibliography Styles}
The way that a reference is formatted in your bibliography depends on the bibliography
style, which is specified near the beginning of your document with the\break
{\tt \verb|\bibliographstyle{file}|} command.  The file {\tt file.bst} is the name of the 
bibliography style file.  Standard \BibTeX{} bibliography style files include {\tt plain},
{\tt unsrt}, {\tt alpha}, and {\tt abbrev}.  The bibliography style governs whether or not
references are sorted, whether first names or initials are used for authors, whether or 
not last names are listed first, the location of the year in the references (after the
author or at the end of the reference), {\em etc.}.  You may be required by your
department or major professor to follow as style for a particular journal.  If so, then you
will need to find a \BibTeX{} style file to suit your needs.  Most major journals have
style files.  If you cannot locate an appropriate \BibTeX{} style file, then choose the
one which is closest and then edit the {\tt .bbl} file by hand.  See Section~\ref{BBL}
for a brief discussion on the {\tt .bbl} file.  Some common, but non-standard \BibTeX{}
styles include
\begin{tabbing}
{\tt jacs-new.bstxxxx}\= {\em Journal of the American Chemical Society}\kill
{\tt acm.bst}\>The Association for Computing Machinery\\
{\tt ieeetr.bst}\> The {\em IEEE Transactions} style\\
{\tt jacs-new.bst}\> {\em Journal of the American Chemical Society}
\end{tabbing}

\section{The Database}
The  {\tt \verb|\bibliography{file}|} command is placed in your input file at the location
where the ``List of References'' section\footnote{or ``Bibliography'' 
if {\tt \char92 altbibtitle } has been specified in the preamble.} would be.  It specifies the name (or names) of
your bibliographic data base, {\tt file.bib}.  An example entry in a \BibTeX{}
database is:
\begin{quote}\singlespace\tt\begin{verbatim}
@book{ lamport86 ,
     author =    "Leslie Lamport" ,
     title =     "\LaTeX: A Document Preparation System" ,
     publisher = "Addison--Wesley Pub.\ Co." ,
     year =      "1986" ,
     address =   "Reading, MA" 
}
\end{verbatim}\end{quote}

The citation key is the first field in this entry--- citing this book in a \LaTeX{}
file would look like
\begin{quote}\singlespace\tt\begin{verbatim}
According to Lamport~\cite{lamport86} ...
\end{verbatim}\end{quote}
The tilde ({\tt \verb|~|}) is used to tie the word ``Lamport'' to the citation
generated.  The space between these words is then unbreakable---the word ``Lamport''
and the citation \cite{lamport} will not be split across two lines if they happen to occur
near the end of a line.

A listing of all entry types with their required and optional fields is given in 
Appendix~\ref{bibrefs}. There are several tools which exist to help in editing a \BibTeX{}
file, however, their use is beyond the scope of this manual and can be found by searching
the net.  You can simply use a plain text editor like {\tt vi} or {\tt WordPad} to edit
and create the database files.

There are several rules which you must follow when creating your database.  Authors are
always listed by their full names, first name first, and multiple authors are separated
by {\tt and}.  For example
\begin{quote}\singlespace\tt\begin{verbatim}
author = "John Jay Park and Frederick Gene Watson and
          Michelle Catherine Smith",
\end{verbatim}\end{quote}
If you were using {\tt abbrv} as your {\tt bibliographystyle}, a reference for these
authors may look like:
\begin{quote}
J.J. Park, F.G. Watson, and M.C. Smith \ldots
\end{quote}

Some styles only capitalize the first word of the title.  If you use any acronyms or
other words that should always be capitalized in titles, then they should be 
enclosed in {\tt \{\}}'s ({\em e.g.}, {\tt \{Fortran\}}, {\tt \{N\}ewton}).
This protects the case of these characters.

There are several other rules for \BibTeX{} listed in~\cite{lamport} which should be
referred to because they are not discussed here.

\section{Putting It All Together}
\label{BBL}
To aid the reader in understanding how all of this works together, the following 
excerpt was taken from Lamport~\cite{lamport}:
\begin{quotation}\singlespace
When you ran \LaTeX{} with the input file {\tt sample.tex}, you may have
noticed that \LaTeX{} created a file named {\tt sample.aux}.  This file,
called an {\em auxiliary} file, contains cross-referencing information.  Since
{\tt sample.tex} contains no cross-referencing commands, the auxiliary file it
produces has no information.  However, suppose that \LaTeX{} is run with an
input file named {\tt myfile.tex} that has citations and bibliography-making
[or referencing] commands.  The auxiliary file {\tt myfile.aux} that it produces
will contain all of the citation keys and the arguments of the {\tt \verb|\bibliography|}
and {\tt\verb|\bibliographystyle|} commands.  When \BibTeX{} is run, it reads
this information from the auxiliary file and produces a file named {\tt myfile.bbl}
containing \LaTeX{} commands to produce the source list \ldots The next time
\LaTeX{} is run on {\tt myfile.tex}, the {\tt \verb|\bibliography|} command reads
the {\tt bbl} file ({\tt myfile.bbl}), which generates the source list.
\end{quotation}

Thus, the command sequence for a source file called {\tt main.tex} which is going to
use \BibTeX{} would be:
\begin{quote}\singlespace\tt\begin{verbatim}
latex main.tex
bibtex main
latex main
latex main
\end{verbatim}\end{quote}
The first \LaTeX{} is to collect all of the citations for \BibTeX{}.  Then
\BibTeX{} is run to generate the bibliography.  \LaTeX{} is run again to
incorporate the bibliography into the document and the run the last time to
update any references (like pages in the Table of Contents) which changed when
the bibliography was included.
           % Chapter 4 From PU Thesis styles, by J.D. McCauley
% usage.tex
%
% This file explains how to use the withesis style
%   it is heavily modelled after a similar chapter by McCauley
%   for the Purdue Thesis style
%
% Eric Benedict, May 2000
%
% It is provided without warranty on an AS IS basis.


\chapter{Using the {\tt withesis} Style}

You can get a copy of the \LaTeX{} style for creating a University
of Wisconsin--Madison thesis or dissertation from:

{\tt http://www.cae.wisc.edu/\verb+~+benedict/LaTeX.html}

After somehow unpacking it, you will have the style files ({\tt withesis.sty}
{\tt withe10.sty}, and {\tt withe12.sty}) as well the files used to create
this document.  The files used for this document can be copied and used as a
template for your own thesis or dissertation.

The final printed form of this document is useful, but the
combination of the source code and final copy form a much more valuable
reference.  Keeping a working copy of the this document can be helpful
when you are later working on your thesis or disseration and want to know
how to do something.  If you find a similar example in this document,
then you can simply look at the corresponding source code and add it to
your document.    Because many parts of this document were written by
different people, the styles and techniques are also different and provide
different ways of achieving the same or similar results.

Because of the typical size of theses, it makes sense to break the document
up into several smaller files.  Usually this is done at the chapter level.
These files can then be {\tt \verb|\include|}d in a {\em root} file.  It is
the {\em root} file that you will run \LaTeX{} on.  For this manual, the
root file is called {\tt main.tex}.

\section{The Root File and the Preamble}
The {\tt \verb|\documentclass|} command is used to tell \LaTeX{} that you will
be using the {\tt withesis} document class and it is the first command in your
root file.  Class options such as {\tt 10pt}, {\tt 12pt}, {\tt msthesis} or
{\tt margincheck} are specified here:

{\tt \verb|\documentclass[12pt,msthesis]{withesis}|}

The class option {\tt msthesis} sets the margins to be appropriate for depositing
with the UW library, namely a 1.25 inch left margin with the remaining margins 1 inch.
The defaults for the title page are also defined for a thesis and for a Master of
Science degree.

The class option {\tt margincheck} will place a small black square at the end of
each line which exceeds the margins.\footnote{In reality, the square is
placed at the end of lines which exceed their {\tt \char92hbox}.  This usually
(but not always) indicates a  margin violation on the right margin.  Left
margin violations aren't indicated and if the margin violation is large enough,
there isn't room for the black box to be visiable.}  This is visible both in the {\tt .dvi} file
as well as in the {\tt .ps} file.

The area immediately following this command is called the {\em preamble} and is
used for things like including different style packages,
defining new macros and declaring the page style.

The style packages can be used to easily change the thesis font.  For example,
this document is set in Times Roman instead of the \LaTeX default of Computer
Modern.  This change was performed by including the {\tt times} package:

{\tt\verb|\usepackage{times}|}\footnote{In this document, the typewriter font
{\tt $\backslash$tt} was redefined to use the Computer Modern font with the command
{\tt $\backslash$renewcommand\{$\backslash$ttdefault\}\{cmtt\}}.  
For more information, see~\cite{goossens}.}

Remember that if you change the fonts from the default Computer Modern to
PostScript ({\em e.g.} Times Roman) then in order to correctly see the
document, you will need to convert the {\tt *.dvi} output into a {\tt *.ps}
file and view the document with a PostScript viewer. This is required since 
most {\tt *.dvi} previewer programs cannot 
display PostScript fonts.  Usually, the previewer will substitute
default fonts so the document may be viewed; however, since the alternate
fonts may not be the same size, the formatting of the document may appear
to be incorrect.

The style package for including Postscript figures, {\tt epsfig}, is included with

{\tt\verb|\usepackage{epsfig}|}

If multiple style packages are required, then they can be combined into one statement
as follows:

{\tt\verb|\usepackage{epsfig,times}|}

Many different style packages are available.  For more information, see~\cite{goossens}.

The page styles are defined using a similar method.
A special style is defined for the {\tt withesis} style:

{\tt\verb|\pagestyle{thesisdraft}|}

This style causes the footer text to become:

{\verb| DRAFT: Do Not Distribute        <time><Date>        <input file name>|}

This appears at the bottom of every page.

In addition to the page style command, the {\tt withesis} has defined several useful
commands which are specified in the preamble.  They include {\tt \verb| \draftmargin|},
{\tt \verb|\draftscreen|}, {\tt \verb|\noappendixtables|}, and
{\tt \verb|\noappendixfigures|}.

The command  {\tt \verb|\draftmargin|} draws a PostScript box with the dimensions of
the margins.  This makes it easy to check that the margins are correct and to see if
any of the text or figures are outside of the required margins.  This box is only visible
in the {\tt .ps} file since it is a PostScript special.


The command  {\tt \verb|\draftscreen|} draws a PostScript screen with the word {\em DRAFT}
in light grey and diagonally across the page.  This screen is only visible
in the {\tt .ps} file since it is a PostScript special.

The commands {\tt \verb|\noappendixtables|} and/or {\tt \verb|\noappendixfigures|} should
be used if the appendix does not have either tables or figures respectively.  These commands
inhibit the Appendix Table or Appendix Figure titles in the List of Tables or List of
Figures.\label{usage:noapp}


If you have specified the {\tt psfig} or {\tt epsfig} document style package, then a useful
command is {\tt \verb|\psdraft|}.  This command will show the bounding box that the figure
would occupy (instead of actually including the figure).  This speeds up the draft copy
printing, reduces toner usage and the drawn box is visible in the {\tt .dvi} file.

The next usual command is {\tt \verb|\begin{document}|}.  The following example is part
of the root file used for this manual.

\begin{quote} \singlespace\footnotesize\tt
\begin{verbatim}
\bibliographystyle{plain}
% prelude.tex
%   - titlepage
%   - dedication
%   - acknowledgments
%   - table of contents, list of tables and list of figures
%   - nomenclature
%   - abstract
%============================================================================


\clearpage\pagenumbering{roman}  % This makes the page numbers Roman (i, ii, etc)



% TITLE PAGE
%   - define \title{} \author{} \date{}
\title{How to \LaTeX\ a Thesis}
\author{Eric R. L. Benedict}
\date{2000}
%   - The default degree is ``Doctor of Philosophy''
%     (unless the document style msthesis is specified
%      and then the default degree is ``Master of Science'')
%     Degree can be changed using the command \degree{}
\degree{Master \TeX nician}
%   - The default is dissertation, unless the document style
%     msthesis was specified in which case it becomes thesis.
%     If msthesis is specified for the MS margins, you can
%     still have a dissertation if you specify \disseration
%\disseration
%   - for a masters project report, specify \project
%\project
%   - for a preliminary report, specify \prelim
\prelim
%   - for a masters thesis, specify \thesis
%\thesis
%   - The default department is ``Electrical Engineering''
%     The department can be changed using the command \department{}
%\department{New Department}
%   - once the above are defined, use \maketitle to generate the titlepage
\maketitle

% COPYRIGHT PAGE
%   - To include a copyright page use \copyrightpage
\copyrightpage

% DEDICATION
\begin{dedication}
To my pet rock, Skippy.
\end{dedication}

% ACKNOWLEDGMENTS
\begin{acknowledgments}
I thank the many people who have done lots of nice things for me.
\end{acknowledgments}

% CONTENTS, TABLES, FIGURES
\tableofcontents
\listoftables
\listoffigures

% NOMENCLATURE
\begin{nomenclature}
\begin{description}
\item{\makebox[0.75in][l]{\TeX}}
       \parbox[t]{5in}{a typesetting system by Donald Knuth~\cite{knuth}.  It
       also refers to the ``plain'' format.  The proper pronounciation
       rhymes with ``heck'' and ``peck'' and does not sound like
       ``hex'' or ``Rex.''\\}

\item{\makebox[0.75in][l]{\LaTeX}}  
        \parbox[t]{5in}{a set of \TeX{} macros originally written by Leslie 
        Lamport~\cite{lamport}.  The proper pronunciation is 
        {\tt l\={a}$\cdot$tek'} and not {\tt l\={a}'$\cdot$teks} (see above).\\}

\item{\makebox[0.75in][l]{{\sc Bib}\TeX}} 
         \parbox[t]{5in}{a bibliography generation program by Oren 
                Patashnik~\cite{lamport}
                that can be used with either plain \TeX{} or \LaTeX{}.\\}

\item{\makebox[0.75in][l]{$C_1$}} Constant 1

\item{\makebox[0.75in][l]{$V$}}    Voltage 

\item{\makebox[0.75in][l]{\$}}     US Dollars
\end{description}
\end{nomenclature}


\advisorname{Bucky J. Badger}
\advisortitle{Assistant Professor}
% ABSTRACT
\begin{umiabstract}
  % abstract.tex
%
% This file has the abstract for the withesis style documentation
%
% Eric Benedict, Aug 2000
%
% It is provided without warranty on an AS IS basis.

\noindent       % Don't indent this paragraph.
This is not a thesis or dissertation and Master \TeX nician is not a
degree granted at the University of Wisconsin-Madison.

\vspace*{0.5em}
\noindent       % Don't indent this paragraph.
This explains the basics for using \LaTeX\ to typeset a dissertation,
thesis or masters project or preliminary report for the University of 
Wisconsin-Madison. Chapter
1 talks briefly about the thesis formatting at UW-Madison.  Chapter 2 gives
an overview of the ``essentials'' of \LaTeX{} and was written by Jon Warbrick.
Chapter 3 talks about figures and tables and what a {\em float} is.  Chapter 4
briefly introduces the {\sc Bib}\TeX{} program.  And finally, Chapter 5 discusses
some of the details for using the {\tt withesis} style file.  The material in
Chapters 2-4 basically are a review of fundamental \LaTeX{} usage and form
a reasonable basic tutorial.

\vspace*{0.5em}
\noindent       % Don't indent this paragraph.
The style discussed in this manual was originally written by Dave Kraynie and
edited by James Darrell McCauley as the {\tt puthesis} style for Purdue
University's theses.  This style was modified to form the {\tt withesis} style. This
manual is largely based on a similar manual by James Darrell McCauley and Scott Hucker.
Permission to use, copy, modify and distribute this software and its documentation
for any purpose and without fee is here by granted.  This software and its documentation
is provided ``as is'' without any express or implied warranty.

\end{umiabstract}

\begin{abstract}
  % abstract.tex
%
% This file has the abstract for the withesis style documentation
%
% Eric Benedict, Aug 2000
%
% It is provided without warranty on an AS IS basis.

\noindent       % Don't indent this paragraph.
This is not a thesis or dissertation and Master \TeX nician is not a
degree granted at the University of Wisconsin-Madison.

\vspace*{0.5em}
\noindent       % Don't indent this paragraph.
This explains the basics for using \LaTeX\ to typeset a dissertation,
thesis or masters project or preliminary report for the University of 
Wisconsin-Madison. Chapter
1 talks briefly about the thesis formatting at UW-Madison.  Chapter 2 gives
an overview of the ``essentials'' of \LaTeX{} and was written by Jon Warbrick.
Chapter 3 talks about figures and tables and what a {\em float} is.  Chapter 4
briefly introduces the {\sc Bib}\TeX{} program.  And finally, Chapter 5 discusses
some of the details for using the {\tt withesis} style file.  The material in
Chapters 2-4 basically are a review of fundamental \LaTeX{} usage and form
a reasonable basic tutorial.

\vspace*{0.5em}
\noindent       % Don't indent this paragraph.
The style discussed in this manual was originally written by Dave Kraynie and
edited by James Darrell McCauley as the {\tt puthesis} style for Purdue
University's theses.  This style was modified to form the {\tt withesis} style. This
manual is largely based on a similar manual by James Darrell McCauley and Scott Hucker.
Permission to use, copy, modify and distribute this software and its documentation
for any purpose and without fee is here by granted.  This software and its documentation
is provided ``as is'' without any express or implied warranty.

\end{abstract}


\clearpage\pagenumbering{arabic} % This makes the page numbers Arabic (1, 2, etc)
        % Title page, abstract, table of contents, etc
% Pre-lim
% by Eric Benedict


\chapter{Introducing the {\tt withesis} \LaTeX{} Style Guide}
This manual is was written to test the {\tt withesis} style
file and to provide documentation for this style file.  

\section{History}
The
idea for this came from a similar manual written by James Darrell
McCauley and Scott Hucker in 1993 for the Purdue University thesis
style file.  Content ideas were liberally borrowed from this document.
The {\tt withesis} style file is based on the Purdue thesis file
written by Dave Kraynie and edited by Darrell McCauley.  This base was
edited to meet the format requirements of the University of 
Wisconsin--Madison and several additional new commands were created.
In addition, environments from the UW Mathematics Department were also
incorporated.

\section{Producing Your Thesis or Dissertation}
The {\tt withesis} style file will take care of most of the formatting
requirements for submitting your thesis or dissertation at the University
of Wisconsin-Madison.  There are some requirements on the printing of your
document.  From the Graduate School's {\em UW-Madison Guide To Preparing 
Your Doctoral Dissertation},
\begin{quote}\singlespace
Print your dissertation on a laser printer. (Some high quality dot-matrix
printers may be acceptable.) The printer must produce output that
meets all format and legibility requirements. A professional copy shop
can produce an acceptable copy to be submitted to the Graduate School.
Some copiers enlarge the original between one and two percent. To avoid
problems with margins, produce the original copy with margins larger than
the required minimum. Look carefully at the copy before paying for the
services and ask for pages to be recopied if necessary. Common flaws are:
smudges, copy lines, specks, missing pages, margin shifts, slanting of
the printed image on the page, and poor paper quality.
\end{quote}

\subsection{Required Paper}
The paper which is used for PhD Dissertations should be:
\begin{itemize}
\item 8-1/2 x 11 inches
\item High-quality, white
\item 20 pound weight, bond
\end{itemize}
 
While for Masters Theses, the paper should be:

\begin{itemize}
\item 8-1/2 x 11 inches
\item White
\item Acid-free or pH neutral
\item 20 pound weight
\item 25\% cotton bond minimum
\end{itemize}

Paper that meets these requirements can be purchased at book and stationery
stores.

\subsection{Copyright Page}
\label{copyright}
If you choose to retain and register copyright of the dissertation, prepare
a copyright page using the {\tt withesis} {\tt \verb|\copyrightpage|} command. 
Center the text in the bottom third of the page within the dissertation
margins. This page is not numbered. There is an additional fee for copyrighting
your dissertation which is payable at the bursars office along with the
microfilming and binding fee.

\subsection{Prechecks}
The Graduate School has reserved 9:00-9:30 each morning to answer specific formatting questions
(for example: use of tables, graphs and charts). You may bring in 8-10
pages to be reviewed. No appointment is necessary.

\subsection{Final Checks}
\sloppypar
For information about the final Graduate School review and about depositing
your dissertation in the library, see {\em The Three D's: Deadlines, Defending, 
Depositing Your Doctoral Dissertation} or look
at the web site 
\begin{quote}
{\tt http://www.wisc.edu/grad/gs/degrees/ddd.html}
\end{quote}

\section{Disclaimer}
This software and documentation is provided ``as is'' without any
express or implied warranty.
While care has been taken by the authors of this style file such that the
final product will probably meet the University of Wisconsin's formatting 
requirements this is not guaranteed. 
          % Chapter 1
\include{essentials}     % Edited ``Essential LaTeX'' by Jon Warbrick
\chapter{Figures and Tables}\label{quad}
This chapter\footnote{Most of the text in this chapter's introduction is from {\em How to
\TeX{} a Thesis: The Purdue Thesis Styles}} shows some example ways of incorporating tables and figures into \LaTeX{}.
Special environments exist for tables and figures and are special because they are
allowed to {\em float}---that is, \LaTeX{} doesn't always put them in the exact place
that they occur in your input file.  An algorithm is used to place the floating environments,
or floats, at locations which are typographically correct.  This may cause endless frustration
if you want to have a figure or table occur at a specific location.  There are a few
methods for solving this.

You can exert some influence on \LaTeX{}'s float placement algorithm by using
{\em float position specifiers}.  These specifiers, listed below, tell \LaTeX{}
what you prefer.
\begin{tabbing}
{\tt hhhhhh} \= ``bottom'' \=  \kill
{\tt h}\> ``here'' \> do not move this object \\
{\tt p}\> ``page'' \> put this object on a page of floats \\
{\tt b}\> ``bottom'' \> put this object at the bottom of a page\\
{\tt t}\> ``top'' \> put this object at the top of a page\\
\end{tabbing}

Any combination of these can be used:
\begin{quote}\tt\singlespace\begin{verbatim}
\begin{figure}[htbp]
 ...
\caption{A Figure!}
\end{figure}
\end{verbatim}\end{quote}

In this example, we asked \LaTeX{} to ``put the figure `here' if possible.  If it
is not possible (according to the rule encoded in the float algorithm), put it on the
next float page.  A float page is a page which contains nothing but floating objects,
{\em e.g.} a page of nothing but figures or tables.  If this isn't possible, try to put it
at the `top' of a page.  The last thing to try is to put the figure at the `bottom' of
a page.''

The remainder of this chapter deals with some examples of what to put into the figure,
the ellipsis (\ldots ) in the example above.

\section{Tables}
Table~\ref{pde.tab1} is an example table from the UW Math Department.
\begin{table}[htbp]
\centering
\caption{PDE solve times, $15^3+1$
equations.\label{pde.tab1}}
\begin{tabular}{||l|l|l|l|l|l||}\hline
Precond. & Time & Nonlinear & Krylov
& Function & Precond. \\
 & & Iterations & Iterations & calls & solves \\ \hline
None & 1260.9u & 3 & 26 & 30 & 0  \\
 &(21:09) & & & &  \\ \hline
FFT  & 983.4u & 2  & 5  & 8  & 7 \\
&(16:31) & & & & \\ \hline
\end{tabular}
\end{table}
The code to generate it is as follows:
\begin{quote}\tt\singlespace\begin{verbatim}
\begin{table}[htbp]
\centering
\caption{PDE solve times, $15^3+1$
equations.\label{pde.tab1}}
\begin{tabular}{||l|l|l|l|l|l||}\hline
Precond. & Time & Nonlinear & Krylov
& Function & Precond. \\
 & & Iterations & Iterations & calls & solves \\ \hline
None & 1260.9u & 3 & 26 & 30 & 0  \\
 &(21:09) & & & &  \\ \hline
FFT  & 983.4u & 2  & 5  & 8  & 7 \\
&(16:31) & & & & \\ \hline
\end{tabular}
\end{table}
\end{verbatim}\end{quote}

\section{Figures}
There are many different ways to incorporate figures into a \LaTeX{}
document.  \LaTeX{} has an internal {\tt picture} environment and
some programs will generate files which are in this format and can
be simply {\tt include}d.  In addition to \LaTeX{} native {\tt picture}
format, additional packages can be loaded in the {\tt\verb|\documentstyle|}
command (or using the {\tt input} command) to allow \LaTeX{} to process
non-native formats such as PostScript.

\subsection{\tt gnuplot}
The graph of Figure~\ref{gelfand.fig2}
 was created by gnuplot. For simple graphs this is a
 great utility.  For example, if you want a sin curve in your thesis
 try the following:
\begin{quote}\tt\singlespace\begin{verbatim}
 (terminal window): gnuplot
 (in gnuplot):
                 set terminal latex
                 set output "foo.tex"
                 plot sin(x)
                 quit
\end{verbatim}\end{quote}
This will generate a file called {\tt foo.tex} which can be read in
with the following statements.
\begin{figure}[htbp]
\centering
% GNUPLOT: LaTeX picture
\setlength{\unitlength}{0.240900pt}
\ifx\plotpoint\undefined\newsavebox{\plotpoint}\fi
\sbox{\plotpoint}{\rule[-0.175pt]{0.350pt}{0.350pt}}%
\begin{picture}(1500,900)(0,0)
%\tenrm
\sbox{\plotpoint}{\rule[-0.175pt]{0.350pt}{0.350pt}}%
\put(264,158){\rule[-0.175pt]{282.335pt}{0.350pt}}
\put(264,158){\rule[-0.175pt]{0.350pt}{151.526pt}}
\put(264,158){\rule[-0.175pt]{4.818pt}{0.350pt}}
%\put(242,158){\makebox(0,0)[r]{0}}
\put(1416,158){\rule[-0.175pt]{4.818pt}{0.350pt}}
\put(264,284){\rule[-0.175pt]{4.818pt}{0.350pt}}
%\put(242,284){\makebox(0,0)[r]{2}}
\put(1416,284){\rule[-0.175pt]{4.818pt}{0.350pt}}
\put(264,410){\rule[-0.175pt]{4.818pt}{0.350pt}}
%\put(242,410){\makebox(0,0)[r]{4}}
\put(1416,410){\rule[-0.175pt]{4.818pt}{0.350pt}}
\put(264,535){\rule[-0.175pt]{4.818pt}{0.350pt}}
%\put(242,535){\makebox(0,0)[r]{6}}
\put(1416,535){\rule[-0.175pt]{4.818pt}{0.350pt}}
\put(264,661){\rule[-0.175pt]{4.818pt}{0.350pt}}
%\put(242,661){\makebox(0,0)[r]{8}}
\put(1416,661){\rule[-0.175pt]{4.818pt}{0.350pt}}
\put(264,787){\rule[-0.175pt]{4.818pt}{0.350pt}}
%\put(242,787){\makebox(0,0)[r]{10}}
\put(1416,787){\rule[-0.175pt]{4.818pt}{0.350pt}}
\put(264,158){\rule[-0.175pt]{0.350pt}{4.818pt}}
%\put(264,113){\makebox(0,0){0}}
\put(264,767){\rule[-0.175pt]{0.350pt}{4.818pt}}
\put(411,158){\rule[-0.175pt]{0.350pt}{4.818pt}}
%\put(411,113){\makebox(0,0){0.5}}
\put(411,767){\rule[-0.175pt]{0.350pt}{4.818pt}}
\put(557,158){\rule[-0.175pt]{0.350pt}{4.818pt}}
%\put(557,113){\makebox(0,0){1}}
\put(557,767){\rule[-0.175pt]{0.350pt}{4.818pt}}
\put(704,158){\rule[-0.175pt]{0.350pt}{4.818pt}}
%\put(704,113){\makebox(0,0){1.5}}
\put(704,767){\rule[-0.175pt]{0.350pt}{4.818pt}}
\put(850,158){\rule[-0.175pt]{0.350pt}{4.818pt}}
%\put(850,113){\makebox(0,0){2}}
\put(850,767){\rule[-0.175pt]{0.350pt}{4.818pt}}
\put(997,158){\rule[-0.175pt]{0.350pt}{4.818pt}}
%\put(997,113){\makebox(0,0){2.5}}
\put(997,767){\rule[-0.175pt]{0.350pt}{4.818pt}}
\put(1143,158){\rule[-0.175pt]{0.350pt}{4.818pt}}
%\put(1143,113){\makebox(0,0){3}}
\put(1143,767){\rule[-0.175pt]{0.350pt}{4.818pt}}
\put(1290,158){\rule[-0.175pt]{0.350pt}{4.818pt}}
%\put(1290,113){\makebox(0,0){3.5}}
\put(1290,767){\rule[-0.175pt]{0.350pt}{4.818pt}}
\put(1436,158){\rule[-0.175pt]{0.350pt}{4.818pt}}
%\put(1436,113){\makebox(0,0){4}}
\put(1436,767){\rule[-0.175pt]{0.350pt}{4.818pt}}
\put(264,158){\rule[-0.175pt]{282.335pt}{0.350pt}}
\put(1436,158){\rule[-0.175pt]{0.350pt}{151.526pt}}
\put(264,787){\rule[-0.175pt]{282.335pt}{0.350pt}}
\put(100,472){\makebox(0,0)[l]{\shortstack{$\| u\|$}}}
\put(850,68){\makebox(0,0){$\lambda$}}
%\put(850,832){\makebox(0,0){plot}}
\put(264,158){\rule[-0.175pt]{0.350pt}{151.526pt}}
%\put(1306,722){\makebox(0,0)[r]{}}
%\put(1328,722){\rule[-0.175pt]{15.899pt}{0.350pt}}
\put(264,158){\usebox{\plotpoint}}
\put(264,158){\rule[-0.175pt]{6.304pt}{0.350pt}}
\put(290,159){\rule[-0.175pt]{6.304pt}{0.350pt}}
\put(316,160){\rule[-0.175pt]{6.304pt}{0.350pt}}
\put(342,161){\rule[-0.175pt]{6.304pt}{0.350pt}}
\put(368,162){\rule[-0.175pt]{6.304pt}{0.350pt}}
\put(394,163){\rule[-0.175pt]{6.304pt}{0.350pt}}
\put(420,164){\rule[-0.175pt]{5.644pt}{0.350pt}}
\put(444,165){\rule[-0.175pt]{5.644pt}{0.350pt}}
\put(467,166){\rule[-0.175pt]{5.644pt}{0.350pt}}
\put(491,167){\rule[-0.175pt]{5.644pt}{0.350pt}}
\put(514,168){\rule[-0.175pt]{5.644pt}{0.350pt}}
\put(538,169){\rule[-0.175pt]{5.644pt}{0.350pt}}
\put(561,170){\rule[-0.175pt]{5.644pt}{0.350pt}}
\put(585,171){\rule[-0.175pt]{6.384pt}{0.350pt}}
\put(611,172){\rule[-0.175pt]{6.384pt}{0.350pt}}
\put(638,173){\rule[-0.175pt]{6.384pt}{0.350pt}}
\put(664,174){\rule[-0.175pt]{6.384pt}{0.350pt}}
\put(691,175){\rule[-0.175pt]{6.384pt}{0.350pt}}
\put(717,176){\rule[-0.175pt]{6.384pt}{0.350pt}}
\put(744,177){\rule[-0.175pt]{5.862pt}{0.350pt}}
\put(768,178){\rule[-0.175pt]{5.862pt}{0.350pt}}
\put(792,179){\rule[-0.175pt]{5.862pt}{0.350pt}}
\put(816,180){\rule[-0.175pt]{5.862pt}{0.350pt}}
\put(841,181){\rule[-0.175pt]{5.862pt}{0.350pt}}
\put(865,182){\rule[-0.175pt]{5.862pt}{0.350pt}}
\put(889,183){\rule[-0.175pt]{4.371pt}{0.350pt}}
\put(908,184){\rule[-0.175pt]{4.371pt}{0.350pt}}
\put(926,185){\rule[-0.175pt]{4.371pt}{0.350pt}}
\put(944,186){\rule[-0.175pt]{4.371pt}{0.350pt}}
\put(962,187){\rule[-0.175pt]{4.371pt}{0.350pt}}
\put(980,188){\rule[-0.175pt]{4.371pt}{0.350pt}}
\put(998,189){\rule[-0.175pt]{4.371pt}{0.350pt}}
\put(1017,190){\rule[-0.175pt]{4.216pt}{0.350pt}}
\put(1034,191){\rule[-0.175pt]{4.216pt}{0.350pt}}
\put(1052,192){\rule[-0.175pt]{4.216pt}{0.350pt}}
\put(1069,193){\rule[-0.175pt]{4.216pt}{0.350pt}}
\put(1087,194){\rule[-0.175pt]{4.216pt}{0.350pt}}
\put(1104,195){\rule[-0.175pt]{4.216pt}{0.350pt}}
\put(1122,196){\rule[-0.175pt]{3.172pt}{0.350pt}}
\put(1135,197){\rule[-0.175pt]{3.172pt}{0.350pt}}
\put(1148,198){\rule[-0.175pt]{3.172pt}{0.350pt}}
\put(1161,199){\rule[-0.175pt]{3.172pt}{0.350pt}}
\put(1174,200){\rule[-0.175pt]{3.172pt}{0.350pt}}
\put(1187,201){\rule[-0.175pt]{3.172pt}{0.350pt}}
\put(1200,202){\rule[-0.175pt]{1.893pt}{0.350pt}}
\put(1208,203){\rule[-0.175pt]{1.893pt}{0.350pt}}
\put(1216,204){\rule[-0.175pt]{1.893pt}{0.350pt}}
\put(1224,205){\rule[-0.175pt]{1.893pt}{0.350pt}}
\put(1232,206){\rule[-0.175pt]{1.893pt}{0.350pt}}
\put(1240,207){\rule[-0.175pt]{1.893pt}{0.350pt}}
\put(1248,208){\rule[-0.175pt]{1.893pt}{0.350pt}}
\put(1256,209){\rule[-0.175pt]{1.245pt}{0.350pt}}
\put(1261,210){\rule[-0.175pt]{1.245pt}{0.350pt}}
\put(1266,211){\rule[-0.175pt]{1.245pt}{0.350pt}}
\put(1271,212){\rule[-0.175pt]{1.245pt}{0.350pt}}
\put(1276,213){\rule[-0.175pt]{1.245pt}{0.350pt}}
\put(1281,214){\rule[-0.175pt]{1.245pt}{0.350pt}}
\put(1286,215){\usebox{\plotpoint}}
\put(1288,216){\usebox{\plotpoint}}
\put(1289,217){\usebox{\plotpoint}}
\put(1291,218){\usebox{\plotpoint}}
\put(1292,219){\usebox{\plotpoint}}
\put(1294,220){\usebox{\plotpoint}}
\put(1295,221){\usebox{\plotpoint}}
\put(1295,222){\rule[-0.175pt]{0.361pt}{0.350pt}}
\put(1294,223){\rule[-0.175pt]{0.361pt}{0.350pt}}
\put(1292,224){\rule[-0.175pt]{0.361pt}{0.350pt}}
\put(1291,225){\rule[-0.175pt]{0.361pt}{0.350pt}}
\put(1289,226){\rule[-0.175pt]{0.361pt}{0.350pt}}
\put(1288,227){\rule[-0.175pt]{0.361pt}{0.350pt}}
\put(1284,228){\rule[-0.175pt]{0.964pt}{0.350pt}}
\put(1280,229){\rule[-0.175pt]{0.964pt}{0.350pt}}
\put(1276,230){\rule[-0.175pt]{0.964pt}{0.350pt}}
\put(1272,231){\rule[-0.175pt]{0.964pt}{0.350pt}}
\put(1268,232){\rule[-0.175pt]{0.964pt}{0.350pt}}
\put(1264,233){\rule[-0.175pt]{0.964pt}{0.350pt}}
\put(1258,234){\rule[-0.175pt]{1.273pt}{0.350pt}}
\put(1253,235){\rule[-0.175pt]{1.273pt}{0.350pt}}
\put(1248,236){\rule[-0.175pt]{1.273pt}{0.350pt}}
\put(1242,237){\rule[-0.175pt]{1.273pt}{0.350pt}}
\put(1237,238){\rule[-0.175pt]{1.273pt}{0.350pt}}
\put(1232,239){\rule[-0.175pt]{1.273pt}{0.350pt}}
\put(1227,240){\rule[-0.175pt]{1.273pt}{0.350pt}}
\put(1219,241){\rule[-0.175pt]{1.847pt}{0.350pt}}
\put(1211,242){\rule[-0.175pt]{1.847pt}{0.350pt}}
\put(1204,243){\rule[-0.175pt]{1.847pt}{0.350pt}}
\put(1196,244){\rule[-0.175pt]{1.847pt}{0.350pt}}
\put(1188,245){\rule[-0.175pt]{1.847pt}{0.350pt}}
\put(1181,246){\rule[-0.175pt]{1.847pt}{0.350pt}}
\put(1172,247){\rule[-0.175pt]{2.128pt}{0.350pt}}
\put(1163,248){\rule[-0.175pt]{2.128pt}{0.350pt}}
\put(1154,249){\rule[-0.175pt]{2.128pt}{0.350pt}}
\put(1145,250){\rule[-0.175pt]{2.128pt}{0.350pt}}
\put(1136,251){\rule[-0.175pt]{2.128pt}{0.350pt}}
\put(1128,252){\rule[-0.175pt]{2.128pt}{0.350pt}}
\put(1120,253){\rule[-0.175pt]{1.893pt}{0.350pt}}
\put(1112,254){\rule[-0.175pt]{1.893pt}{0.350pt}}
\put(1104,255){\rule[-0.175pt]{1.893pt}{0.350pt}}
\put(1096,256){\rule[-0.175pt]{1.893pt}{0.350pt}}
\put(1088,257){\rule[-0.175pt]{1.893pt}{0.350pt}}
\put(1080,258){\rule[-0.175pt]{1.893pt}{0.350pt}}
\put(1073,259){\rule[-0.175pt]{1.893pt}{0.350pt}}
\put(1063,260){\rule[-0.175pt]{2.208pt}{0.350pt}}
\put(1054,261){\rule[-0.175pt]{2.208pt}{0.350pt}}
\put(1045,262){\rule[-0.175pt]{2.208pt}{0.350pt}}
\put(1036,263){\rule[-0.175pt]{2.208pt}{0.350pt}}
\put(1027,264){\rule[-0.175pt]{2.208pt}{0.350pt}}
\put(1018,265){\rule[-0.175pt]{2.208pt}{0.350pt}}
\put(1009,266){\rule[-0.175pt]{2.168pt}{0.350pt}}
\put(1000,267){\rule[-0.175pt]{2.168pt}{0.350pt}}
\put(991,268){\rule[-0.175pt]{2.168pt}{0.350pt}}
\put(982,269){\rule[-0.175pt]{2.168pt}{0.350pt}}
\put(973,270){\rule[-0.175pt]{2.168pt}{0.350pt}}
\put(964,271){\rule[-0.175pt]{2.168pt}{0.350pt}}
\put(957,272){\rule[-0.175pt]{1.686pt}{0.350pt}}
\put(950,273){\rule[-0.175pt]{1.686pt}{0.350pt}}
\put(943,274){\rule[-0.175pt]{1.686pt}{0.350pt}}
\put(936,275){\rule[-0.175pt]{1.686pt}{0.350pt}}
\put(929,276){\rule[-0.175pt]{1.686pt}{0.350pt}}
\put(922,277){\rule[-0.175pt]{1.686pt}{0.350pt}}
\put(915,278){\rule[-0.175pt]{1.686pt}{0.350pt}}
\put(907,279){\rule[-0.175pt]{1.767pt}{0.350pt}}
\put(900,280){\rule[-0.175pt]{1.767pt}{0.350pt}}
\put(893,281){\rule[-0.175pt]{1.767pt}{0.350pt}}
\put(885,282){\rule[-0.175pt]{1.767pt}{0.350pt}}
\put(878,283){\rule[-0.175pt]{1.767pt}{0.350pt}}
\put(871,284){\rule[-0.175pt]{1.767pt}{0.350pt}}
\put(864,285){\rule[-0.175pt]{1.486pt}{0.350pt}}
\put(858,286){\rule[-0.175pt]{1.486pt}{0.350pt}}
\put(852,287){\rule[-0.175pt]{1.486pt}{0.350pt}}
\put(846,288){\rule[-0.175pt]{1.486pt}{0.350pt}}
\put(840,289){\rule[-0.175pt]{1.486pt}{0.350pt}}
\put(834,290){\rule[-0.175pt]{1.486pt}{0.350pt}}
\put(829,291){\rule[-0.175pt]{0.998pt}{0.350pt}}
\put(825,292){\rule[-0.175pt]{0.998pt}{0.350pt}}
\put(821,293){\rule[-0.175pt]{0.998pt}{0.350pt}}
\put(817,294){\rule[-0.175pt]{0.998pt}{0.350pt}}
\put(813,295){\rule[-0.175pt]{0.998pt}{0.350pt}}
\put(809,296){\rule[-0.175pt]{0.998pt}{0.350pt}}
\put(805,297){\rule[-0.175pt]{0.998pt}{0.350pt}}
\put(801,298){\rule[-0.175pt]{0.883pt}{0.350pt}}
\put(797,299){\rule[-0.175pt]{0.883pt}{0.350pt}}
\put(793,300){\rule[-0.175pt]{0.883pt}{0.350pt}}
\put(790,301){\rule[-0.175pt]{0.883pt}{0.350pt}}
\put(786,302){\rule[-0.175pt]{0.883pt}{0.350pt}}
\put(783,303){\rule[-0.175pt]{0.883pt}{0.350pt}}
\put(780,304){\rule[-0.175pt]{0.522pt}{0.350pt}}
\put(778,305){\rule[-0.175pt]{0.522pt}{0.350pt}}
\put(776,306){\rule[-0.175pt]{0.522pt}{0.350pt}}
\put(774,307){\rule[-0.175pt]{0.522pt}{0.350pt}}
\put(772,308){\rule[-0.175pt]{0.522pt}{0.350pt}}
\put(770,309){\rule[-0.175pt]{0.522pt}{0.350pt}}
\put(770,310){\usebox{\plotpoint}}
\put(769,311){\usebox{\plotpoint}}
\put(768,312){\usebox{\plotpoint}}
\put(767,314){\usebox{\plotpoint}}
\put(766,315){\usebox{\plotpoint}}
\put(765,316){\rule[-0.175pt]{0.350pt}{0.723pt}}
\put(766,320){\rule[-0.175pt]{0.350pt}{0.723pt}}
\put(767,323){\usebox{\plotpoint}}
\put(768,324){\usebox{\plotpoint}}
\put(769,325){\usebox{\plotpoint}}
\put(771,326){\usebox{\plotpoint}}
\put(772,327){\usebox{\plotpoint}}
\put(774,328){\usebox{\plotpoint}}
\put(775,329){\usebox{\plotpoint}}
\put(777,330){\rule[-0.175pt]{0.602pt}{0.350pt}}
\put(779,331){\rule[-0.175pt]{0.602pt}{0.350pt}}
\put(782,332){\rule[-0.175pt]{0.602pt}{0.350pt}}
\put(784,333){\rule[-0.175pt]{0.602pt}{0.350pt}}
\put(787,334){\rule[-0.175pt]{0.602pt}{0.350pt}}
\put(789,335){\rule[-0.175pt]{0.602pt}{0.350pt}}
\put(792,336){\rule[-0.175pt]{0.843pt}{0.350pt}}
\put(795,337){\rule[-0.175pt]{0.843pt}{0.350pt}}
\put(799,338){\rule[-0.175pt]{0.843pt}{0.350pt}}
\put(802,339){\rule[-0.175pt]{0.843pt}{0.350pt}}
\put(806,340){\rule[-0.175pt]{0.843pt}{0.350pt}}
\put(809,341){\rule[-0.175pt]{0.843pt}{0.350pt}}
\put(813,342){\rule[-0.175pt]{0.826pt}{0.350pt}}
\put(816,343){\rule[-0.175pt]{0.826pt}{0.350pt}}
\put(819,344){\rule[-0.175pt]{0.826pt}{0.350pt}}
\put(823,345){\rule[-0.175pt]{0.826pt}{0.350pt}}
\put(826,346){\rule[-0.175pt]{0.826pt}{0.350pt}}
\put(830,347){\rule[-0.175pt]{0.826pt}{0.350pt}}
\put(833,348){\rule[-0.175pt]{0.826pt}{0.350pt}}
\put(837,349){\rule[-0.175pt]{1.084pt}{0.350pt}}
\put(841,350){\rule[-0.175pt]{1.084pt}{0.350pt}}
\put(846,351){\rule[-0.175pt]{1.084pt}{0.350pt}}
\put(850,352){\rule[-0.175pt]{1.084pt}{0.350pt}}
\put(855,353){\rule[-0.175pt]{1.084pt}{0.350pt}}
\put(859,354){\rule[-0.175pt]{1.084pt}{0.350pt}}
\put(864,355){\rule[-0.175pt]{1.164pt}{0.350pt}}
\put(868,356){\rule[-0.175pt]{1.164pt}{0.350pt}}
\put(873,357){\rule[-0.175pt]{1.164pt}{0.350pt}}
\put(878,358){\rule[-0.175pt]{1.164pt}{0.350pt}}
\put(883,359){\rule[-0.175pt]{1.164pt}{0.350pt}}
\put(888,360){\rule[-0.175pt]{1.164pt}{0.350pt}}
\put(892,361){\rule[-0.175pt]{1.032pt}{0.350pt}}
\put(897,362){\rule[-0.175pt]{1.032pt}{0.350pt}}
\put(901,363){\rule[-0.175pt]{1.032pt}{0.350pt}}
\put(905,364){\rule[-0.175pt]{1.032pt}{0.350pt}}
\put(910,365){\rule[-0.175pt]{1.032pt}{0.350pt}}
\put(914,366){\rule[-0.175pt]{1.032pt}{0.350pt}}
\put(918,367){\rule[-0.175pt]{1.032pt}{0.350pt}}
\put(922,368){\rule[-0.175pt]{1.205pt}{0.350pt}}
\put(928,369){\rule[-0.175pt]{1.204pt}{0.350pt}}
\put(933,370){\rule[-0.175pt]{1.204pt}{0.350pt}}
\put(938,371){\rule[-0.175pt]{1.204pt}{0.350pt}}
\put(943,372){\rule[-0.175pt]{1.204pt}{0.350pt}}
\put(948,373){\rule[-0.175pt]{1.204pt}{0.350pt}}
\put(953,374){\rule[-0.175pt]{1.124pt}{0.350pt}}
\put(957,375){\rule[-0.175pt]{1.124pt}{0.350pt}}
\put(962,376){\rule[-0.175pt]{1.124pt}{0.350pt}}
\put(967,377){\rule[-0.175pt]{1.124pt}{0.350pt}}
\put(971,378){\rule[-0.175pt]{1.124pt}{0.350pt}}
\put(976,379){\rule[-0.175pt]{1.124pt}{0.350pt}}
\put(981,380){\rule[-0.175pt]{0.929pt}{0.350pt}}
\put(984,381){\rule[-0.175pt]{0.929pt}{0.350pt}}
\put(988,382){\rule[-0.175pt]{0.929pt}{0.350pt}}
\put(992,383){\rule[-0.175pt]{0.929pt}{0.350pt}}
\put(996,384){\rule[-0.175pt]{0.929pt}{0.350pt}}
\put(1000,385){\rule[-0.175pt]{0.929pt}{0.350pt}}
\put(1004,386){\rule[-0.175pt]{0.929pt}{0.350pt}}
\put(1007,387){\rule[-0.175pt]{0.923pt}{0.350pt}}
\put(1011,388){\rule[-0.175pt]{0.923pt}{0.350pt}}
\put(1015,389){\rule[-0.175pt]{0.923pt}{0.350pt}}
\put(1019,390){\rule[-0.175pt]{0.923pt}{0.350pt}}
\put(1023,391){\rule[-0.175pt]{0.923pt}{0.350pt}}
\put(1027,392){\rule[-0.175pt]{0.923pt}{0.350pt}}
\put(1031,393){\rule[-0.175pt]{0.843pt}{0.350pt}}
\put(1034,394){\rule[-0.175pt]{0.843pt}{0.350pt}}
\put(1038,395){\rule[-0.175pt]{0.843pt}{0.350pt}}
\put(1041,396){\rule[-0.175pt]{0.843pt}{0.350pt}}
\put(1045,397){\rule[-0.175pt]{0.843pt}{0.350pt}}
\put(1048,398){\rule[-0.175pt]{0.843pt}{0.350pt}}
\put(1052,399){\rule[-0.175pt]{0.585pt}{0.350pt}}
\put(1054,400){\rule[-0.175pt]{0.585pt}{0.350pt}}
\put(1056,401){\rule[-0.175pt]{0.585pt}{0.350pt}}
\put(1059,402){\rule[-0.175pt]{0.585pt}{0.350pt}}
\put(1061,403){\rule[-0.175pt]{0.585pt}{0.350pt}}
\put(1064,404){\rule[-0.175pt]{0.585pt}{0.350pt}}
\put(1066,405){\rule[-0.175pt]{0.585pt}{0.350pt}}
\put(1069,406){\rule[-0.175pt]{0.522pt}{0.350pt}}
\put(1071,407){\rule[-0.175pt]{0.522pt}{0.350pt}}
\put(1073,408){\rule[-0.175pt]{0.522pt}{0.350pt}}
\put(1075,409){\rule[-0.175pt]{0.522pt}{0.350pt}}
\put(1077,410){\rule[-0.175pt]{0.522pt}{0.350pt}}
\put(1079,411){\rule[-0.175pt]{0.522pt}{0.350pt}}
\put(1081,412){\rule[-0.175pt]{0.402pt}{0.350pt}}
\put(1083,413){\rule[-0.175pt]{0.401pt}{0.350pt}}
\put(1085,414){\rule[-0.175pt]{0.401pt}{0.350pt}}
\put(1086,415){\rule[-0.175pt]{0.401pt}{0.350pt}}
\put(1088,416){\rule[-0.175pt]{0.401pt}{0.350pt}}
\put(1090,417){\rule[-0.175pt]{0.401pt}{0.350pt}}
\put(1091,418){\usebox{\plotpoint}}
\put(1092,418){\usebox{\plotpoint}}
\put(1093,419){\usebox{\plotpoint}}
\put(1094,420){\usebox{\plotpoint}}
\put(1095,422){\usebox{\plotpoint}}
\put(1096,423){\usebox{\plotpoint}}
\put(1097,424){\rule[-0.175pt]{0.350pt}{0.723pt}}
\put(1098,428){\rule[-0.175pt]{0.350pt}{0.723pt}}
\put(1099,431){\rule[-0.175pt]{0.350pt}{1.686pt}}
\put(1098,438){\usebox{\plotpoint}}
\put(1097,439){\usebox{\plotpoint}}
\put(1096,440){\usebox{\plotpoint}}
\put(1095,441){\usebox{\plotpoint}}
\put(1094,442){\usebox{\plotpoint}}
\put(1091,444){\usebox{\plotpoint}}
\put(1090,445){\usebox{\plotpoint}}
\put(1089,446){\usebox{\plotpoint}}
\put(1088,447){\usebox{\plotpoint}}
\put(1087,448){\usebox{\plotpoint}}
\put(1086,449){\usebox{\plotpoint}}
\put(1084,450){\usebox{\plotpoint}}
\put(1083,451){\usebox{\plotpoint}}
\put(1081,452){\usebox{\plotpoint}}
\put(1080,453){\usebox{\plotpoint}}
\put(1078,454){\usebox{\plotpoint}}
\put(1077,455){\usebox{\plotpoint}}
\put(1076,456){\usebox{\plotpoint}}
\put(1074,457){\rule[-0.175pt]{0.442pt}{0.350pt}}
\put(1072,458){\rule[-0.175pt]{0.442pt}{0.350pt}}
\put(1070,459){\rule[-0.175pt]{0.442pt}{0.350pt}}
\put(1068,460){\rule[-0.175pt]{0.442pt}{0.350pt}}
\put(1066,461){\rule[-0.175pt]{0.442pt}{0.350pt}}
\put(1065,462){\rule[-0.175pt]{0.442pt}{0.350pt}}
\put(1063,463){\rule[-0.175pt]{0.482pt}{0.350pt}}
\put(1061,464){\rule[-0.175pt]{0.482pt}{0.350pt}}
\put(1059,465){\rule[-0.175pt]{0.482pt}{0.350pt}}
\put(1057,466){\rule[-0.175pt]{0.482pt}{0.350pt}}
\put(1055,467){\rule[-0.175pt]{0.482pt}{0.350pt}}
\put(1053,468){\rule[-0.175pt]{0.482pt}{0.350pt}}
\put(1051,469){\rule[-0.175pt]{0.482pt}{0.350pt}}
\put(1049,470){\rule[-0.175pt]{0.482pt}{0.350pt}}
\put(1047,471){\rule[-0.175pt]{0.482pt}{0.350pt}}
\put(1045,472){\rule[-0.175pt]{0.482pt}{0.350pt}}
\put(1043,473){\rule[-0.175pt]{0.482pt}{0.350pt}}
\put(1041,474){\rule[-0.175pt]{0.482pt}{0.350pt}}
\put(1039,475){\rule[-0.175pt]{0.482pt}{0.350pt}}
\put(1036,476){\rule[-0.175pt]{0.522pt}{0.350pt}}
\put(1034,477){\rule[-0.175pt]{0.522pt}{0.350pt}}
\put(1032,478){\rule[-0.175pt]{0.522pt}{0.350pt}}
\put(1030,479){\rule[-0.175pt]{0.522pt}{0.350pt}}
\put(1028,480){\rule[-0.175pt]{0.522pt}{0.350pt}}
\put(1026,481){\rule[-0.175pt]{0.522pt}{0.350pt}}
\put(1023,482){\rule[-0.175pt]{0.522pt}{0.350pt}}
\put(1021,483){\rule[-0.175pt]{0.522pt}{0.350pt}}
\put(1019,484){\rule[-0.175pt]{0.522pt}{0.350pt}}
\put(1017,485){\rule[-0.175pt]{0.522pt}{0.350pt}}
\put(1015,486){\rule[-0.175pt]{0.522pt}{0.350pt}}
\put(1013,487){\rule[-0.175pt]{0.522pt}{0.350pt}}
\put(1011,488){\rule[-0.175pt]{0.447pt}{0.350pt}}
\put(1009,489){\rule[-0.175pt]{0.447pt}{0.350pt}}
\put(1007,490){\rule[-0.175pt]{0.447pt}{0.350pt}}
\put(1005,491){\rule[-0.175pt]{0.447pt}{0.350pt}}
\put(1003,492){\rule[-0.175pt]{0.447pt}{0.350pt}}
\put(1001,493){\rule[-0.175pt]{0.447pt}{0.350pt}}
\put(1000,494){\rule[-0.175pt]{0.447pt}{0.350pt}}
\put(998,495){\rule[-0.175pt]{0.442pt}{0.350pt}}
\put(996,496){\rule[-0.175pt]{0.442pt}{0.350pt}}
\put(994,497){\rule[-0.175pt]{0.442pt}{0.350pt}}
\put(992,498){\rule[-0.175pt]{0.442pt}{0.350pt}}
\put(990,499){\rule[-0.175pt]{0.442pt}{0.350pt}}
\put(989,500){\rule[-0.175pt]{0.442pt}{0.350pt}}
\put(987,501){\rule[-0.175pt]{0.442pt}{0.350pt}}
\put(985,502){\rule[-0.175pt]{0.442pt}{0.350pt}}
\put(983,503){\rule[-0.175pt]{0.442pt}{0.350pt}}
\put(981,504){\rule[-0.175pt]{0.442pt}{0.350pt}}
\put(979,505){\rule[-0.175pt]{0.442pt}{0.350pt}}
\put(978,506){\rule[-0.175pt]{0.442pt}{0.350pt}}
\put(976,507){\usebox{\plotpoint}}
\put(975,508){\usebox{\plotpoint}}
\put(974,509){\usebox{\plotpoint}}
\put(972,510){\usebox{\plotpoint}}
\put(971,511){\usebox{\plotpoint}}
\put(970,512){\usebox{\plotpoint}}
\put(969,513){\usebox{\plotpoint}}
\put(967,514){\usebox{\plotpoint}}
\put(966,515){\usebox{\plotpoint}}
\put(965,516){\usebox{\plotpoint}}
\put(964,517){\usebox{\plotpoint}}
\put(963,518){\usebox{\plotpoint}}
\put(962,519){\usebox{\plotpoint}}
\put(962,520){\usebox{\plotpoint}}
\put(961,521){\usebox{\plotpoint}}
\put(960,522){\usebox{\plotpoint}}
\put(959,524){\usebox{\plotpoint}}
\put(958,525){\usebox{\plotpoint}}
\put(957,527){\rule[-0.175pt]{0.350pt}{0.361pt}}
\put(956,528){\rule[-0.175pt]{0.350pt}{0.361pt}}
\put(955,530){\rule[-0.175pt]{0.350pt}{0.361pt}}
\put(954,531){\rule[-0.175pt]{0.350pt}{0.361pt}}
\put(953,533){\rule[-0.175pt]{0.350pt}{0.723pt}}
\put(952,536){\rule[-0.175pt]{0.350pt}{0.723pt}}
\put(951,539){\rule[-0.175pt]{0.350pt}{1.686pt}}
\put(950,546){\rule[-0.175pt]{0.350pt}{1.445pt}}
\put(951,552){\rule[-0.175pt]{0.350pt}{0.482pt}}
\put(952,554){\rule[-0.175pt]{0.350pt}{0.482pt}}
\put(953,556){\rule[-0.175pt]{0.350pt}{0.482pt}}
\put(954,558){\rule[-0.175pt]{0.350pt}{0.562pt}}
\put(955,560){\rule[-0.175pt]{0.350pt}{0.562pt}}
\put(956,562){\rule[-0.175pt]{0.350pt}{0.562pt}}
\put(957,564){\usebox{\plotpoint}}
\put(958,566){\usebox{\plotpoint}}
\put(959,567){\usebox{\plotpoint}}
\put(960,568){\usebox{\plotpoint}}
\put(961,569){\usebox{\plotpoint}}
\put(962,571){\usebox{\plotpoint}}
\put(963,572){\usebox{\plotpoint}}
\put(964,573){\usebox{\plotpoint}}
\put(965,574){\usebox{\plotpoint}}
\put(966,575){\usebox{\plotpoint}}
\put(967,577){\usebox{\plotpoint}}
\put(968,578){\usebox{\plotpoint}}
\put(969,579){\usebox{\plotpoint}}
\put(970,580){\usebox{\plotpoint}}
\put(971,581){\usebox{\plotpoint}}
\put(972,582){\usebox{\plotpoint}}
\put(973,584){\usebox{\plotpoint}}
\put(974,585){\usebox{\plotpoint}}
\put(975,586){\usebox{\plotpoint}}
\put(976,587){\usebox{\plotpoint}}
\put(977,588){\usebox{\plotpoint}}
\put(978,589){\usebox{\plotpoint}}
\put(979,590){\usebox{\plotpoint}}
\put(980,591){\usebox{\plotpoint}}
\put(981,592){\usebox{\plotpoint}}
\put(982,593){\usebox{\plotpoint}}
\put(983,594){\usebox{\plotpoint}}
\put(984,595){\usebox{\plotpoint}}
\put(985,596){\usebox{\plotpoint}}
\put(986,597){\usebox{\plotpoint}}
\put(987,598){\usebox{\plotpoint}}
\put(988,600){\usebox{\plotpoint}}
\put(989,601){\usebox{\plotpoint}}
\put(990,603){\usebox{\plotpoint}}
\put(991,604){\usebox{\plotpoint}}
\put(992,605){\usebox{\plotpoint}}
\put(993,606){\usebox{\plotpoint}}
\put(994,607){\usebox{\plotpoint}}
\put(995,608){\usebox{\plotpoint}}
\put(996,609){\usebox{\plotpoint}}
\put(997,610){\usebox{\plotpoint}}
\put(998,611){\usebox{\plotpoint}}
\put(999,612){\usebox{\plotpoint}}
\put(1000,613){\usebox{\plotpoint}}
\put(1001,615){\usebox{\plotpoint}}
\put(1002,616){\usebox{\plotpoint}}
\put(1003,617){\usebox{\plotpoint}}
\put(1004,619){\usebox{\plotpoint}}
\put(1005,620){\usebox{\plotpoint}}
\put(1006,622){\rule[-0.175pt]{0.350pt}{0.361pt}}
\put(1007,623){\rule[-0.175pt]{0.350pt}{0.361pt}}
\put(1008,625){\rule[-0.175pt]{0.350pt}{0.361pt}}
\put(1009,626){\rule[-0.175pt]{0.350pt}{0.361pt}}
\put(1010,628){\rule[-0.175pt]{0.350pt}{0.562pt}}
\put(1011,630){\rule[-0.175pt]{0.350pt}{0.562pt}}
\put(1012,632){\rule[-0.175pt]{0.350pt}{0.562pt}}
\put(1013,634){\rule[-0.175pt]{0.350pt}{0.723pt}}
\put(1014,638){\rule[-0.175pt]{0.350pt}{0.723pt}}
\put(1015,641){\rule[-0.175pt]{0.350pt}{1.445pt}}
\put(1016,647){\rule[-0.175pt]{0.350pt}{1.686pt}}
\put(1017,654){\rule[-0.175pt]{0.350pt}{3.734pt}}
\put(1016,669){\rule[-0.175pt]{0.350pt}{0.843pt}}
\put(1015,673){\rule[-0.175pt]{0.350pt}{1.445pt}}
\put(1014,679){\rule[-0.175pt]{0.350pt}{0.723pt}}
\put(1013,682){\rule[-0.175pt]{0.350pt}{0.723pt}}
\put(1012,685){\rule[-0.175pt]{0.350pt}{0.562pt}}
\put(1011,687){\rule[-0.175pt]{0.350pt}{0.562pt}}
\put(1010,689){\rule[-0.175pt]{0.350pt}{0.562pt}}
\put(1009,691){\rule[-0.175pt]{0.350pt}{0.723pt}}
\put(1008,695){\rule[-0.175pt]{0.350pt}{0.723pt}}
\put(1007,698){\rule[-0.175pt]{0.350pt}{0.482pt}}
\put(1006,700){\rule[-0.175pt]{0.350pt}{0.482pt}}
\put(1005,702){\rule[-0.175pt]{0.350pt}{0.482pt}}
\put(1004,704){\rule[-0.175pt]{0.350pt}{0.562pt}}
\put(1003,706){\rule[-0.175pt]{0.350pt}{0.562pt}}
\put(1002,708){\rule[-0.175pt]{0.350pt}{0.562pt}}
\put(1001,710){\rule[-0.175pt]{0.350pt}{0.723pt}}
\put(1000,714){\rule[-0.175pt]{0.350pt}{0.723pt}}
\put(999,717){\rule[-0.175pt]{0.350pt}{0.482pt}}
\put(998,719){\rule[-0.175pt]{0.350pt}{0.482pt}}
\put(997,721){\rule[-0.175pt]{0.350pt}{0.482pt}}
\put(996,723){\rule[-0.175pt]{0.350pt}{0.843pt}}
\put(995,726){\rule[-0.175pt]{0.350pt}{0.843pt}}
\put(994,730){\rule[-0.175pt]{0.350pt}{0.723pt}}
\put(993,733){\rule[-0.175pt]{0.350pt}{0.723pt}}
\put(992,736){\rule[-0.175pt]{0.350pt}{0.843pt}}
\put(991,739){\rule[-0.175pt]{0.350pt}{0.843pt}}
\put(990,743){\rule[-0.175pt]{0.350pt}{1.445pt}}
\put(989,749){\rule[-0.175pt]{0.350pt}{1.445pt}}
\put(988,755){\rule[-0.175pt]{0.350pt}{1.686pt}}
\put(987,762){\rule[-0.175pt]{0.350pt}{4.577pt}}
\put(988,781){\rule[-0.175pt]{0.350pt}{1.445pt}}
\end{picture}

\caption{Gelfand equation on the ball, $3\leq n \leq 9$.
\label{gelfand.fig2}}
\end{figure}
\begin{quote}\tt\singlespace\begin{verbatim}
\begin{figure}[htbp]
\centering
% GNUPLOT: LaTeX picture
\setlength{\unitlength}{0.240900pt}
\ifx\plotpoint\undefined\newsavebox{\plotpoint}\fi
\sbox{\plotpoint}{\rule[-0.175pt]{0.350pt}{0.350pt}}%
\begin{picture}(1500,900)(0,0)
%\tenrm
\sbox{\plotpoint}{\rule[-0.175pt]{0.350pt}{0.350pt}}%
\put(264,158){\rule[-0.175pt]{282.335pt}{0.350pt}}
\put(264,158){\rule[-0.175pt]{0.350pt}{151.526pt}}
\put(264,158){\rule[-0.175pt]{4.818pt}{0.350pt}}
%\put(242,158){\makebox(0,0)[r]{0}}
\put(1416,158){\rule[-0.175pt]{4.818pt}{0.350pt}}
\put(264,284){\rule[-0.175pt]{4.818pt}{0.350pt}}
%\put(242,284){\makebox(0,0)[r]{2}}
\put(1416,284){\rule[-0.175pt]{4.818pt}{0.350pt}}
\put(264,410){\rule[-0.175pt]{4.818pt}{0.350pt}}
%\put(242,410){\makebox(0,0)[r]{4}}
\put(1416,410){\rule[-0.175pt]{4.818pt}{0.350pt}}
\put(264,535){\rule[-0.175pt]{4.818pt}{0.350pt}}
%\put(242,535){\makebox(0,0)[r]{6}}
\put(1416,535){\rule[-0.175pt]{4.818pt}{0.350pt}}
\put(264,661){\rule[-0.175pt]{4.818pt}{0.350pt}}
%\put(242,661){\makebox(0,0)[r]{8}}
\put(1416,661){\rule[-0.175pt]{4.818pt}{0.350pt}}
\put(264,787){\rule[-0.175pt]{4.818pt}{0.350pt}}
%\put(242,787){\makebox(0,0)[r]{10}}
\put(1416,787){\rule[-0.175pt]{4.818pt}{0.350pt}}
\put(264,158){\rule[-0.175pt]{0.350pt}{4.818pt}}
%\put(264,113){\makebox(0,0){0}}
\put(264,767){\rule[-0.175pt]{0.350pt}{4.818pt}}
\put(411,158){\rule[-0.175pt]{0.350pt}{4.818pt}}
%\put(411,113){\makebox(0,0){0.5}}
\put(411,767){\rule[-0.175pt]{0.350pt}{4.818pt}}
\put(557,158){\rule[-0.175pt]{0.350pt}{4.818pt}}
%\put(557,113){\makebox(0,0){1}}
\put(557,767){\rule[-0.175pt]{0.350pt}{4.818pt}}
\put(704,158){\rule[-0.175pt]{0.350pt}{4.818pt}}
%\put(704,113){\makebox(0,0){1.5}}
\put(704,767){\rule[-0.175pt]{0.350pt}{4.818pt}}
\put(850,158){\rule[-0.175pt]{0.350pt}{4.818pt}}
%\put(850,113){\makebox(0,0){2}}
\put(850,767){\rule[-0.175pt]{0.350pt}{4.818pt}}
\put(997,158){\rule[-0.175pt]{0.350pt}{4.818pt}}
%\put(997,113){\makebox(0,0){2.5}}
\put(997,767){\rule[-0.175pt]{0.350pt}{4.818pt}}
\put(1143,158){\rule[-0.175pt]{0.350pt}{4.818pt}}
%\put(1143,113){\makebox(0,0){3}}
\put(1143,767){\rule[-0.175pt]{0.350pt}{4.818pt}}
\put(1290,158){\rule[-0.175pt]{0.350pt}{4.818pt}}
%\put(1290,113){\makebox(0,0){3.5}}
\put(1290,767){\rule[-0.175pt]{0.350pt}{4.818pt}}
\put(1436,158){\rule[-0.175pt]{0.350pt}{4.818pt}}
%\put(1436,113){\makebox(0,0){4}}
\put(1436,767){\rule[-0.175pt]{0.350pt}{4.818pt}}
\put(264,158){\rule[-0.175pt]{282.335pt}{0.350pt}}
\put(1436,158){\rule[-0.175pt]{0.350pt}{151.526pt}}
\put(264,787){\rule[-0.175pt]{282.335pt}{0.350pt}}
\put(100,472){\makebox(0,0)[l]{\shortstack{$\| u\|$}}}
\put(850,68){\makebox(0,0){$\lambda$}}
%\put(850,832){\makebox(0,0){plot}}
\put(264,158){\rule[-0.175pt]{0.350pt}{151.526pt}}
%\put(1306,722){\makebox(0,0)[r]{}}
%\put(1328,722){\rule[-0.175pt]{15.899pt}{0.350pt}}
\put(264,158){\usebox{\plotpoint}}
\put(264,158){\rule[-0.175pt]{6.304pt}{0.350pt}}
\put(290,159){\rule[-0.175pt]{6.304pt}{0.350pt}}
\put(316,160){\rule[-0.175pt]{6.304pt}{0.350pt}}
\put(342,161){\rule[-0.175pt]{6.304pt}{0.350pt}}
\put(368,162){\rule[-0.175pt]{6.304pt}{0.350pt}}
\put(394,163){\rule[-0.175pt]{6.304pt}{0.350pt}}
\put(420,164){\rule[-0.175pt]{5.644pt}{0.350pt}}
\put(444,165){\rule[-0.175pt]{5.644pt}{0.350pt}}
\put(467,166){\rule[-0.175pt]{5.644pt}{0.350pt}}
\put(491,167){\rule[-0.175pt]{5.644pt}{0.350pt}}
\put(514,168){\rule[-0.175pt]{5.644pt}{0.350pt}}
\put(538,169){\rule[-0.175pt]{5.644pt}{0.350pt}}
\put(561,170){\rule[-0.175pt]{5.644pt}{0.350pt}}
\put(585,171){\rule[-0.175pt]{6.384pt}{0.350pt}}
\put(611,172){\rule[-0.175pt]{6.384pt}{0.350pt}}
\put(638,173){\rule[-0.175pt]{6.384pt}{0.350pt}}
\put(664,174){\rule[-0.175pt]{6.384pt}{0.350pt}}
\put(691,175){\rule[-0.175pt]{6.384pt}{0.350pt}}
\put(717,176){\rule[-0.175pt]{6.384pt}{0.350pt}}
\put(744,177){\rule[-0.175pt]{5.862pt}{0.350pt}}
\put(768,178){\rule[-0.175pt]{5.862pt}{0.350pt}}
\put(792,179){\rule[-0.175pt]{5.862pt}{0.350pt}}
\put(816,180){\rule[-0.175pt]{5.862pt}{0.350pt}}
\put(841,181){\rule[-0.175pt]{5.862pt}{0.350pt}}
\put(865,182){\rule[-0.175pt]{5.862pt}{0.350pt}}
\put(889,183){\rule[-0.175pt]{4.371pt}{0.350pt}}
\put(908,184){\rule[-0.175pt]{4.371pt}{0.350pt}}
\put(926,185){\rule[-0.175pt]{4.371pt}{0.350pt}}
\put(944,186){\rule[-0.175pt]{4.371pt}{0.350pt}}
\put(962,187){\rule[-0.175pt]{4.371pt}{0.350pt}}
\put(980,188){\rule[-0.175pt]{4.371pt}{0.350pt}}
\put(998,189){\rule[-0.175pt]{4.371pt}{0.350pt}}
\put(1017,190){\rule[-0.175pt]{4.216pt}{0.350pt}}
\put(1034,191){\rule[-0.175pt]{4.216pt}{0.350pt}}
\put(1052,192){\rule[-0.175pt]{4.216pt}{0.350pt}}
\put(1069,193){\rule[-0.175pt]{4.216pt}{0.350pt}}
\put(1087,194){\rule[-0.175pt]{4.216pt}{0.350pt}}
\put(1104,195){\rule[-0.175pt]{4.216pt}{0.350pt}}
\put(1122,196){\rule[-0.175pt]{3.172pt}{0.350pt}}
\put(1135,197){\rule[-0.175pt]{3.172pt}{0.350pt}}
\put(1148,198){\rule[-0.175pt]{3.172pt}{0.350pt}}
\put(1161,199){\rule[-0.175pt]{3.172pt}{0.350pt}}
\put(1174,200){\rule[-0.175pt]{3.172pt}{0.350pt}}
\put(1187,201){\rule[-0.175pt]{3.172pt}{0.350pt}}
\put(1200,202){\rule[-0.175pt]{1.893pt}{0.350pt}}
\put(1208,203){\rule[-0.175pt]{1.893pt}{0.350pt}}
\put(1216,204){\rule[-0.175pt]{1.893pt}{0.350pt}}
\put(1224,205){\rule[-0.175pt]{1.893pt}{0.350pt}}
\put(1232,206){\rule[-0.175pt]{1.893pt}{0.350pt}}
\put(1240,207){\rule[-0.175pt]{1.893pt}{0.350pt}}
\put(1248,208){\rule[-0.175pt]{1.893pt}{0.350pt}}
\put(1256,209){\rule[-0.175pt]{1.245pt}{0.350pt}}
\put(1261,210){\rule[-0.175pt]{1.245pt}{0.350pt}}
\put(1266,211){\rule[-0.175pt]{1.245pt}{0.350pt}}
\put(1271,212){\rule[-0.175pt]{1.245pt}{0.350pt}}
\put(1276,213){\rule[-0.175pt]{1.245pt}{0.350pt}}
\put(1281,214){\rule[-0.175pt]{1.245pt}{0.350pt}}
\put(1286,215){\usebox{\plotpoint}}
\put(1288,216){\usebox{\plotpoint}}
\put(1289,217){\usebox{\plotpoint}}
\put(1291,218){\usebox{\plotpoint}}
\put(1292,219){\usebox{\plotpoint}}
\put(1294,220){\usebox{\plotpoint}}
\put(1295,221){\usebox{\plotpoint}}
\put(1295,222){\rule[-0.175pt]{0.361pt}{0.350pt}}
\put(1294,223){\rule[-0.175pt]{0.361pt}{0.350pt}}
\put(1292,224){\rule[-0.175pt]{0.361pt}{0.350pt}}
\put(1291,225){\rule[-0.175pt]{0.361pt}{0.350pt}}
\put(1289,226){\rule[-0.175pt]{0.361pt}{0.350pt}}
\put(1288,227){\rule[-0.175pt]{0.361pt}{0.350pt}}
\put(1284,228){\rule[-0.175pt]{0.964pt}{0.350pt}}
\put(1280,229){\rule[-0.175pt]{0.964pt}{0.350pt}}
\put(1276,230){\rule[-0.175pt]{0.964pt}{0.350pt}}
\put(1272,231){\rule[-0.175pt]{0.964pt}{0.350pt}}
\put(1268,232){\rule[-0.175pt]{0.964pt}{0.350pt}}
\put(1264,233){\rule[-0.175pt]{0.964pt}{0.350pt}}
\put(1258,234){\rule[-0.175pt]{1.273pt}{0.350pt}}
\put(1253,235){\rule[-0.175pt]{1.273pt}{0.350pt}}
\put(1248,236){\rule[-0.175pt]{1.273pt}{0.350pt}}
\put(1242,237){\rule[-0.175pt]{1.273pt}{0.350pt}}
\put(1237,238){\rule[-0.175pt]{1.273pt}{0.350pt}}
\put(1232,239){\rule[-0.175pt]{1.273pt}{0.350pt}}
\put(1227,240){\rule[-0.175pt]{1.273pt}{0.350pt}}
\put(1219,241){\rule[-0.175pt]{1.847pt}{0.350pt}}
\put(1211,242){\rule[-0.175pt]{1.847pt}{0.350pt}}
\put(1204,243){\rule[-0.175pt]{1.847pt}{0.350pt}}
\put(1196,244){\rule[-0.175pt]{1.847pt}{0.350pt}}
\put(1188,245){\rule[-0.175pt]{1.847pt}{0.350pt}}
\put(1181,246){\rule[-0.175pt]{1.847pt}{0.350pt}}
\put(1172,247){\rule[-0.175pt]{2.128pt}{0.350pt}}
\put(1163,248){\rule[-0.175pt]{2.128pt}{0.350pt}}
\put(1154,249){\rule[-0.175pt]{2.128pt}{0.350pt}}
\put(1145,250){\rule[-0.175pt]{2.128pt}{0.350pt}}
\put(1136,251){\rule[-0.175pt]{2.128pt}{0.350pt}}
\put(1128,252){\rule[-0.175pt]{2.128pt}{0.350pt}}
\put(1120,253){\rule[-0.175pt]{1.893pt}{0.350pt}}
\put(1112,254){\rule[-0.175pt]{1.893pt}{0.350pt}}
\put(1104,255){\rule[-0.175pt]{1.893pt}{0.350pt}}
\put(1096,256){\rule[-0.175pt]{1.893pt}{0.350pt}}
\put(1088,257){\rule[-0.175pt]{1.893pt}{0.350pt}}
\put(1080,258){\rule[-0.175pt]{1.893pt}{0.350pt}}
\put(1073,259){\rule[-0.175pt]{1.893pt}{0.350pt}}
\put(1063,260){\rule[-0.175pt]{2.208pt}{0.350pt}}
\put(1054,261){\rule[-0.175pt]{2.208pt}{0.350pt}}
\put(1045,262){\rule[-0.175pt]{2.208pt}{0.350pt}}
\put(1036,263){\rule[-0.175pt]{2.208pt}{0.350pt}}
\put(1027,264){\rule[-0.175pt]{2.208pt}{0.350pt}}
\put(1018,265){\rule[-0.175pt]{2.208pt}{0.350pt}}
\put(1009,266){\rule[-0.175pt]{2.168pt}{0.350pt}}
\put(1000,267){\rule[-0.175pt]{2.168pt}{0.350pt}}
\put(991,268){\rule[-0.175pt]{2.168pt}{0.350pt}}
\put(982,269){\rule[-0.175pt]{2.168pt}{0.350pt}}
\put(973,270){\rule[-0.175pt]{2.168pt}{0.350pt}}
\put(964,271){\rule[-0.175pt]{2.168pt}{0.350pt}}
\put(957,272){\rule[-0.175pt]{1.686pt}{0.350pt}}
\put(950,273){\rule[-0.175pt]{1.686pt}{0.350pt}}
\put(943,274){\rule[-0.175pt]{1.686pt}{0.350pt}}
\put(936,275){\rule[-0.175pt]{1.686pt}{0.350pt}}
\put(929,276){\rule[-0.175pt]{1.686pt}{0.350pt}}
\put(922,277){\rule[-0.175pt]{1.686pt}{0.350pt}}
\put(915,278){\rule[-0.175pt]{1.686pt}{0.350pt}}
\put(907,279){\rule[-0.175pt]{1.767pt}{0.350pt}}
\put(900,280){\rule[-0.175pt]{1.767pt}{0.350pt}}
\put(893,281){\rule[-0.175pt]{1.767pt}{0.350pt}}
\put(885,282){\rule[-0.175pt]{1.767pt}{0.350pt}}
\put(878,283){\rule[-0.175pt]{1.767pt}{0.350pt}}
\put(871,284){\rule[-0.175pt]{1.767pt}{0.350pt}}
\put(864,285){\rule[-0.175pt]{1.486pt}{0.350pt}}
\put(858,286){\rule[-0.175pt]{1.486pt}{0.350pt}}
\put(852,287){\rule[-0.175pt]{1.486pt}{0.350pt}}
\put(846,288){\rule[-0.175pt]{1.486pt}{0.350pt}}
\put(840,289){\rule[-0.175pt]{1.486pt}{0.350pt}}
\put(834,290){\rule[-0.175pt]{1.486pt}{0.350pt}}
\put(829,291){\rule[-0.175pt]{0.998pt}{0.350pt}}
\put(825,292){\rule[-0.175pt]{0.998pt}{0.350pt}}
\put(821,293){\rule[-0.175pt]{0.998pt}{0.350pt}}
\put(817,294){\rule[-0.175pt]{0.998pt}{0.350pt}}
\put(813,295){\rule[-0.175pt]{0.998pt}{0.350pt}}
\put(809,296){\rule[-0.175pt]{0.998pt}{0.350pt}}
\put(805,297){\rule[-0.175pt]{0.998pt}{0.350pt}}
\put(801,298){\rule[-0.175pt]{0.883pt}{0.350pt}}
\put(797,299){\rule[-0.175pt]{0.883pt}{0.350pt}}
\put(793,300){\rule[-0.175pt]{0.883pt}{0.350pt}}
\put(790,301){\rule[-0.175pt]{0.883pt}{0.350pt}}
\put(786,302){\rule[-0.175pt]{0.883pt}{0.350pt}}
\put(783,303){\rule[-0.175pt]{0.883pt}{0.350pt}}
\put(780,304){\rule[-0.175pt]{0.522pt}{0.350pt}}
\put(778,305){\rule[-0.175pt]{0.522pt}{0.350pt}}
\put(776,306){\rule[-0.175pt]{0.522pt}{0.350pt}}
\put(774,307){\rule[-0.175pt]{0.522pt}{0.350pt}}
\put(772,308){\rule[-0.175pt]{0.522pt}{0.350pt}}
\put(770,309){\rule[-0.175pt]{0.522pt}{0.350pt}}
\put(770,310){\usebox{\plotpoint}}
\put(769,311){\usebox{\plotpoint}}
\put(768,312){\usebox{\plotpoint}}
\put(767,314){\usebox{\plotpoint}}
\put(766,315){\usebox{\plotpoint}}
\put(765,316){\rule[-0.175pt]{0.350pt}{0.723pt}}
\put(766,320){\rule[-0.175pt]{0.350pt}{0.723pt}}
\put(767,323){\usebox{\plotpoint}}
\put(768,324){\usebox{\plotpoint}}
\put(769,325){\usebox{\plotpoint}}
\put(771,326){\usebox{\plotpoint}}
\put(772,327){\usebox{\plotpoint}}
\put(774,328){\usebox{\plotpoint}}
\put(775,329){\usebox{\plotpoint}}
\put(777,330){\rule[-0.175pt]{0.602pt}{0.350pt}}
\put(779,331){\rule[-0.175pt]{0.602pt}{0.350pt}}
\put(782,332){\rule[-0.175pt]{0.602pt}{0.350pt}}
\put(784,333){\rule[-0.175pt]{0.602pt}{0.350pt}}
\put(787,334){\rule[-0.175pt]{0.602pt}{0.350pt}}
\put(789,335){\rule[-0.175pt]{0.602pt}{0.350pt}}
\put(792,336){\rule[-0.175pt]{0.843pt}{0.350pt}}
\put(795,337){\rule[-0.175pt]{0.843pt}{0.350pt}}
\put(799,338){\rule[-0.175pt]{0.843pt}{0.350pt}}
\put(802,339){\rule[-0.175pt]{0.843pt}{0.350pt}}
\put(806,340){\rule[-0.175pt]{0.843pt}{0.350pt}}
\put(809,341){\rule[-0.175pt]{0.843pt}{0.350pt}}
\put(813,342){\rule[-0.175pt]{0.826pt}{0.350pt}}
\put(816,343){\rule[-0.175pt]{0.826pt}{0.350pt}}
\put(819,344){\rule[-0.175pt]{0.826pt}{0.350pt}}
\put(823,345){\rule[-0.175pt]{0.826pt}{0.350pt}}
\put(826,346){\rule[-0.175pt]{0.826pt}{0.350pt}}
\put(830,347){\rule[-0.175pt]{0.826pt}{0.350pt}}
\put(833,348){\rule[-0.175pt]{0.826pt}{0.350pt}}
\put(837,349){\rule[-0.175pt]{1.084pt}{0.350pt}}
\put(841,350){\rule[-0.175pt]{1.084pt}{0.350pt}}
\put(846,351){\rule[-0.175pt]{1.084pt}{0.350pt}}
\put(850,352){\rule[-0.175pt]{1.084pt}{0.350pt}}
\put(855,353){\rule[-0.175pt]{1.084pt}{0.350pt}}
\put(859,354){\rule[-0.175pt]{1.084pt}{0.350pt}}
\put(864,355){\rule[-0.175pt]{1.164pt}{0.350pt}}
\put(868,356){\rule[-0.175pt]{1.164pt}{0.350pt}}
\put(873,357){\rule[-0.175pt]{1.164pt}{0.350pt}}
\put(878,358){\rule[-0.175pt]{1.164pt}{0.350pt}}
\put(883,359){\rule[-0.175pt]{1.164pt}{0.350pt}}
\put(888,360){\rule[-0.175pt]{1.164pt}{0.350pt}}
\put(892,361){\rule[-0.175pt]{1.032pt}{0.350pt}}
\put(897,362){\rule[-0.175pt]{1.032pt}{0.350pt}}
\put(901,363){\rule[-0.175pt]{1.032pt}{0.350pt}}
\put(905,364){\rule[-0.175pt]{1.032pt}{0.350pt}}
\put(910,365){\rule[-0.175pt]{1.032pt}{0.350pt}}
\put(914,366){\rule[-0.175pt]{1.032pt}{0.350pt}}
\put(918,367){\rule[-0.175pt]{1.032pt}{0.350pt}}
\put(922,368){\rule[-0.175pt]{1.205pt}{0.350pt}}
\put(928,369){\rule[-0.175pt]{1.204pt}{0.350pt}}
\put(933,370){\rule[-0.175pt]{1.204pt}{0.350pt}}
\put(938,371){\rule[-0.175pt]{1.204pt}{0.350pt}}
\put(943,372){\rule[-0.175pt]{1.204pt}{0.350pt}}
\put(948,373){\rule[-0.175pt]{1.204pt}{0.350pt}}
\put(953,374){\rule[-0.175pt]{1.124pt}{0.350pt}}
\put(957,375){\rule[-0.175pt]{1.124pt}{0.350pt}}
\put(962,376){\rule[-0.175pt]{1.124pt}{0.350pt}}
\put(967,377){\rule[-0.175pt]{1.124pt}{0.350pt}}
\put(971,378){\rule[-0.175pt]{1.124pt}{0.350pt}}
\put(976,379){\rule[-0.175pt]{1.124pt}{0.350pt}}
\put(981,380){\rule[-0.175pt]{0.929pt}{0.350pt}}
\put(984,381){\rule[-0.175pt]{0.929pt}{0.350pt}}
\put(988,382){\rule[-0.175pt]{0.929pt}{0.350pt}}
\put(992,383){\rule[-0.175pt]{0.929pt}{0.350pt}}
\put(996,384){\rule[-0.175pt]{0.929pt}{0.350pt}}
\put(1000,385){\rule[-0.175pt]{0.929pt}{0.350pt}}
\put(1004,386){\rule[-0.175pt]{0.929pt}{0.350pt}}
\put(1007,387){\rule[-0.175pt]{0.923pt}{0.350pt}}
\put(1011,388){\rule[-0.175pt]{0.923pt}{0.350pt}}
\put(1015,389){\rule[-0.175pt]{0.923pt}{0.350pt}}
\put(1019,390){\rule[-0.175pt]{0.923pt}{0.350pt}}
\put(1023,391){\rule[-0.175pt]{0.923pt}{0.350pt}}
\put(1027,392){\rule[-0.175pt]{0.923pt}{0.350pt}}
\put(1031,393){\rule[-0.175pt]{0.843pt}{0.350pt}}
\put(1034,394){\rule[-0.175pt]{0.843pt}{0.350pt}}
\put(1038,395){\rule[-0.175pt]{0.843pt}{0.350pt}}
\put(1041,396){\rule[-0.175pt]{0.843pt}{0.350pt}}
\put(1045,397){\rule[-0.175pt]{0.843pt}{0.350pt}}
\put(1048,398){\rule[-0.175pt]{0.843pt}{0.350pt}}
\put(1052,399){\rule[-0.175pt]{0.585pt}{0.350pt}}
\put(1054,400){\rule[-0.175pt]{0.585pt}{0.350pt}}
\put(1056,401){\rule[-0.175pt]{0.585pt}{0.350pt}}
\put(1059,402){\rule[-0.175pt]{0.585pt}{0.350pt}}
\put(1061,403){\rule[-0.175pt]{0.585pt}{0.350pt}}
\put(1064,404){\rule[-0.175pt]{0.585pt}{0.350pt}}
\put(1066,405){\rule[-0.175pt]{0.585pt}{0.350pt}}
\put(1069,406){\rule[-0.175pt]{0.522pt}{0.350pt}}
\put(1071,407){\rule[-0.175pt]{0.522pt}{0.350pt}}
\put(1073,408){\rule[-0.175pt]{0.522pt}{0.350pt}}
\put(1075,409){\rule[-0.175pt]{0.522pt}{0.350pt}}
\put(1077,410){\rule[-0.175pt]{0.522pt}{0.350pt}}
\put(1079,411){\rule[-0.175pt]{0.522pt}{0.350pt}}
\put(1081,412){\rule[-0.175pt]{0.402pt}{0.350pt}}
\put(1083,413){\rule[-0.175pt]{0.401pt}{0.350pt}}
\put(1085,414){\rule[-0.175pt]{0.401pt}{0.350pt}}
\put(1086,415){\rule[-0.175pt]{0.401pt}{0.350pt}}
\put(1088,416){\rule[-0.175pt]{0.401pt}{0.350pt}}
\put(1090,417){\rule[-0.175pt]{0.401pt}{0.350pt}}
\put(1091,418){\usebox{\plotpoint}}
\put(1092,418){\usebox{\plotpoint}}
\put(1093,419){\usebox{\plotpoint}}
\put(1094,420){\usebox{\plotpoint}}
\put(1095,422){\usebox{\plotpoint}}
\put(1096,423){\usebox{\plotpoint}}
\put(1097,424){\rule[-0.175pt]{0.350pt}{0.723pt}}
\put(1098,428){\rule[-0.175pt]{0.350pt}{0.723pt}}
\put(1099,431){\rule[-0.175pt]{0.350pt}{1.686pt}}
\put(1098,438){\usebox{\plotpoint}}
\put(1097,439){\usebox{\plotpoint}}
\put(1096,440){\usebox{\plotpoint}}
\put(1095,441){\usebox{\plotpoint}}
\put(1094,442){\usebox{\plotpoint}}
\put(1091,444){\usebox{\plotpoint}}
\put(1090,445){\usebox{\plotpoint}}
\put(1089,446){\usebox{\plotpoint}}
\put(1088,447){\usebox{\plotpoint}}
\put(1087,448){\usebox{\plotpoint}}
\put(1086,449){\usebox{\plotpoint}}
\put(1084,450){\usebox{\plotpoint}}
\put(1083,451){\usebox{\plotpoint}}
\put(1081,452){\usebox{\plotpoint}}
\put(1080,453){\usebox{\plotpoint}}
\put(1078,454){\usebox{\plotpoint}}
\put(1077,455){\usebox{\plotpoint}}
\put(1076,456){\usebox{\plotpoint}}
\put(1074,457){\rule[-0.175pt]{0.442pt}{0.350pt}}
\put(1072,458){\rule[-0.175pt]{0.442pt}{0.350pt}}
\put(1070,459){\rule[-0.175pt]{0.442pt}{0.350pt}}
\put(1068,460){\rule[-0.175pt]{0.442pt}{0.350pt}}
\put(1066,461){\rule[-0.175pt]{0.442pt}{0.350pt}}
\put(1065,462){\rule[-0.175pt]{0.442pt}{0.350pt}}
\put(1063,463){\rule[-0.175pt]{0.482pt}{0.350pt}}
\put(1061,464){\rule[-0.175pt]{0.482pt}{0.350pt}}
\put(1059,465){\rule[-0.175pt]{0.482pt}{0.350pt}}
\put(1057,466){\rule[-0.175pt]{0.482pt}{0.350pt}}
\put(1055,467){\rule[-0.175pt]{0.482pt}{0.350pt}}
\put(1053,468){\rule[-0.175pt]{0.482pt}{0.350pt}}
\put(1051,469){\rule[-0.175pt]{0.482pt}{0.350pt}}
\put(1049,470){\rule[-0.175pt]{0.482pt}{0.350pt}}
\put(1047,471){\rule[-0.175pt]{0.482pt}{0.350pt}}
\put(1045,472){\rule[-0.175pt]{0.482pt}{0.350pt}}
\put(1043,473){\rule[-0.175pt]{0.482pt}{0.350pt}}
\put(1041,474){\rule[-0.175pt]{0.482pt}{0.350pt}}
\put(1039,475){\rule[-0.175pt]{0.482pt}{0.350pt}}
\put(1036,476){\rule[-0.175pt]{0.522pt}{0.350pt}}
\put(1034,477){\rule[-0.175pt]{0.522pt}{0.350pt}}
\put(1032,478){\rule[-0.175pt]{0.522pt}{0.350pt}}
\put(1030,479){\rule[-0.175pt]{0.522pt}{0.350pt}}
\put(1028,480){\rule[-0.175pt]{0.522pt}{0.350pt}}
\put(1026,481){\rule[-0.175pt]{0.522pt}{0.350pt}}
\put(1023,482){\rule[-0.175pt]{0.522pt}{0.350pt}}
\put(1021,483){\rule[-0.175pt]{0.522pt}{0.350pt}}
\put(1019,484){\rule[-0.175pt]{0.522pt}{0.350pt}}
\put(1017,485){\rule[-0.175pt]{0.522pt}{0.350pt}}
\put(1015,486){\rule[-0.175pt]{0.522pt}{0.350pt}}
\put(1013,487){\rule[-0.175pt]{0.522pt}{0.350pt}}
\put(1011,488){\rule[-0.175pt]{0.447pt}{0.350pt}}
\put(1009,489){\rule[-0.175pt]{0.447pt}{0.350pt}}
\put(1007,490){\rule[-0.175pt]{0.447pt}{0.350pt}}
\put(1005,491){\rule[-0.175pt]{0.447pt}{0.350pt}}
\put(1003,492){\rule[-0.175pt]{0.447pt}{0.350pt}}
\put(1001,493){\rule[-0.175pt]{0.447pt}{0.350pt}}
\put(1000,494){\rule[-0.175pt]{0.447pt}{0.350pt}}
\put(998,495){\rule[-0.175pt]{0.442pt}{0.350pt}}
\put(996,496){\rule[-0.175pt]{0.442pt}{0.350pt}}
\put(994,497){\rule[-0.175pt]{0.442pt}{0.350pt}}
\put(992,498){\rule[-0.175pt]{0.442pt}{0.350pt}}
\put(990,499){\rule[-0.175pt]{0.442pt}{0.350pt}}
\put(989,500){\rule[-0.175pt]{0.442pt}{0.350pt}}
\put(987,501){\rule[-0.175pt]{0.442pt}{0.350pt}}
\put(985,502){\rule[-0.175pt]{0.442pt}{0.350pt}}
\put(983,503){\rule[-0.175pt]{0.442pt}{0.350pt}}
\put(981,504){\rule[-0.175pt]{0.442pt}{0.350pt}}
\put(979,505){\rule[-0.175pt]{0.442pt}{0.350pt}}
\put(978,506){\rule[-0.175pt]{0.442pt}{0.350pt}}
\put(976,507){\usebox{\plotpoint}}
\put(975,508){\usebox{\plotpoint}}
\put(974,509){\usebox{\plotpoint}}
\put(972,510){\usebox{\plotpoint}}
\put(971,511){\usebox{\plotpoint}}
\put(970,512){\usebox{\plotpoint}}
\put(969,513){\usebox{\plotpoint}}
\put(967,514){\usebox{\plotpoint}}
\put(966,515){\usebox{\plotpoint}}
\put(965,516){\usebox{\plotpoint}}
\put(964,517){\usebox{\plotpoint}}
\put(963,518){\usebox{\plotpoint}}
\put(962,519){\usebox{\plotpoint}}
\put(962,520){\usebox{\plotpoint}}
\put(961,521){\usebox{\plotpoint}}
\put(960,522){\usebox{\plotpoint}}
\put(959,524){\usebox{\plotpoint}}
\put(958,525){\usebox{\plotpoint}}
\put(957,527){\rule[-0.175pt]{0.350pt}{0.361pt}}
\put(956,528){\rule[-0.175pt]{0.350pt}{0.361pt}}
\put(955,530){\rule[-0.175pt]{0.350pt}{0.361pt}}
\put(954,531){\rule[-0.175pt]{0.350pt}{0.361pt}}
\put(953,533){\rule[-0.175pt]{0.350pt}{0.723pt}}
\put(952,536){\rule[-0.175pt]{0.350pt}{0.723pt}}
\put(951,539){\rule[-0.175pt]{0.350pt}{1.686pt}}
\put(950,546){\rule[-0.175pt]{0.350pt}{1.445pt}}
\put(951,552){\rule[-0.175pt]{0.350pt}{0.482pt}}
\put(952,554){\rule[-0.175pt]{0.350pt}{0.482pt}}
\put(953,556){\rule[-0.175pt]{0.350pt}{0.482pt}}
\put(954,558){\rule[-0.175pt]{0.350pt}{0.562pt}}
\put(955,560){\rule[-0.175pt]{0.350pt}{0.562pt}}
\put(956,562){\rule[-0.175pt]{0.350pt}{0.562pt}}
\put(957,564){\usebox{\plotpoint}}
\put(958,566){\usebox{\plotpoint}}
\put(959,567){\usebox{\plotpoint}}
\put(960,568){\usebox{\plotpoint}}
\put(961,569){\usebox{\plotpoint}}
\put(962,571){\usebox{\plotpoint}}
\put(963,572){\usebox{\plotpoint}}
\put(964,573){\usebox{\plotpoint}}
\put(965,574){\usebox{\plotpoint}}
\put(966,575){\usebox{\plotpoint}}
\put(967,577){\usebox{\plotpoint}}
\put(968,578){\usebox{\plotpoint}}
\put(969,579){\usebox{\plotpoint}}
\put(970,580){\usebox{\plotpoint}}
\put(971,581){\usebox{\plotpoint}}
\put(972,582){\usebox{\plotpoint}}
\put(973,584){\usebox{\plotpoint}}
\put(974,585){\usebox{\plotpoint}}
\put(975,586){\usebox{\plotpoint}}
\put(976,587){\usebox{\plotpoint}}
\put(977,588){\usebox{\plotpoint}}
\put(978,589){\usebox{\plotpoint}}
\put(979,590){\usebox{\plotpoint}}
\put(980,591){\usebox{\plotpoint}}
\put(981,592){\usebox{\plotpoint}}
\put(982,593){\usebox{\plotpoint}}
\put(983,594){\usebox{\plotpoint}}
\put(984,595){\usebox{\plotpoint}}
\put(985,596){\usebox{\plotpoint}}
\put(986,597){\usebox{\plotpoint}}
\put(987,598){\usebox{\plotpoint}}
\put(988,600){\usebox{\plotpoint}}
\put(989,601){\usebox{\plotpoint}}
\put(990,603){\usebox{\plotpoint}}
\put(991,604){\usebox{\plotpoint}}
\put(992,605){\usebox{\plotpoint}}
\put(993,606){\usebox{\plotpoint}}
\put(994,607){\usebox{\plotpoint}}
\put(995,608){\usebox{\plotpoint}}
\put(996,609){\usebox{\plotpoint}}
\put(997,610){\usebox{\plotpoint}}
\put(998,611){\usebox{\plotpoint}}
\put(999,612){\usebox{\plotpoint}}
\put(1000,613){\usebox{\plotpoint}}
\put(1001,615){\usebox{\plotpoint}}
\put(1002,616){\usebox{\plotpoint}}
\put(1003,617){\usebox{\plotpoint}}
\put(1004,619){\usebox{\plotpoint}}
\put(1005,620){\usebox{\plotpoint}}
\put(1006,622){\rule[-0.175pt]{0.350pt}{0.361pt}}
\put(1007,623){\rule[-0.175pt]{0.350pt}{0.361pt}}
\put(1008,625){\rule[-0.175pt]{0.350pt}{0.361pt}}
\put(1009,626){\rule[-0.175pt]{0.350pt}{0.361pt}}
\put(1010,628){\rule[-0.175pt]{0.350pt}{0.562pt}}
\put(1011,630){\rule[-0.175pt]{0.350pt}{0.562pt}}
\put(1012,632){\rule[-0.175pt]{0.350pt}{0.562pt}}
\put(1013,634){\rule[-0.175pt]{0.350pt}{0.723pt}}
\put(1014,638){\rule[-0.175pt]{0.350pt}{0.723pt}}
\put(1015,641){\rule[-0.175pt]{0.350pt}{1.445pt}}
\put(1016,647){\rule[-0.175pt]{0.350pt}{1.686pt}}
\put(1017,654){\rule[-0.175pt]{0.350pt}{3.734pt}}
\put(1016,669){\rule[-0.175pt]{0.350pt}{0.843pt}}
\put(1015,673){\rule[-0.175pt]{0.350pt}{1.445pt}}
\put(1014,679){\rule[-0.175pt]{0.350pt}{0.723pt}}
\put(1013,682){\rule[-0.175pt]{0.350pt}{0.723pt}}
\put(1012,685){\rule[-0.175pt]{0.350pt}{0.562pt}}
\put(1011,687){\rule[-0.175pt]{0.350pt}{0.562pt}}
\put(1010,689){\rule[-0.175pt]{0.350pt}{0.562pt}}
\put(1009,691){\rule[-0.175pt]{0.350pt}{0.723pt}}
\put(1008,695){\rule[-0.175pt]{0.350pt}{0.723pt}}
\put(1007,698){\rule[-0.175pt]{0.350pt}{0.482pt}}
\put(1006,700){\rule[-0.175pt]{0.350pt}{0.482pt}}
\put(1005,702){\rule[-0.175pt]{0.350pt}{0.482pt}}
\put(1004,704){\rule[-0.175pt]{0.350pt}{0.562pt}}
\put(1003,706){\rule[-0.175pt]{0.350pt}{0.562pt}}
\put(1002,708){\rule[-0.175pt]{0.350pt}{0.562pt}}
\put(1001,710){\rule[-0.175pt]{0.350pt}{0.723pt}}
\put(1000,714){\rule[-0.175pt]{0.350pt}{0.723pt}}
\put(999,717){\rule[-0.175pt]{0.350pt}{0.482pt}}
\put(998,719){\rule[-0.175pt]{0.350pt}{0.482pt}}
\put(997,721){\rule[-0.175pt]{0.350pt}{0.482pt}}
\put(996,723){\rule[-0.175pt]{0.350pt}{0.843pt}}
\put(995,726){\rule[-0.175pt]{0.350pt}{0.843pt}}
\put(994,730){\rule[-0.175pt]{0.350pt}{0.723pt}}
\put(993,733){\rule[-0.175pt]{0.350pt}{0.723pt}}
\put(992,736){\rule[-0.175pt]{0.350pt}{0.843pt}}
\put(991,739){\rule[-0.175pt]{0.350pt}{0.843pt}}
\put(990,743){\rule[-0.175pt]{0.350pt}{1.445pt}}
\put(989,749){\rule[-0.175pt]{0.350pt}{1.445pt}}
\put(988,755){\rule[-0.175pt]{0.350pt}{1.686pt}}
\put(987,762){\rule[-0.175pt]{0.350pt}{4.577pt}}
\put(988,781){\rule[-0.175pt]{0.350pt}{1.445pt}}
\end{picture}

\caption{Gelfand equation on the ball, $3\leq n \leq 9$.
\label{gelfand.fig2}}
\end{figure}
\end{verbatim}\end{quote}
One advantage to using the native \LaTeX{} {\tt picture} environment
is that the fonts will be assured to agree and the pictures can be viewed
in the {\tt .dvi} viewer.

\subsection{PostScript}
Many drawing applications now allow the export of a graphic to the
{\em Encapsulated PostScript} format.  These files have a suffix of
{\tt .EPS} or {\tt .EPSF} and are similar to a regular PostScript
file except that they contain a {\em bounding box} which describes
the dimensions of the figure.

In order to include PostScript figures, the {\tt epsfig} (or {\tt psfig}
depending on the system you are using) style file must be included in either
the {\tt\verb|\documentstyle|} command or the preamble using the {\tt input} command.

Figure~\ref{vwcontr} is a plot from Matlab.
\begin{figure}[htbp]
\centerline{
\psfig{figure=vwcontr.eps,width=5in,angle=0}
           }
\caption{$\sigma$ as a Function of Voltage and Speed, $\alpha = 20$}
\label{vwcontr}
\end{figure}
The commands to include this figure are
\begin{quote}\tt\singlespace\begin{verbatim}
\begin{figure}[htbp]
\centerline{
\psfig{figure=vwcontr.ps,width=5in,angle=0}
           }
\caption{$\sigma$ as a Function of Voltage and Speed, $\alpha = 20$}
\label{vwcontr}
\end{figure}
\end{verbatim}\end{quote}

Observe that the {\tt \verb|\psfig|} command allows the scaling of the figure
by setting either the {\tt width} or {\tt height} of the figure.  If only one
dimension is specified, the other is computed to keep the same aspect ratio.
The figure can also be rotated by setting {\tt angle} to the desired value in
degrees.
           % Chapter 3 Edited from UW Math Dept's Sample Thesis
% bibs.tex
%
% This chapter briefly talks about BibTex and is mostly
% copied from a similar chapter from "How to TeX a Thesis:
% The Purdue Thesis Styles" by James Darrell McCauley and
% Scott Hucker
%

\newcommand{\BibTeX}{{\sc Bib}\TeX}

\chapter{Citations and Bibliographies}
This chapter is an edited form of the same chapter from {\em How to 
\TeX{} a Thesis: The Purdue Thesis Styles} by James Darrell McCauley and
Scott Hucker.

The task of compiling and formatting the sources cited in papers can
be quite tedious, especially for large documents like theses.  A program
separate from \LaTeX{}, called ``\BibTeX{},''can be used to automate this task~\cite{lamport}.

\section{The Citation Command}
When referring to the work of someone else, the {\tt \verb|\cite|} command is used.
This generates the citation in the text for you.  In the above paragraph, the command
{\tt \verb|\cite{lamport}|} was used after the word ``task.''  The formatting of your
citation is handled by either the document style or a style option.  The default citation
style uses the number system (a number in square brackets).  Other citation styles
may use the author-date system, (Lamport, 1986) or the superscript$^3$ system.

\section{Bibliography Styles}
The way that a reference is formatted in your bibliography depends on the bibliography
style, which is specified near the beginning of your document with the\break
{\tt \verb|\bibliographstyle{file}|} command.  The file {\tt file.bst} is the name of the 
bibliography style file.  Standard \BibTeX{} bibliography style files include {\tt plain},
{\tt unsrt}, {\tt alpha}, and {\tt abbrev}.  The bibliography style governs whether or not
references are sorted, whether first names or initials are used for authors, whether or 
not last names are listed first, the location of the year in the references (after the
author or at the end of the reference), {\em etc.}.  You may be required by your
department or major professor to follow as style for a particular journal.  If so, then you
will need to find a \BibTeX{} style file to suit your needs.  Most major journals have
style files.  If you cannot locate an appropriate \BibTeX{} style file, then choose the
one which is closest and then edit the {\tt .bbl} file by hand.  See Section~\ref{BBL}
for a brief discussion on the {\tt .bbl} file.  Some common, but non-standard \BibTeX{}
styles include
\begin{tabbing}
{\tt jacs-new.bstxxxx}\= {\em Journal of the American Chemical Society}\kill
{\tt acm.bst}\>The Association for Computing Machinery\\
{\tt ieeetr.bst}\> The {\em IEEE Transactions} style\\
{\tt jacs-new.bst}\> {\em Journal of the American Chemical Society}
\end{tabbing}

\section{The Database}
The  {\tt \verb|\bibliography{file}|} command is placed in your input file at the location
where the ``List of References'' section\footnote{or ``Bibliography'' 
if {\tt \char92 altbibtitle } has been specified in the preamble.} would be.  It specifies the name (or names) of
your bibliographic data base, {\tt file.bib}.  An example entry in a \BibTeX{}
database is:
\begin{quote}\singlespace\tt\begin{verbatim}
@book{ lamport86 ,
     author =    "Leslie Lamport" ,
     title =     "\LaTeX: A Document Preparation System" ,
     publisher = "Addison--Wesley Pub.\ Co." ,
     year =      "1986" ,
     address =   "Reading, MA" 
}
\end{verbatim}\end{quote}

The citation key is the first field in this entry--- citing this book in a \LaTeX{}
file would look like
\begin{quote}\singlespace\tt\begin{verbatim}
According to Lamport~\cite{lamport86} ...
\end{verbatim}\end{quote}
The tilde ({\tt \verb|~|}) is used to tie the word ``Lamport'' to the citation
generated.  The space between these words is then unbreakable---the word ``Lamport''
and the citation \cite{lamport} will not be split across two lines if they happen to occur
near the end of a line.

A listing of all entry types with their required and optional fields is given in 
Appendix~\ref{bibrefs}. There are several tools which exist to help in editing a \BibTeX{}
file, however, their use is beyond the scope of this manual and can be found by searching
the net.  You can simply use a plain text editor like {\tt vi} or {\tt WordPad} to edit
and create the database files.

There are several rules which you must follow when creating your database.  Authors are
always listed by their full names, first name first, and multiple authors are separated
by {\tt and}.  For example
\begin{quote}\singlespace\tt\begin{verbatim}
author = "John Jay Park and Frederick Gene Watson and
          Michelle Catherine Smith",
\end{verbatim}\end{quote}
If you were using {\tt abbrv} as your {\tt bibliographystyle}, a reference for these
authors may look like:
\begin{quote}
J.J. Park, F.G. Watson, and M.C. Smith \ldots
\end{quote}

Some styles only capitalize the first word of the title.  If you use any acronyms or
other words that should always be capitalized in titles, then they should be 
enclosed in {\tt \{\}}'s ({\em e.g.}, {\tt \{Fortran\}}, {\tt \{N\}ewton}).
This protects the case of these characters.

There are several other rules for \BibTeX{} listed in~\cite{lamport} which should be
referred to because they are not discussed here.

\section{Putting It All Together}
\label{BBL}
To aid the reader in understanding how all of this works together, the following 
excerpt was taken from Lamport~\cite{lamport}:
\begin{quotation}\singlespace
When you ran \LaTeX{} with the input file {\tt sample.tex}, you may have
noticed that \LaTeX{} created a file named {\tt sample.aux}.  This file,
called an {\em auxiliary} file, contains cross-referencing information.  Since
{\tt sample.tex} contains no cross-referencing commands, the auxiliary file it
produces has no information.  However, suppose that \LaTeX{} is run with an
input file named {\tt myfile.tex} that has citations and bibliography-making
[or referencing] commands.  The auxiliary file {\tt myfile.aux} that it produces
will contain all of the citation keys and the arguments of the {\tt \verb|\bibliography|}
and {\tt\verb|\bibliographystyle|} commands.  When \BibTeX{} is run, it reads
this information from the auxiliary file and produces a file named {\tt myfile.bbl}
containing \LaTeX{} commands to produce the source list \ldots The next time
\LaTeX{} is run on {\tt myfile.tex}, the {\tt \verb|\bibliography|} command reads
the {\tt bbl} file ({\tt myfile.bbl}), which generates the source list.
\end{quotation}

Thus, the command sequence for a source file called {\tt main.tex} which is going to
use \BibTeX{} would be:
\begin{quote}\singlespace\tt\begin{verbatim}
latex main.tex
bibtex main
latex main
latex main
\end{verbatim}\end{quote}
The first \LaTeX{} is to collect all of the citations for \BibTeX{}.  Then
\BibTeX{} is run to generate the bibliography.  \LaTeX{} is run again to
incorporate the bibliography into the document and the run the last time to
update any references (like pages in the Table of Contents) which changed when
the bibliography was included.
           % Chapter 4 From PU Thesis styles, by J.D. McCauley
% usage.tex
%
% This file explains how to use the withesis style
%   it is heavily modelled after a similar chapter by McCauley
%   for the Purdue Thesis style
%
% Eric Benedict, May 2000
%
% It is provided without warranty on an AS IS basis.


\chapter{Using the {\tt withesis} Style}

You can get a copy of the \LaTeX{} style for creating a University
of Wisconsin--Madison thesis or dissertation from:

{\tt http://www.cae.wisc.edu/\verb+~+benedict/LaTeX.html}

After somehow unpacking it, you will have the style files ({\tt withesis.sty}
{\tt withe10.sty}, and {\tt withe12.sty}) as well the files used to create
this document.  The files used for this document can be copied and used as a
template for your own thesis or dissertation.

The final printed form of this document is useful, but the
combination of the source code and final copy form a much more valuable
reference.  Keeping a working copy of the this document can be helpful
when you are later working on your thesis or disseration and want to know
how to do something.  If you find a similar example in this document,
then you can simply look at the corresponding source code and add it to
your document.    Because many parts of this document were written by
different people, the styles and techniques are also different and provide
different ways of achieving the same or similar results.

Because of the typical size of theses, it makes sense to break the document
up into several smaller files.  Usually this is done at the chapter level.
These files can then be {\tt \verb|\include|}d in a {\em root} file.  It is
the {\em root} file that you will run \LaTeX{} on.  For this manual, the
root file is called {\tt main.tex}.

\section{The Root File and the Preamble}
The {\tt \verb|\documentclass|} command is used to tell \LaTeX{} that you will
be using the {\tt withesis} document class and it is the first command in your
root file.  Class options such as {\tt 10pt}, {\tt 12pt}, {\tt msthesis} or
{\tt margincheck} are specified here:

{\tt \verb|\documentclass[12pt,msthesis]{withesis}|}

The class option {\tt msthesis} sets the margins to be appropriate for depositing
with the UW library, namely a 1.25 inch left margin with the remaining margins 1 inch.
The defaults for the title page are also defined for a thesis and for a Master of
Science degree.

The class option {\tt margincheck} will place a small black square at the end of
each line which exceeds the margins.\footnote{In reality, the square is
placed at the end of lines which exceed their {\tt \char92hbox}.  This usually
(but not always) indicates a  margin violation on the right margin.  Left
margin violations aren't indicated and if the margin violation is large enough,
there isn't room for the black box to be visiable.}  This is visible both in the {\tt .dvi} file
as well as in the {\tt .ps} file.

The area immediately following this command is called the {\em preamble} and is
used for things like including different style packages,
defining new macros and declaring the page style.

The style packages can be used to easily change the thesis font.  For example,
this document is set in Times Roman instead of the \LaTeX default of Computer
Modern.  This change was performed by including the {\tt times} package:

{\tt\verb|\usepackage{times}|}\footnote{In this document, the typewriter font
{\tt $\backslash$tt} was redefined to use the Computer Modern font with the command
{\tt $\backslash$renewcommand\{$\backslash$ttdefault\}\{cmtt\}}.  
For more information, see~\cite{goossens}.}

Remember that if you change the fonts from the default Computer Modern to
PostScript ({\em e.g.} Times Roman) then in order to correctly see the
document, you will need to convert the {\tt *.dvi} output into a {\tt *.ps}
file and view the document with a PostScript viewer. This is required since 
most {\tt *.dvi} previewer programs cannot 
display PostScript fonts.  Usually, the previewer will substitute
default fonts so the document may be viewed; however, since the alternate
fonts may not be the same size, the formatting of the document may appear
to be incorrect.

The style package for including Postscript figures, {\tt epsfig}, is included with

{\tt\verb|\usepackage{epsfig}|}

If multiple style packages are required, then they can be combined into one statement
as follows:

{\tt\verb|\usepackage{epsfig,times}|}

Many different style packages are available.  For more information, see~\cite{goossens}.

The page styles are defined using a similar method.
A special style is defined for the {\tt withesis} style:

{\tt\verb|\pagestyle{thesisdraft}|}

This style causes the footer text to become:

{\verb| DRAFT: Do Not Distribute        <time><Date>        <input file name>|}

This appears at the bottom of every page.

In addition to the page style command, the {\tt withesis} has defined several useful
commands which are specified in the preamble.  They include {\tt \verb| \draftmargin|},
{\tt \verb|\draftscreen|}, {\tt \verb|\noappendixtables|}, and
{\tt \verb|\noappendixfigures|}.

The command  {\tt \verb|\draftmargin|} draws a PostScript box with the dimensions of
the margins.  This makes it easy to check that the margins are correct and to see if
any of the text or figures are outside of the required margins.  This box is only visible
in the {\tt .ps} file since it is a PostScript special.


The command  {\tt \verb|\draftscreen|} draws a PostScript screen with the word {\em DRAFT}
in light grey and diagonally across the page.  This screen is only visible
in the {\tt .ps} file since it is a PostScript special.

The commands {\tt \verb|\noappendixtables|} and/or {\tt \verb|\noappendixfigures|} should
be used if the appendix does not have either tables or figures respectively.  These commands
inhibit the Appendix Table or Appendix Figure titles in the List of Tables or List of
Figures.\label{usage:noapp}


If you have specified the {\tt psfig} or {\tt epsfig} document style package, then a useful
command is {\tt \verb|\psdraft|}.  This command will show the bounding box that the figure
would occupy (instead of actually including the figure).  This speeds up the draft copy
printing, reduces toner usage and the drawn box is visible in the {\tt .dvi} file.

The next usual command is {\tt \verb|\begin{document}|}.  The following example is part
of the root file used for this manual.

\begin{quote} \singlespace\footnotesize\tt
\begin{verbatim}
\bibliographystyle{plain}
% prelude.tex
%   - titlepage
%   - dedication
%   - acknowledgments
%   - table of contents, list of tables and list of figures
%   - nomenclature
%   - abstract
%============================================================================


\clearpage\pagenumbering{roman}  % This makes the page numbers Roman (i, ii, etc)



% TITLE PAGE
%   - define \title{} \author{} \date{}
\title{How to \LaTeX\ a Thesis}
\author{Eric R. L. Benedict}
\date{2000}
%   - The default degree is ``Doctor of Philosophy''
%     (unless the document style msthesis is specified
%      and then the default degree is ``Master of Science'')
%     Degree can be changed using the command \degree{}
\degree{Master \TeX nician}
%   - The default is dissertation, unless the document style
%     msthesis was specified in which case it becomes thesis.
%     If msthesis is specified for the MS margins, you can
%     still have a dissertation if you specify \disseration
%\disseration
%   - for a masters project report, specify \project
%\project
%   - for a preliminary report, specify \prelim
\prelim
%   - for a masters thesis, specify \thesis
%\thesis
%   - The default department is ``Electrical Engineering''
%     The department can be changed using the command \department{}
%\department{New Department}
%   - once the above are defined, use \maketitle to generate the titlepage
\maketitle

% COPYRIGHT PAGE
%   - To include a copyright page use \copyrightpage
\copyrightpage

% DEDICATION
\begin{dedication}
To my pet rock, Skippy.
\end{dedication}

% ACKNOWLEDGMENTS
\begin{acknowledgments}
I thank the many people who have done lots of nice things for me.
\end{acknowledgments}

% CONTENTS, TABLES, FIGURES
\tableofcontents
\listoftables
\listoffigures

% NOMENCLATURE
\begin{nomenclature}
\begin{description}
\item{\makebox[0.75in][l]{\TeX}}
       \parbox[t]{5in}{a typesetting system by Donald Knuth~\cite{knuth}.  It
       also refers to the ``plain'' format.  The proper pronounciation
       rhymes with ``heck'' and ``peck'' and does not sound like
       ``hex'' or ``Rex.''\\}

\item{\makebox[0.75in][l]{\LaTeX}}  
        \parbox[t]{5in}{a set of \TeX{} macros originally written by Leslie 
        Lamport~\cite{lamport}.  The proper pronunciation is 
        {\tt l\={a}$\cdot$tek'} and not {\tt l\={a}'$\cdot$teks} (see above).\\}

\item{\makebox[0.75in][l]{{\sc Bib}\TeX}} 
         \parbox[t]{5in}{a bibliography generation program by Oren 
                Patashnik~\cite{lamport}
                that can be used with either plain \TeX{} or \LaTeX{}.\\}

\item{\makebox[0.75in][l]{$C_1$}} Constant 1

\item{\makebox[0.75in][l]{$V$}}    Voltage 

\item{\makebox[0.75in][l]{\$}}     US Dollars
\end{description}
\end{nomenclature}


\advisorname{Bucky J. Badger}
\advisortitle{Assistant Professor}
% ABSTRACT
\begin{umiabstract}
  \input{abstract}
\end{umiabstract}

\begin{abstract}
  \input{abstract}
\end{abstract}


\clearpage\pagenumbering{arabic} % This makes the page numbers Arabic (1, 2, etc)
        % Title page, abstract, table of contents, etc
% Pre-lim
% by Eric Benedict


\chapter{Introducing the {\tt withesis} \LaTeX{} Style Guide}
This manual is was written to test the {\tt withesis} style
file and to provide documentation for this style file.  

\section{History}
The
idea for this came from a similar manual written by James Darrell
McCauley and Scott Hucker in 1993 for the Purdue University thesis
style file.  Content ideas were liberally borrowed from this document.
The {\tt withesis} style file is based on the Purdue thesis file
written by Dave Kraynie and edited by Darrell McCauley.  This base was
edited to meet the format requirements of the University of 
Wisconsin--Madison and several additional new commands were created.
In addition, environments from the UW Mathematics Department were also
incorporated.

\section{Producing Your Thesis or Dissertation}
The {\tt withesis} style file will take care of most of the formatting
requirements for submitting your thesis or dissertation at the University
of Wisconsin-Madison.  There are some requirements on the printing of your
document.  From the Graduate School's {\em UW-Madison Guide To Preparing 
Your Doctoral Dissertation},
\begin{quote}\singlespace
Print your dissertation on a laser printer. (Some high quality dot-matrix
printers may be acceptable.) The printer must produce output that
meets all format and legibility requirements. A professional copy shop
can produce an acceptable copy to be submitted to the Graduate School.
Some copiers enlarge the original between one and two percent. To avoid
problems with margins, produce the original copy with margins larger than
the required minimum. Look carefully at the copy before paying for the
services and ask for pages to be recopied if necessary. Common flaws are:
smudges, copy lines, specks, missing pages, margin shifts, slanting of
the printed image on the page, and poor paper quality.
\end{quote}

\subsection{Required Paper}
The paper which is used for PhD Dissertations should be:
\begin{itemize}
\item 8-1/2 x 11 inches
\item High-quality, white
\item 20 pound weight, bond
\end{itemize}
 
While for Masters Theses, the paper should be:

\begin{itemize}
\item 8-1/2 x 11 inches
\item White
\item Acid-free or pH neutral
\item 20 pound weight
\item 25\% cotton bond minimum
\end{itemize}

Paper that meets these requirements can be purchased at book and stationery
stores.

\subsection{Copyright Page}
\label{copyright}
If you choose to retain and register copyright of the dissertation, prepare
a copyright page using the {\tt withesis} {\tt \verb|\copyrightpage|} command. 
Center the text in the bottom third of the page within the dissertation
margins. This page is not numbered. There is an additional fee for copyrighting
your dissertation which is payable at the bursars office along with the
microfilming and binding fee.

\subsection{Prechecks}
The Graduate School has reserved 9:00-9:30 each morning to answer specific formatting questions
(for example: use of tables, graphs and charts). You may bring in 8-10
pages to be reviewed. No appointment is necessary.

\subsection{Final Checks}
\sloppypar
For information about the final Graduate School review and about depositing
your dissertation in the library, see {\em The Three D's: Deadlines, Defending, 
Depositing Your Doctoral Dissertation} or look
at the web site 
\begin{quote}
{\tt http://www.wisc.edu/grad/gs/degrees/ddd.html}
\end{quote}

\section{Disclaimer}
This software and documentation is provided ``as is'' without any
express or implied warranty.
While care has been taken by the authors of this style file such that the
final product will probably meet the University of Wisconsin's formatting 
requirements this is not guaranteed. 
          % Chapter 1
\include{essentials}     % Edited ``Essential LaTeX'' by Jon Warbrick
\chapter{Figures and Tables}\label{quad}
This chapter\footnote{Most of the text in this chapter's introduction is from {\em How to
\TeX{} a Thesis: The Purdue Thesis Styles}} shows some example ways of incorporating tables and figures into \LaTeX{}.
Special environments exist for tables and figures and are special because they are
allowed to {\em float}---that is, \LaTeX{} doesn't always put them in the exact place
that they occur in your input file.  An algorithm is used to place the floating environments,
or floats, at locations which are typographically correct.  This may cause endless frustration
if you want to have a figure or table occur at a specific location.  There are a few
methods for solving this.

You can exert some influence on \LaTeX{}'s float placement algorithm by using
{\em float position specifiers}.  These specifiers, listed below, tell \LaTeX{}
what you prefer.
\begin{tabbing}
{\tt hhhhhh} \= ``bottom'' \=  \kill
{\tt h}\> ``here'' \> do not move this object \\
{\tt p}\> ``page'' \> put this object on a page of floats \\
{\tt b}\> ``bottom'' \> put this object at the bottom of a page\\
{\tt t}\> ``top'' \> put this object at the top of a page\\
\end{tabbing}

Any combination of these can be used:
\begin{quote}\tt\singlespace\begin{verbatim}
\begin{figure}[htbp]
 ...
\caption{A Figure!}
\end{figure}
\end{verbatim}\end{quote}

In this example, we asked \LaTeX{} to ``put the figure `here' if possible.  If it
is not possible (according to the rule encoded in the float algorithm), put it on the
next float page.  A float page is a page which contains nothing but floating objects,
{\em e.g.} a page of nothing but figures or tables.  If this isn't possible, try to put it
at the `top' of a page.  The last thing to try is to put the figure at the `bottom' of
a page.''

The remainder of this chapter deals with some examples of what to put into the figure,
the ellipsis (\ldots ) in the example above.

\section{Tables}
Table~\ref{pde.tab1} is an example table from the UW Math Department.
\begin{table}[htbp]
\centering
\caption{PDE solve times, $15^3+1$
equations.\label{pde.tab1}}
\begin{tabular}{||l|l|l|l|l|l||}\hline
Precond. & Time & Nonlinear & Krylov
& Function & Precond. \\
 & & Iterations & Iterations & calls & solves \\ \hline
None & 1260.9u & 3 & 26 & 30 & 0  \\
 &(21:09) & & & &  \\ \hline
FFT  & 983.4u & 2  & 5  & 8  & 7 \\
&(16:31) & & & & \\ \hline
\end{tabular}
\end{table}
The code to generate it is as follows:
\begin{quote}\tt\singlespace\begin{verbatim}
\begin{table}[htbp]
\centering
\caption{PDE solve times, $15^3+1$
equations.\label{pde.tab1}}
\begin{tabular}{||l|l|l|l|l|l||}\hline
Precond. & Time & Nonlinear & Krylov
& Function & Precond. \\
 & & Iterations & Iterations & calls & solves \\ \hline
None & 1260.9u & 3 & 26 & 30 & 0  \\
 &(21:09) & & & &  \\ \hline
FFT  & 983.4u & 2  & 5  & 8  & 7 \\
&(16:31) & & & & \\ \hline
\end{tabular}
\end{table}
\end{verbatim}\end{quote}

\section{Figures}
There are many different ways to incorporate figures into a \LaTeX{}
document.  \LaTeX{} has an internal {\tt picture} environment and
some programs will generate files which are in this format and can
be simply {\tt include}d.  In addition to \LaTeX{} native {\tt picture}
format, additional packages can be loaded in the {\tt\verb|\documentstyle|}
command (or using the {\tt input} command) to allow \LaTeX{} to process
non-native formats such as PostScript.

\subsection{\tt gnuplot}
The graph of Figure~\ref{gelfand.fig2}
 was created by gnuplot. For simple graphs this is a
 great utility.  For example, if you want a sin curve in your thesis
 try the following:
\begin{quote}\tt\singlespace\begin{verbatim}
 (terminal window): gnuplot
 (in gnuplot):
                 set terminal latex
                 set output "foo.tex"
                 plot sin(x)
                 quit
\end{verbatim}\end{quote}
This will generate a file called {\tt foo.tex} which can be read in
with the following statements.
\begin{figure}[htbp]
\centering
\input{fig2.tex}
\caption{Gelfand equation on the ball, $3\leq n \leq 9$.
\label{gelfand.fig2}}
\end{figure}
\begin{quote}\tt\singlespace\begin{verbatim}
\begin{figure}[htbp]
\centering
\input{fig2.tex}
\caption{Gelfand equation on the ball, $3\leq n \leq 9$.
\label{gelfand.fig2}}
\end{figure}
\end{verbatim}\end{quote}
One advantage to using the native \LaTeX{} {\tt picture} environment
is that the fonts will be assured to agree and the pictures can be viewed
in the {\tt .dvi} viewer.

\subsection{PostScript}
Many drawing applications now allow the export of a graphic to the
{\em Encapsulated PostScript} format.  These files have a suffix of
{\tt .EPS} or {\tt .EPSF} and are similar to a regular PostScript
file except that they contain a {\em bounding box} which describes
the dimensions of the figure.

In order to include PostScript figures, the {\tt epsfig} (or {\tt psfig}
depending on the system you are using) style file must be included in either
the {\tt\verb|\documentstyle|} command or the preamble using the {\tt input} command.

Figure~\ref{vwcontr} is a plot from Matlab.
\begin{figure}[htbp]
\centerline{
\psfig{figure=vwcontr.eps,width=5in,angle=0}
           }
\caption{$\sigma$ as a Function of Voltage and Speed, $\alpha = 20$}
\label{vwcontr}
\end{figure}
The commands to include this figure are
\begin{quote}\tt\singlespace\begin{verbatim}
\begin{figure}[htbp]
\centerline{
\psfig{figure=vwcontr.ps,width=5in,angle=0}
           }
\caption{$\sigma$ as a Function of Voltage and Speed, $\alpha = 20$}
\label{vwcontr}
\end{figure}
\end{verbatim}\end{quote}

Observe that the {\tt \verb|\psfig|} command allows the scaling of the figure
by setting either the {\tt width} or {\tt height} of the figure.  If only one
dimension is specified, the other is computed to keep the same aspect ratio.
The figure can also be rotated by setting {\tt angle} to the desired value in
degrees.
           % Chapter 3 Edited from UW Math Dept's Sample Thesis
% bibs.tex
%
% This chapter briefly talks about BibTex and is mostly
% copied from a similar chapter from "How to TeX a Thesis:
% The Purdue Thesis Styles" by James Darrell McCauley and
% Scott Hucker
%

\newcommand{\BibTeX}{{\sc Bib}\TeX}

\chapter{Citations and Bibliographies}
This chapter is an edited form of the same chapter from {\em How to 
\TeX{} a Thesis: The Purdue Thesis Styles} by James Darrell McCauley and
Scott Hucker.

The task of compiling and formatting the sources cited in papers can
be quite tedious, especially for large documents like theses.  A program
separate from \LaTeX{}, called ``\BibTeX{},''can be used to automate this task~\cite{lamport}.

\section{The Citation Command}
When referring to the work of someone else, the {\tt \verb|\cite|} command is used.
This generates the citation in the text for you.  In the above paragraph, the command
{\tt \verb|\cite{lamport}|} was used after the word ``task.''  The formatting of your
citation is handled by either the document style or a style option.  The default citation
style uses the number system (a number in square brackets).  Other citation styles
may use the author-date system, (Lamport, 1986) or the superscript$^3$ system.

\section{Bibliography Styles}
The way that a reference is formatted in your bibliography depends on the bibliography
style, which is specified near the beginning of your document with the\break
{\tt \verb|\bibliographstyle{file}|} command.  The file {\tt file.bst} is the name of the 
bibliography style file.  Standard \BibTeX{} bibliography style files include {\tt plain},
{\tt unsrt}, {\tt alpha}, and {\tt abbrev}.  The bibliography style governs whether or not
references are sorted, whether first names or initials are used for authors, whether or 
not last names are listed first, the location of the year in the references (after the
author or at the end of the reference), {\em etc.}.  You may be required by your
department or major professor to follow as style for a particular journal.  If so, then you
will need to find a \BibTeX{} style file to suit your needs.  Most major journals have
style files.  If you cannot locate an appropriate \BibTeX{} style file, then choose the
one which is closest and then edit the {\tt .bbl} file by hand.  See Section~\ref{BBL}
for a brief discussion on the {\tt .bbl} file.  Some common, but non-standard \BibTeX{}
styles include
\begin{tabbing}
{\tt jacs-new.bstxxxx}\= {\em Journal of the American Chemical Society}\kill
{\tt acm.bst}\>The Association for Computing Machinery\\
{\tt ieeetr.bst}\> The {\em IEEE Transactions} style\\
{\tt jacs-new.bst}\> {\em Journal of the American Chemical Society}
\end{tabbing}

\section{The Database}
The  {\tt \verb|\bibliography{file}|} command is placed in your input file at the location
where the ``List of References'' section\footnote{or ``Bibliography'' 
if {\tt \char92 altbibtitle } has been specified in the preamble.} would be.  It specifies the name (or names) of
your bibliographic data base, {\tt file.bib}.  An example entry in a \BibTeX{}
database is:
\begin{quote}\singlespace\tt\begin{verbatim}
@book{ lamport86 ,
     author =    "Leslie Lamport" ,
     title =     "\LaTeX: A Document Preparation System" ,
     publisher = "Addison--Wesley Pub.\ Co." ,
     year =      "1986" ,
     address =   "Reading, MA" 
}
\end{verbatim}\end{quote}

The citation key is the first field in this entry--- citing this book in a \LaTeX{}
file would look like
\begin{quote}\singlespace\tt\begin{verbatim}
According to Lamport~\cite{lamport86} ...
\end{verbatim}\end{quote}
The tilde ({\tt \verb|~|}) is used to tie the word ``Lamport'' to the citation
generated.  The space between these words is then unbreakable---the word ``Lamport''
and the citation \cite{lamport} will not be split across two lines if they happen to occur
near the end of a line.

A listing of all entry types with their required and optional fields is given in 
Appendix~\ref{bibrefs}. There are several tools which exist to help in editing a \BibTeX{}
file, however, their use is beyond the scope of this manual and can be found by searching
the net.  You can simply use a plain text editor like {\tt vi} or {\tt WordPad} to edit
and create the database files.

There are several rules which you must follow when creating your database.  Authors are
always listed by their full names, first name first, and multiple authors are separated
by {\tt and}.  For example
\begin{quote}\singlespace\tt\begin{verbatim}
author = "John Jay Park and Frederick Gene Watson and
          Michelle Catherine Smith",
\end{verbatim}\end{quote}
If you were using {\tt abbrv} as your {\tt bibliographystyle}, a reference for these
authors may look like:
\begin{quote}
J.J. Park, F.G. Watson, and M.C. Smith \ldots
\end{quote}

Some styles only capitalize the first word of the title.  If you use any acronyms or
other words that should always be capitalized in titles, then they should be 
enclosed in {\tt \{\}}'s ({\em e.g.}, {\tt \{Fortran\}}, {\tt \{N\}ewton}).
This protects the case of these characters.

There are several other rules for \BibTeX{} listed in~\cite{lamport} which should be
referred to because they are not discussed here.

\section{Putting It All Together}
\label{BBL}
To aid the reader in understanding how all of this works together, the following 
excerpt was taken from Lamport~\cite{lamport}:
\begin{quotation}\singlespace
When you ran \LaTeX{} with the input file {\tt sample.tex}, you may have
noticed that \LaTeX{} created a file named {\tt sample.aux}.  This file,
called an {\em auxiliary} file, contains cross-referencing information.  Since
{\tt sample.tex} contains no cross-referencing commands, the auxiliary file it
produces has no information.  However, suppose that \LaTeX{} is run with an
input file named {\tt myfile.tex} that has citations and bibliography-making
[or referencing] commands.  The auxiliary file {\tt myfile.aux} that it produces
will contain all of the citation keys and the arguments of the {\tt \verb|\bibliography|}
and {\tt\verb|\bibliographystyle|} commands.  When \BibTeX{} is run, it reads
this information from the auxiliary file and produces a file named {\tt myfile.bbl}
containing \LaTeX{} commands to produce the source list \ldots The next time
\LaTeX{} is run on {\tt myfile.tex}, the {\tt \verb|\bibliography|} command reads
the {\tt bbl} file ({\tt myfile.bbl}), which generates the source list.
\end{quotation}

Thus, the command sequence for a source file called {\tt main.tex} which is going to
use \BibTeX{} would be:
\begin{quote}\singlespace\tt\begin{verbatim}
latex main.tex
bibtex main
latex main
latex main
\end{verbatim}\end{quote}
The first \LaTeX{} is to collect all of the citations for \BibTeX{}.  Then
\BibTeX{} is run to generate the bibliography.  \LaTeX{} is run again to
incorporate the bibliography into the document and the run the last time to
update any references (like pages in the Table of Contents) which changed when
the bibliography was included.
           % Chapter 4 From PU Thesis styles, by J.D. McCauley
% usage.tex
%
% This file explains how to use the withesis style
%   it is heavily modelled after a similar chapter by McCauley
%   for the Purdue Thesis style
%
% Eric Benedict, May 2000
%
% It is provided without warranty on an AS IS basis.


\chapter{Using the {\tt withesis} Style}

You can get a copy of the \LaTeX{} style for creating a University
of Wisconsin--Madison thesis or dissertation from:

{\tt http://www.cae.wisc.edu/\verb+~+benedict/LaTeX.html}

After somehow unpacking it, you will have the style files ({\tt withesis.sty}
{\tt withe10.sty}, and {\tt withe12.sty}) as well the files used to create
this document.  The files used for this document can be copied and used as a
template for your own thesis or dissertation.

The final printed form of this document is useful, but the
combination of the source code and final copy form a much more valuable
reference.  Keeping a working copy of the this document can be helpful
when you are later working on your thesis or disseration and want to know
how to do something.  If you find a similar example in this document,
then you can simply look at the corresponding source code and add it to
your document.    Because many parts of this document were written by
different people, the styles and techniques are also different and provide
different ways of achieving the same or similar results.

Because of the typical size of theses, it makes sense to break the document
up into several smaller files.  Usually this is done at the chapter level.
These files can then be {\tt \verb|\include|}d in a {\em root} file.  It is
the {\em root} file that you will run \LaTeX{} on.  For this manual, the
root file is called {\tt main.tex}.

\section{The Root File and the Preamble}
The {\tt \verb|\documentclass|} command is used to tell \LaTeX{} that you will
be using the {\tt withesis} document class and it is the first command in your
root file.  Class options such as {\tt 10pt}, {\tt 12pt}, {\tt msthesis} or
{\tt margincheck} are specified here:

{\tt \verb|\documentclass[12pt,msthesis]{withesis}|}

The class option {\tt msthesis} sets the margins to be appropriate for depositing
with the UW library, namely a 1.25 inch left margin with the remaining margins 1 inch.
The defaults for the title page are also defined for a thesis and for a Master of
Science degree.

The class option {\tt margincheck} will place a small black square at the end of
each line which exceeds the margins.\footnote{In reality, the square is
placed at the end of lines which exceed their {\tt \char92hbox}.  This usually
(but not always) indicates a  margin violation on the right margin.  Left
margin violations aren't indicated and if the margin violation is large enough,
there isn't room for the black box to be visiable.}  This is visible both in the {\tt .dvi} file
as well as in the {\tt .ps} file.

The area immediately following this command is called the {\em preamble} and is
used for things like including different style packages,
defining new macros and declaring the page style.

The style packages can be used to easily change the thesis font.  For example,
this document is set in Times Roman instead of the \LaTeX default of Computer
Modern.  This change was performed by including the {\tt times} package:

{\tt\verb|\usepackage{times}|}\footnote{In this document, the typewriter font
{\tt $\backslash$tt} was redefined to use the Computer Modern font with the command
{\tt $\backslash$renewcommand\{$\backslash$ttdefault\}\{cmtt\}}.  
For more information, see~\cite{goossens}.}

Remember that if you change the fonts from the default Computer Modern to
PostScript ({\em e.g.} Times Roman) then in order to correctly see the
document, you will need to convert the {\tt *.dvi} output into a {\tt *.ps}
file and view the document with a PostScript viewer. This is required since 
most {\tt *.dvi} previewer programs cannot 
display PostScript fonts.  Usually, the previewer will substitute
default fonts so the document may be viewed; however, since the alternate
fonts may not be the same size, the formatting of the document may appear
to be incorrect.

The style package for including Postscript figures, {\tt epsfig}, is included with

{\tt\verb|\usepackage{epsfig}|}

If multiple style packages are required, then they can be combined into one statement
as follows:

{\tt\verb|\usepackage{epsfig,times}|}

Many different style packages are available.  For more information, see~\cite{goossens}.

The page styles are defined using a similar method.
A special style is defined for the {\tt withesis} style:

{\tt\verb|\pagestyle{thesisdraft}|}

This style causes the footer text to become:

{\verb| DRAFT: Do Not Distribute        <time><Date>        <input file name>|}

This appears at the bottom of every page.

In addition to the page style command, the {\tt withesis} has defined several useful
commands which are specified in the preamble.  They include {\tt \verb| \draftmargin|},
{\tt \verb|\draftscreen|}, {\tt \verb|\noappendixtables|}, and
{\tt \verb|\noappendixfigures|}.

The command  {\tt \verb|\draftmargin|} draws a PostScript box with the dimensions of
the margins.  This makes it easy to check that the margins are correct and to see if
any of the text or figures are outside of the required margins.  This box is only visible
in the {\tt .ps} file since it is a PostScript special.


The command  {\tt \verb|\draftscreen|} draws a PostScript screen with the word {\em DRAFT}
in light grey and diagonally across the page.  This screen is only visible
in the {\tt .ps} file since it is a PostScript special.

The commands {\tt \verb|\noappendixtables|} and/or {\tt \verb|\noappendixfigures|} should
be used if the appendix does not have either tables or figures respectively.  These commands
inhibit the Appendix Table or Appendix Figure titles in the List of Tables or List of
Figures.\label{usage:noapp}


If you have specified the {\tt psfig} or {\tt epsfig} document style package, then a useful
command is {\tt \verb|\psdraft|}.  This command will show the bounding box that the figure
would occupy (instead of actually including the figure).  This speeds up the draft copy
printing, reduces toner usage and the drawn box is visible in the {\tt .dvi} file.

The next usual command is {\tt \verb|\begin{document}|}.  The following example is part
of the root file used for this manual.

\begin{quote} \singlespace\footnotesize\tt
\begin{verbatim}
\bibliographystyle{plain}
\include{prelude}        % Title page, abstract, table of contents, etc
\include{intro}          % Chapter 1
\include{essentials}     % Edited ``Essential LaTeX'' by Jon Warbrick
\include{figs}           % Chapter 3 Edited from UW Math Dept's Sample Thesis
\include{bibs}           % Chapter 4 From PU Thesis styles, by J.D. McCauley
\include{usage}          % Chapter 5 Strongly based on similar by J.D. McCauley
\bibliography{refs}      % Make the bibliography
\begin{appendices}       % Start of the Appendix Chapters.  If there is only
                         % one Appendix Chapter, then use \begin{appendix}
\include{code}         % Including computer code listings
\include{bibref}         % a BibTeX reference
\include{math}           % Complex Equations from the UW Math Department
\include{acro}           % A discussion on generating PDF files.
\end{appendices}         % End of the Appendix Chapters.  ibid on \end{appendix}
%\include{vita}          % Optional Vita, use \begin{vita} vita text \end{vita}
\end{document}
\end{verbatim}
\end{quote}

\section{Prelude}
After the {\tt \verb|\begin{document}|} comes the preliminary information found in
theses.  In this manual, the information is kept in the file {\tt prelude.tex} (see
above).  These pages will need to be numbered with roman numerals, so use
\begin{quote}\tt\singlespace\begin{verbatim}
\clearpage\pagenumbering{roman}
\end{verbatim}\end{quote}

Next, comes your thesis or dissertation title, your name, date of graduation, department
and degree.
\begin{quote}\tt\singlespace\begin{verbatim}
\title{How to \LaTeX\ a Thesis}
\author{Eric R. Benedict}
\date{2000}
%   - The default degree is ``Doctor of Philosophy''
%     Degree can be changed using the command \degree{}
%\degree{New Degree}
%   - for a PhD dissertation (default), specify \dissertation
%\dissertation
%   - for a masters project report, specify \project
%\project
%   - for a preliminary report, specify \prelim
%\prelim
%   - for a masters thesis, specify \thesis
%\thesis
%   - The default department is ``Electrical Engineering''
%     The department can be changed using the command \department{}
%\department{New Department}
\end{verbatim}\end{quote}

If you specified the class option {\tt msthesis}, then the degree is changed to
{\em Master \break of Science} and the {\tt \verb|\thesis|} option is specified.  If you
want to have the masters margins with another document, then the {\tt \verb|\degree|}
and {\tt \verb|\dissertation|},  {\tt \verb|\project|}, {\em etc.\/} can be specified
as needed.

Once the
above are all defined, use  {\tt \verb|\maketitle|} to generate the title page.
\begin{quote}\tt\singlespace\begin{verbatim}
\maketitle
\end{verbatim}\end{quote}

If you wish to include a copyright page (see Section~\ref{copyright} for
information on registering the copyright.), then add the command
\begin{quote}\tt\singlespace\begin{verbatim}
\copyrightpage
\end{verbatim}\end{quote}
This will generate the proper copyright page and will use the name and date specified
in {\tt \verb|\author{}|} and {\tt \verb|\date{}|}.

Next are the dedications and acknowledgements:
\begin{quote}\tt\singlespace\begin{verbatim}
\begin{dedication}
To my pet rock, Skippy.
\end{dedication}

\begin{acknowledgments}
I thank the many people who have done lots of nice things for me.
\end{acknowledgments}
\end{verbatim}\end{quote}

You must tell \LaTeX{} to generate a table of contents, a list of tables and a list of
figures:
\begin{quote}\tt\singlespace\begin{verbatim}
\tableofcontents
\listoftables
\listoffigures
\end{verbatim}\end{quote}

If you wish to have a nomenclature, list of symbols or glossary it can go here.
\begin{quote}\tt\singlespace\begin{verbatim}
\begin{nomenclature}
%\begin{listofsymbols}
%\begin{glossary}
\begin{tabular}{ll}
$C_1$ & Constant 1\\
\ldots
\end{tabular}
%\end{glossary}
%\end{listofsymbols}
\end{nomenclature}
\end{verbatim}\end{quote}

If your abstract will be microfilmed by Bell and Howell (formerly UMI), then you
will need to generate an abstract of less than 350 words.  This abstract can be created
using the {\tt umiabstract} environment.  This environment requires that you define your
advisor and your advisor's title using {\tt \verb|\advisorname{}|} and
{\tt \verb|\advisortitle{}|}.
\begin{quote}\tt\singlespace\begin{verbatim}
\advisorname{Bucky J. Badger}
\advisortitle{Assistant Professor}
% ABSTRACT
\begin{umiabstract}
\noindent       % Don't indent first paragraph.
This explains the basics for using \LaTeX\ to typeset a
dissertation, thesis or project report for the University of
Wisconsin-Madison.

...

\end{umiabstract}
\end{verbatim}\end{quote}
This will place your name, title and required text at the top of the page and follow the
abstract text with your advisor's name at the bottom for your advisor's signature.  This
page is not numbered and would be submitted separately.

If you will have an abstract as part of your document, then the {\tt abstract} environment
should be used.
\begin{quote}\tt\singlespace\begin{verbatim}
\begin{abstract}
\noindent       % Don't indent first paragraph.
This explains the basics for using \LaTeX\ to typeset a
dissertation, thesis or project report for the University of
Wisconsin-Madison.

...

\end{abstract}
\end{verbatim}\end{quote}
This will generate a page number and it will be included in the Table
of Contents.  

If you will have both the UMI and regular abstracts like this document, then
you will probably want to write the abstract once and save it in a seperate
file such as {\tt abstract.tex}.  Then, you can use the same abstract for
both purposes.

\begin{quote}\begin{verbatim}
\begin{umiabstract}
  \input{abstract}
\end{umiabstract}

\begin{abstract}
  \input{abstract}
\end{abstract}
\end{verbatim}\end{quote}

Finally, the page numbers must be changed to arabic numbers to conclude the preliminary
portion of the document.
\begin{quote}\tt\singlespace\begin{verbatim}
\clearpage\pagenumbering{arabic}
\end{verbatim}\end{quote}

\section{The Body}
At the beginning of {\tt intro.tex} there is the following command:
\begin{quote}\tt\singlespace\begin{verbatim}
\chapter{Introducing the {\tt withesis} \LaTeX{} Style Guide}
\end{verbatim}\end{quote}
Following that is the text of the chapter.  The body of your thesis is separated by
sectioning commands like {\tt \verb|\chapter{}|}.  For more information on the sectioning
commands, see Section~\ref{ess:sectioning}.

Remember the basic rule of outlining you learned in grammar school:
\begin{quote}
You cannot have an `A' if you do not have a `B'
\end{quote}
Take care to have at least two {\tt \verb|\section|}s if you use the command; have
two {\tt \verb|\subsection|}s, {\em etc}.



\section{Additional Theorem Like Environments}
The {\tt withesis} style adds numerous additional theorem like environments.  These
environments were included to allow compatibility with the University of Wisconsin's
Math Department's style file.  These environments are
{\tt theorem}, {\tt assertion}, {\tt claim}, {\tt conjecture}, {\tt corollary},
{\tt definition}, {\tt example}, {\tt figger}, {\tt lemma}, {\tt prop} and {\tt remark}.

As an example, consider the following.
\begin{lemma}
Assuming that $\partial\Omega_2 = \emptyset$ and that $h(t) = 1$, we
have $$
\begin{array}{lr}
\Delta u = f, &  x\in\Omega ,\\[2pt]
u =  g_1, &  x\in\partial\Omega .
\end{array}
$$
\end{lemma}
which was produced with the following:
\begin{quote}\tt\singlespace\begin{verbatim}
\begin{lemma}
Assuming that $\partial\Omega_2 = \emptyset$ and that $h(t) = 1$, we
have $$
\begin{array}{lr}
\Delta u = f, &  x\in\Omega ,\\[2pt]
 u =  g_1, &  x\in\partial\Omega .
\end{array}
$$
\end{lemma}
\end{verbatim}\end{quote}

\section{Bibliography or References}
As a final note, the default title for the references chapter is ``LIST OF REFERENCES.''
Since some people may prefer ``BIBLIOGRAPHY'', the command
\break{\tt \verb|\altbibtitle|}
has been added to change the chapter title.

\section{Appendices}
There are two commands which are available to suppress the writing of the auxiliary information
(to the {\tt .lot} and {\tt .lof} files).  They are:
\begin{quote}\tt\singlespace\begin{verbatim}
\noappendixtables                % Don't have appendix tables
\noappendixfigures               % Don't have appendix figures
\end{verbatim}\end{quote}
These commands should be in the preamble.  See Section~\ref{usage:noapp}.

There are two environments for doing the appendix chapter: {\tt appendix} and
\break {\tt appendices}.  If you have only one chapter in the appendix, use the {\tt appendix}
environment.  If you have more than one chapter, like this manual, use the
{\tt appendices} environment.
\begin{quote}\tt\singlespace\footnotesize\begin{verbatim}
\begin{appendices}  % Start of the Appendix Chapters.  If there is only
                    % one Appendix Chapter, then use \begin{appendix}
\include{code}      % Including computer code listings
\include{bibref}    % a BibTeX reference
\include{math}      % Complex Equations from the UW Math Department

\end{appendices}    % End of the Appendix Chapters. ibid on \end{appendix}
\end{verbatim}\end{quote}
The difference between these two environments is the way that the chapter header is
created and how this is listed in the table of contents.
          % Chapter 5 Strongly based on similar by J.D. McCauley
\bibliography{refs}      % Make the bibliography
\begin{appendices}       % Start of the Appendix Chapters.  If there is only
                         % one Appendix Chapter, then use \begin{appendix}
% code.tex
% this file is part of the example UW-Madison Thesis document
% It demonstrates one method for incorporating program listings
% into a document.

\chapter{Matlab Code} \label{matlab}
This is an example of a Matlab m-file.
\verbatimfile{derivs.m}
         % Including computer code listings
\chapter{Bib\TeX\ Entries}
\label{bibrefs}
The following shows the fields required in all types of Bib\TeX\ entries.
Fields with {\tt OPT} prefixed are optional (the three letters {\tt OPT} should 
not be used).  If an optional field is not used, then the entire field can be deleted.

{\tt
\singlespace
\begin{verbatim}

@Unpublished{,                            @Manual{,
  author =      "",                         title =           "",
  title =       "",                         OPTauthor =       "",
  note =        "",                         OPTorganization = "",
  OPTyear =     "",                         OPTaddress =      "",
  OPTmonth =    ""                          OPTedition =      "",
}                                           OPTyear =         "",
                                            OPTmonth =        "",
@TechReport{,                               OPTnote =         "" 
  author =      "",                       }
  title =       "",
  institution = "",                       @InProceedings{,
  year =        "",                         author =          "",
  OPTtype =     "",                         title =           "",
  OPTnumber =   "",                         booktitle =       "",
  OPTaddress =  "",                         year =            "",
  OPTmonth =    "",                         OPTeditor =       "",
  OPTnote =     ""                          OPTpages =        "",
}                                           OPTorganization = "",
                                            OPTpublisher =    "",
@Proceedings{,                              OPTaddress =      "",
  title =           "",                     OPTmonth =        "",
  year =            "",                     OPTnote =         "" 
  OPTeditor =       "",                   }
  OPTpublisher =    "",
  OPTorganization = "",
  OPTaddress =      "",
  OPTmonth =        "",
  OPTnote =         "" 
}



@PhDThesis{,                              @InCollection{,
  author =      "",                         author =          "",
  title =       "",                         title =           "",
  school =      "",                         booktitle =       "",
  year =        "",                         publisher =       "",
  OPTaddress =  "",                         year =            "",
  OPTmonth =    "",                         OPTeditor =       "",
  OPTnote =     ""                          OPTchapter =      "",
}                                           OPTpages =        "",
                                            OPTaddress =      "",
                                            OPTmonth =        "",
                                            OPTnote =         ""
                                          }

 
@Misc{,                                   @InCollection{,
  OPTauthor =       "",                     author =          "",
  OPTtitle =        "",                     title =           "",
  OPThowpublished = "",                     chapter =         "",
  OPTyear =         "",                     publisher =       "",
  OPTmonth =        "",                     year =            "",
  OPTnote =         ""                      OPTeditor =       "",
}                                           OPTpages =        "",
}                                           OPTvolume =       "",
                                            OPTseries =       "",
                                            OPTaddress =      "",
                                            OPTedition =      "",
                                            OPTmonth =        "",
                                            OPTnote =         ""
                                          }

@MastersThesis{,                          @Article{,
  author =      "",                         author =          "",
  title =       "",                         title =           "",
  school =      "",                         journal =         "",
  year =        "",                         year =            "",
  OPTaddress =  "",                         OPTvolume =       "",
  OPTmonth =    "",                         OPTnumber =       "",
  OPTnote =     ""                          OPTpages =        "",
}                                           OPTmonth =        "",
                                            OPTnote =         ""
                                           }\end{verbatim} }
         % a BibTeX reference
\chapter{Mathematics Examples}
This appendix provides an example of \LaTeX's typesetting
capabilities.  Most of text was obtained from the University of
Wisconsin-Madison Math Department's example thesis file.

\section{Matrices}
The equations for the {\em dq}-model of an induction machine in the
synchronous reference frame are
\begin{eqnarray}
 \left[\begin{array}{c} v_{qs}^e\\v_{ds}^e\\v_{qr}^e\\v_{dr}^e  \end{array}\right]                                                                                                                                                                                                                                                                                                                                                                                                                                                                                                              
 &=& \left[ \begin{array}{cccc}
 r_s + x_s\frac{\rho}{\omega_b} & \frac{\omega_e}{\omega_b}x_s & x_m\frac{\rho}{\omega_b} & \frac{\omega_e}{\omega_b}x_m \\
 -\frac{\omega_e}{\omega_b}x_s & r_s + x_s\frac{\rho}{\omega_b} & -\frac{\omega_e}{\omega_b}x_m & x_m\frac{\rho}{\omega_b} \\
 x_m\frac{\rho}{\omega_b} & \frac{\omega_e -\omega_r}{\omega_b}x_m & r_r'+x_r'\frac{\rho}{\omega_b} & \frac{\omega_e - \omega_r}{\omega_b}x_r' \\
 -\frac{\omega_e -\omega_r}{\omega_b}x_m & x_m\frac{\rho}{\omega_b} & -\frac{\omega_e - \omega_r}{\omega_b}x_r' & r_r' + x_r'\frac{\rho}{\omega_b}
 \end{array} \right]
 \left[\begin{array}{c} i_{qs}^e\\i_{ds}^e\\i_{qr}^e\\i_{dr}^e\end{array} \right] \label{volteq}\\
 T_e&=&\frac{3}{2}\frac{P}{2}\frac{x_m}{\omega_b}\left(i_{qs}^ei_{dr}^e - i_{ds}^ei_{qr}^e\right) \label{torqueeq}\\
 T_e-T_l&=&\frac{2J\omega_b}{P}\frac{d}{dt}\left(\frac{\omega_r}{\omega_b}\right) \label{mecheq}.
\end{eqnarray}

\section{Multi-line Equations}

\LaTeX{} has a built-in equation array feature, however the
equation numbers must be on the same line as an equation.  For example:
\begin{eqnarray}
\Delta u + \lambda e^u &= 0&u\in \Omega,  \nonumber \\
u&=0&u\in\partial\Omega.
\end{eqnarray}

Alternatively, the number can be centered in the equation using the
following method.
%
% The equation-array feature in LaTeX is a bad idea.  For centered
% numbers you should set your own equations and arrays as follows:
%
\def\dd{\displaystyle}
\begin{equation}\label{gelfand}
\begin{array}{rl}
\dd \Delta u + \lambda e^u = 0, &
\dd u\in \Omega,\\[8pt] % add 8pt extra vertical space. 1 line=23pt
\dd u=0, & \dd u\in\partial\Omega.
\end{array}
\end{equation}
The previous equation had a label.  It may be referenced as
equation~(\ref{gelfand}).


\section{More Complicated Equations}
\section*{Rellich's identity}\label{rellich.section}
\setcounter{theorem}{0}
%
%

Standard developments of Pohozaev's identity used an identity by
Rellich~\cite{rellich:der40}, reproduced here.

\begin{lemma}[Rellich]
Given $L$ in divergence form and $a,d$ defined above, $u\in C^2
(\Omega )$, we have
\begin{equation}\label{rellich}
\int_{\Omega}(-Lu)\nabla u\cdot (x-\overline{x})\, dx
= (1-\frac{n}{2}) \int_{\Omega} a(\nabla u,\nabla u) \, dx
-
\frac{1}{2} \int_{\Omega}
d(\nabla u, \nabla u) \, dx
\end{equation}
$$
+
\frac{1}{2} \int_{\partial\Omega} a(\nabla u,\nabla u)(x-\overline{x})
\cdot \nu  \, dS
-
\int_{\partial\Omega}
a(\nabla u,\nu )\nabla u\cdot (x-\overline{x}) \, dS.
$$
\end{lemma}
{\bf Proof:}\\
There is no loss in generality to take $\overline{x} = 0$. First
rewrite $L$:
$$Lu = \frac{1}{2}\left[ \sum_{i}\sum_{j}
\frac{\partial}{\partial x_i}
\left( a_{ij} \frac{\partial u}{\partial x_j} \right) +
\sum_{i}\sum_{j}
\frac{\partial}{\partial x_i}
\left( a_{ij} \frac{\partial u}{\partial x_j} \right)
\right]$$
Switching the order of summation on the second term and relabeling
subscripts, $j \rightarrow i$ and $i \rightarrow j$, then using the fact
that $a_{ij}(x)$ is a symmetric matrix,
gives the symmetric form needed to derive Rellich's identity.
\begin{equation}
Lu = \frac{1}{2} \sum_{i,j}\left[
\frac{\partial}{\partial x_i}
\left( a_{ij} \frac{\partial u}{\partial x_j} \right) +
\frac{\partial}{\partial x_j}
\left( a_{ij} \frac{\partial u}{\partial x_i} \right)
\right].
\end{equation}

Multiplying $-Lu$ by $\frac{\partial u}{\partial x_k} x_k$ and integrating
over $\Omega$, yields
$$\int_{\Omega}(-Lu)\frac{\partial u}{\partial x_k} x_k \, dx=
-\frac{1}{2} \int_{\Omega}
\sum_{i,j}\left[
\frac{\partial}{\partial x_i}
\left( a_{ij} \frac{\partial u}{\partial x_j} \right) +
\frac{\partial}{\partial x_j}
\left( a_{ij} \frac{\partial u}{\partial x_i} \right)
\right]
\frac{\partial u}{\partial x_k} x_k \, dx$$
Integrating by parts (for integral theorems see~\cite[p. 20]
{zeidler:nfa88IIa})
gives
$$= \frac{1}{2} \int_{\Omega}
\sum_{i,j} a_{ij} \left[
\frac{\partial u}{\partial x_j}
\frac{\partial^2 u}{\partial x_k\partial x_i} +
\frac{\partial u}{\partial x_i}
\frac{\partial^2 u}{\partial x_k\partial x_j}
\right] x_k \, dx
$$
$$
+
\frac{1}{2} \int_{\Omega}
\sum_{i,j} a_{ij} \left[
\frac{\partial u}{\partial x_j} \delta_{ik} +
\frac{\partial u}{\partial x_i} \delta_{jk}
\right] \frac{\partial u}{\partial x_k} \, dx
$$
$$- \frac{1}{2} \int_{\partial\Omega}
\sum_{i,j} a_{ij} \left[
\frac{\partial u}{\partial x_j} \nu_{i} +
\frac{\partial u}{\partial x_i} \nu_{j}
\right] \frac{\partial u}{\partial x_k} x_k \, dx
$$
= $I_1 + I_2 + I_3$, where the unit normal vector is $\nu$.
One may rewrite $I_1$ as
$$I_1 = \frac{1}{2} \int_{\Omega}
\sum_{i,j} a_{ij} \frac{\partial}{\partial x_k}\left(
\frac{\partial u}{\partial x_i}
\frac{\partial u}{\partial x_j}
\right) x_k \, dx
$$
Integrating the first term by parts again yields
$$I_1 = -\frac{1}{2} \int_{\Omega}
\sum_{i,j} a_{ij} \left(
\frac{\partial u}{\partial x_i}
\frac{\partial u}{\partial x_j}
\right) \, dx
+
\frac{1}{2} \int_{\partial\Omega}
\sum_{i,j} a_{ij} \left(
\frac{\partial u}{\partial x_i}
\frac{\partial u}{\partial x_j}
\right) x_k \nu_k \, dS
$$
$$
-
\frac{1}{2} \int_{\Omega}
\sum_{i,j} \left(
\frac{\partial u}{\partial x_i}
\frac{\partial u}{\partial x_j}
\right) x_k \frac{\partial a_{ij}}{\partial x_k}\, dx.
$$
Summing over $k$ gives
$$\int_{\Omega}(-Lu)(\nabla u\cdot x)\, dx =
-\frac{n}{2} \int_{\Omega}
\sum_{i,j} a_{ij} \left(
\frac{\partial u}{\partial x_i}
\frac{\partial u}{\partial x_j}
\right) \, dx
$$
$$
+
\frac{1}{2} \int_{\partial\Omega}
\sum_{i,j} a_{ij} \left(
\frac{\partial u}{\partial x_i}
\frac{\partial u}{\partial x_j}
\right) (x\cdot \nu ) \, dS
-
\frac{1}{2} \int_{\Omega}
\sum_{i,j} \left(
\frac{\partial u}{\partial x_i}
\frac{\partial u}{\partial x_j}
\right) (x\cdot  \nabla a_{ij}) \, dx
$$
$$
+
\frac{1}{2} \int_{\Omega}
\sum_{i,j,k} a_{ij} \left[
\frac{\partial u}{\partial x_j}
\frac{\partial u}{\partial x_k} \delta_{ik} +
\frac{\partial u}{\partial x_i}
\frac{\partial u}{\partial x_k} \delta_{jk}
\right] \, dx
$$
$$- \frac{1}{2} \int_{\partial\Omega}
\sum_{i,j} a_{ij} \left[
\frac{\partial u}{\partial x_j} \nu_{i} +
\frac{\partial u}{\partial x_i} \nu_{j}
\right] (\nabla u\cdot x) \, dS.
$$
Combining the first and fourth term on the right-hand side
simplifies the expression
$$\int_{\Omega}(-Lu)(\nabla u\cdot x)\, dx
=
(1-\frac{n}{2}) \int_{\Omega}
\sum_{i,j} a_{ij} \left(
\frac{\partial u}{\partial x_i}
\frac{\partial u}{\partial x_j}
\right) \, dx
$$
$$
+
\frac{1}{2} \int_{\partial\Omega}
\sum_{i,j} a_{ij} \left(
\frac{\partial u}{\partial x_i}
\frac{\partial u}{\partial x_j}
\right) (x\cdot \nu ) \, dS
-
\frac{1}{2} \int_{\Omega}
\sum_{i,j} \left(
\frac{\partial u}{\partial x_i}
\frac{\partial u}{\partial x_j}
\right) (x\cdot  \nabla a_{ij}) \, dx
$$
$$
-
\frac{1}{2} \int_{\partial\Omega}
\sum_{i,j} a_{ij} \left[
\frac{\partial u}{\partial x_j} \nu_{i} +
\frac{\partial u}{\partial x_i} \nu_{j}
\right] (\nabla u\cdot x) \, dS.
$$
Using the notation defined above, the result follows.


           % Complex Equations from the UW Math Department
% acrobat.tex
%
% This file explains how to generate Adobe Acrobat files
%
% Eric Benedict, July 2000
%
% It is provided without warranty on an AS IS basis.

\newcommand{\pdf}{\mbox{\tt *.pdf}}

\chapter{Adobe Acrobat (\pdf ) Files}
The Adobe Acrobat file format has pretty much become the {\em de facto}
standard for document sharing.  As such, some faculty members and/or
departments may be requiring a final copy of the thesis in Acrobat format
(\pdf ).

There are several different methods of obtaining a \pdf\ file from a \LaTeX{}
thesis; however, they are all very site specific.  A couple of different
methods which have been found to work are mentioned as suggested ideas to try
as a starting point.  Depending on what is installed at your site/location some
of these may be applicable.

\section{Converting from {\tt *.ps} to \pdf}
One option to obtain the \pdf\ file would be to generate the thesis in a normal
manner and then use the Acrobat\ {\tt Distiller}\ to convert the postscript file into a
\pdf\ file.

If the\ {\tt Distiller}\ program is available and convenient to use, then this is quite easy to do.

Depending on the choice of document fonts, the results may not be satisfactory since
some of the fonts may end up as bit-mapped fonts and will display poorly at any resolution
other than what they were sampled on.  Also, since the\ {\tt Distiller}\ program is an expensive
program to obtain, it is not always available.

An alternative to the Adobe\ {\tt Distiller}\ program is the Alladin\ {\tt Ghostscript}\ program.  This is
available for free from

{\tt \verb|   http://www.cs.wisc.edu/~ghost/index.html|}

This program is available for most common operating systems as a compiled binary, but the source code
is available for other systems.  One drawback is that this conversion must be performed as a
command line invocation and isn't very user friendly.  This may be addressed in a future version of\
{\tt Ghostview}, the program which provides a nice user interface to\ {\tt Ghostscript}.

\section{Converting from {\tt *.dvi} to \pdf}
There are two programs available which will convert from {\tt *.dvi} to \pdf,\ {\tt dvipdf}\ and\
{\tt dvipdfm}.  The\ {\tt dvipdfm}\ program  will be discussed here.  In version 0.12, it can generate
bookmarks, thumbnails (with assistance from\ {\tt Ghostscript}), scaling and rotation, JPEG and
PNG bitmaps and font encoding and re-encoding (to support fonts which aren't fully supported by
the Acrobat suite).  When\ {\tt Ghostscript}\ is properly installed,\ {\tt dvipdfm}\ will automatically
convert any encapsulated PostScript figures into the required \pdf{} format.  This program behaved in a
similar manner to the {\tt dvips} program and was used to produce the \pdf{} format of this document.



\section{Generating \pdf{} Initially}
There are now some programs which are similar to \TeX{} but instead of producing a\ {\tt .dvi}\ output,
they produce \pdf as a native output.  One such program, {\sc pdf}\TeX{} / {\sc pdf}\LaTeX{},
is available from

{\tt \verb|   http://www.tug.org/applications/pdftex|}

Note that as of this date, July 2000, {\sc pdf}\TeX{} / {\sc pdf}\LaTeX{} while currently quite usable, it
is still in a beta version.  Look at the web site for more current information.

The present version was able to produce a \pdf{} file of this document without any required
changes, except for the Postscript figure inclusion  (Figure~\ref{vwcontr}).  To properly include
this figure, requires the conversion of the postscript figure into a \pdf{} figure.  The procedure
is described in the manual for {\sc pdf}\TeX{} / {\sc pdf}\LaTeX{}.  Note that the figure conversion will
require either\ {\tt Distiller}\ or\ {\tt Ghostscript}.
           % A discussion on generating PDF files.
\end{appendices}         % End of the Appendix Chapters.  ibid on \end{appendix}
%\include{vita}          % Optional Vita, use \begin{vita} vita text \end{vita}
\end{document}
\end{verbatim}
\end{quote}

\section{Prelude}
After the {\tt \verb|\begin{document}|} comes the preliminary information found in
theses.  In this manual, the information is kept in the file {\tt prelude.tex} (see
above).  These pages will need to be numbered with roman numerals, so use
\begin{quote}\tt\singlespace\begin{verbatim}
\clearpage\pagenumbering{roman}
\end{verbatim}\end{quote}

Next, comes your thesis or dissertation title, your name, date of graduation, department
and degree.
\begin{quote}\tt\singlespace\begin{verbatim}
\title{How to \LaTeX\ a Thesis}
\author{Eric R. Benedict}
\date{2000}
%   - The default degree is ``Doctor of Philosophy''
%     Degree can be changed using the command \degree{}
%\degree{New Degree}
%   - for a PhD dissertation (default), specify \dissertation
%\dissertation
%   - for a masters project report, specify \project
%\project
%   - for a preliminary report, specify \prelim
%\prelim
%   - for a masters thesis, specify \thesis
%\thesis
%   - The default department is ``Electrical Engineering''
%     The department can be changed using the command \department{}
%\department{New Department}
\end{verbatim}\end{quote}

If you specified the class option {\tt msthesis}, then the degree is changed to
{\em Master \break of Science} and the {\tt \verb|\thesis|} option is specified.  If you
want to have the masters margins with another document, then the {\tt \verb|\degree|}
and {\tt \verb|\dissertation|},  {\tt \verb|\project|}, {\em etc.\/} can be specified
as needed.

Once the
above are all defined, use  {\tt \verb|\maketitle|} to generate the title page.
\begin{quote}\tt\singlespace\begin{verbatim}
\maketitle
\end{verbatim}\end{quote}

If you wish to include a copyright page (see Section~\ref{copyright} for
information on registering the copyright.), then add the command
\begin{quote}\tt\singlespace\begin{verbatim}
\copyrightpage
\end{verbatim}\end{quote}
This will generate the proper copyright page and will use the name and date specified
in {\tt \verb|\author{}|} and {\tt \verb|\date{}|}.

Next are the dedications and acknowledgements:
\begin{quote}\tt\singlespace\begin{verbatim}
\begin{dedication}
To my pet rock, Skippy.
\end{dedication}

\begin{acknowledgments}
I thank the many people who have done lots of nice things for me.
\end{acknowledgments}
\end{verbatim}\end{quote}

You must tell \LaTeX{} to generate a table of contents, a list of tables and a list of
figures:
\begin{quote}\tt\singlespace\begin{verbatim}
\tableofcontents
\listoftables
\listoffigures
\end{verbatim}\end{quote}

If you wish to have a nomenclature, list of symbols or glossary it can go here.
\begin{quote}\tt\singlespace\begin{verbatim}
\begin{nomenclature}
%\begin{listofsymbols}
%\begin{glossary}
\begin{tabular}{ll}
$C_1$ & Constant 1\\
\ldots
\end{tabular}
%\end{glossary}
%\end{listofsymbols}
\end{nomenclature}
\end{verbatim}\end{quote}

If your abstract will be microfilmed by Bell and Howell (formerly UMI), then you
will need to generate an abstract of less than 350 words.  This abstract can be created
using the {\tt umiabstract} environment.  This environment requires that you define your
advisor and your advisor's title using {\tt \verb|\advisorname{}|} and
{\tt \verb|\advisortitle{}|}.
\begin{quote}\tt\singlespace\begin{verbatim}
\advisorname{Bucky J. Badger}
\advisortitle{Assistant Professor}
% ABSTRACT
\begin{umiabstract}
\noindent       % Don't indent first paragraph.
This explains the basics for using \LaTeX\ to typeset a
dissertation, thesis or project report for the University of
Wisconsin-Madison.

...

\end{umiabstract}
\end{verbatim}\end{quote}
This will place your name, title and required text at the top of the page and follow the
abstract text with your advisor's name at the bottom for your advisor's signature.  This
page is not numbered and would be submitted separately.

If you will have an abstract as part of your document, then the {\tt abstract} environment
should be used.
\begin{quote}\tt\singlespace\begin{verbatim}
\begin{abstract}
\noindent       % Don't indent first paragraph.
This explains the basics for using \LaTeX\ to typeset a
dissertation, thesis or project report for the University of
Wisconsin-Madison.

...

\end{abstract}
\end{verbatim}\end{quote}
This will generate a page number and it will be included in the Table
of Contents.  

If you will have both the UMI and regular abstracts like this document, then
you will probably want to write the abstract once and save it in a seperate
file such as {\tt abstract.tex}.  Then, you can use the same abstract for
both purposes.

\begin{quote}\begin{verbatim}
\begin{umiabstract}
  % abstract.tex
%
% This file has the abstract for the withesis style documentation
%
% Eric Benedict, Aug 2000
%
% It is provided without warranty on an AS IS basis.

\noindent       % Don't indent this paragraph.
This is not a thesis or dissertation and Master \TeX nician is not a
degree granted at the University of Wisconsin-Madison.

\vspace*{0.5em}
\noindent       % Don't indent this paragraph.
This explains the basics for using \LaTeX\ to typeset a dissertation,
thesis or masters project or preliminary report for the University of 
Wisconsin-Madison. Chapter
1 talks briefly about the thesis formatting at UW-Madison.  Chapter 2 gives
an overview of the ``essentials'' of \LaTeX{} and was written by Jon Warbrick.
Chapter 3 talks about figures and tables and what a {\em float} is.  Chapter 4
briefly introduces the {\sc Bib}\TeX{} program.  And finally, Chapter 5 discusses
some of the details for using the {\tt withesis} style file.  The material in
Chapters 2-4 basically are a review of fundamental \LaTeX{} usage and form
a reasonable basic tutorial.

\vspace*{0.5em}
\noindent       % Don't indent this paragraph.
The style discussed in this manual was originally written by Dave Kraynie and
edited by James Darrell McCauley as the {\tt puthesis} style for Purdue
University's theses.  This style was modified to form the {\tt withesis} style. This
manual is largely based on a similar manual by James Darrell McCauley and Scott Hucker.
Permission to use, copy, modify and distribute this software and its documentation
for any purpose and without fee is here by granted.  This software and its documentation
is provided ``as is'' without any express or implied warranty.

\end{umiabstract}

\begin{abstract}
  % abstract.tex
%
% This file has the abstract for the withesis style documentation
%
% Eric Benedict, Aug 2000
%
% It is provided without warranty on an AS IS basis.

\noindent       % Don't indent this paragraph.
This is not a thesis or dissertation and Master \TeX nician is not a
degree granted at the University of Wisconsin-Madison.

\vspace*{0.5em}
\noindent       % Don't indent this paragraph.
This explains the basics for using \LaTeX\ to typeset a dissertation,
thesis or masters project or preliminary report for the University of 
Wisconsin-Madison. Chapter
1 talks briefly about the thesis formatting at UW-Madison.  Chapter 2 gives
an overview of the ``essentials'' of \LaTeX{} and was written by Jon Warbrick.
Chapter 3 talks about figures and tables and what a {\em float} is.  Chapter 4
briefly introduces the {\sc Bib}\TeX{} program.  And finally, Chapter 5 discusses
some of the details for using the {\tt withesis} style file.  The material in
Chapters 2-4 basically are a review of fundamental \LaTeX{} usage and form
a reasonable basic tutorial.

\vspace*{0.5em}
\noindent       % Don't indent this paragraph.
The style discussed in this manual was originally written by Dave Kraynie and
edited by James Darrell McCauley as the {\tt puthesis} style for Purdue
University's theses.  This style was modified to form the {\tt withesis} style. This
manual is largely based on a similar manual by James Darrell McCauley and Scott Hucker.
Permission to use, copy, modify and distribute this software and its documentation
for any purpose and without fee is here by granted.  This software and its documentation
is provided ``as is'' without any express or implied warranty.

\end{abstract}
\end{verbatim}\end{quote}

Finally, the page numbers must be changed to arabic numbers to conclude the preliminary
portion of the document.
\begin{quote}\tt\singlespace\begin{verbatim}
\clearpage\pagenumbering{arabic}
\end{verbatim}\end{quote}

\section{The Body}
At the beginning of {\tt intro.tex} there is the following command:
\begin{quote}\tt\singlespace\begin{verbatim}
\chapter{Introducing the {\tt withesis} \LaTeX{} Style Guide}
\end{verbatim}\end{quote}
Following that is the text of the chapter.  The body of your thesis is separated by
sectioning commands like {\tt \verb|\chapter{}|}.  For more information on the sectioning
commands, see Section~\ref{ess:sectioning}.

Remember the basic rule of outlining you learned in grammar school:
\begin{quote}
You cannot have an `A' if you do not have a `B'
\end{quote}
Take care to have at least two {\tt \verb|\section|}s if you use the command; have
two {\tt \verb|\subsection|}s, {\em etc}.



\section{Additional Theorem Like Environments}
The {\tt withesis} style adds numerous additional theorem like environments.  These
environments were included to allow compatibility with the University of Wisconsin's
Math Department's style file.  These environments are
{\tt theorem}, {\tt assertion}, {\tt claim}, {\tt conjecture}, {\tt corollary},
{\tt definition}, {\tt example}, {\tt figger}, {\tt lemma}, {\tt prop} and {\tt remark}.

As an example, consider the following.
\begin{lemma}
Assuming that $\partial\Omega_2 = \emptyset$ and that $h(t) = 1$, we
have $$
\begin{array}{lr}
\Delta u = f, &  x\in\Omega ,\\[2pt]
u =  g_1, &  x\in\partial\Omega .
\end{array}
$$
\end{lemma}
which was produced with the following:
\begin{quote}\tt\singlespace\begin{verbatim}
\begin{lemma}
Assuming that $\partial\Omega_2 = \emptyset$ and that $h(t) = 1$, we
have $$
\begin{array}{lr}
\Delta u = f, &  x\in\Omega ,\\[2pt]
 u =  g_1, &  x\in\partial\Omega .
\end{array}
$$
\end{lemma}
\end{verbatim}\end{quote}

\section{Bibliography or References}
As a final note, the default title for the references chapter is ``LIST OF REFERENCES.''
Since some people may prefer ``BIBLIOGRAPHY'', the command
\break{\tt \verb|\altbibtitle|}
has been added to change the chapter title.

\section{Appendices}
There are two commands which are available to suppress the writing of the auxiliary information
(to the {\tt .lot} and {\tt .lof} files).  They are:
\begin{quote}\tt\singlespace\begin{verbatim}
\noappendixtables                % Don't have appendix tables
\noappendixfigures               % Don't have appendix figures
\end{verbatim}\end{quote}
These commands should be in the preamble.  See Section~\ref{usage:noapp}.

There are two environments for doing the appendix chapter: {\tt appendix} and
\break {\tt appendices}.  If you have only one chapter in the appendix, use the {\tt appendix}
environment.  If you have more than one chapter, like this manual, use the
{\tt appendices} environment.
\begin{quote}\tt\singlespace\footnotesize\begin{verbatim}
\begin{appendices}  % Start of the Appendix Chapters.  If there is only
                    % one Appendix Chapter, then use \begin{appendix}
% code.tex
% this file is part of the example UW-Madison Thesis document
% It demonstrates one method for incorporating program listings
% into a document.

\chapter{Matlab Code} \label{matlab}
This is an example of a Matlab m-file.
\verbatimfile{derivs.m}
      % Including computer code listings
\chapter{Bib\TeX\ Entries}
\label{bibrefs}
The following shows the fields required in all types of Bib\TeX\ entries.
Fields with {\tt OPT} prefixed are optional (the three letters {\tt OPT} should 
not be used).  If an optional field is not used, then the entire field can be deleted.

{\tt
\singlespace
\begin{verbatim}

@Unpublished{,                            @Manual{,
  author =      "",                         title =           "",
  title =       "",                         OPTauthor =       "",
  note =        "",                         OPTorganization = "",
  OPTyear =     "",                         OPTaddress =      "",
  OPTmonth =    ""                          OPTedition =      "",
}                                           OPTyear =         "",
                                            OPTmonth =        "",
@TechReport{,                               OPTnote =         "" 
  author =      "",                       }
  title =       "",
  institution = "",                       @InProceedings{,
  year =        "",                         author =          "",
  OPTtype =     "",                         title =           "",
  OPTnumber =   "",                         booktitle =       "",
  OPTaddress =  "",                         year =            "",
  OPTmonth =    "",                         OPTeditor =       "",
  OPTnote =     ""                          OPTpages =        "",
}                                           OPTorganization = "",
                                            OPTpublisher =    "",
@Proceedings{,                              OPTaddress =      "",
  title =           "",                     OPTmonth =        "",
  year =            "",                     OPTnote =         "" 
  OPTeditor =       "",                   }
  OPTpublisher =    "",
  OPTorganization = "",
  OPTaddress =      "",
  OPTmonth =        "",
  OPTnote =         "" 
}



@PhDThesis{,                              @InCollection{,
  author =      "",                         author =          "",
  title =       "",                         title =           "",
  school =      "",                         booktitle =       "",
  year =        "",                         publisher =       "",
  OPTaddress =  "",                         year =            "",
  OPTmonth =    "",                         OPTeditor =       "",
  OPTnote =     ""                          OPTchapter =      "",
}                                           OPTpages =        "",
                                            OPTaddress =      "",
                                            OPTmonth =        "",
                                            OPTnote =         ""
                                          }

 
@Misc{,                                   @InCollection{,
  OPTauthor =       "",                     author =          "",
  OPTtitle =        "",                     title =           "",
  OPThowpublished = "",                     chapter =         "",
  OPTyear =         "",                     publisher =       "",
  OPTmonth =        "",                     year =            "",
  OPTnote =         ""                      OPTeditor =       "",
}                                           OPTpages =        "",
}                                           OPTvolume =       "",
                                            OPTseries =       "",
                                            OPTaddress =      "",
                                            OPTedition =      "",
                                            OPTmonth =        "",
                                            OPTnote =         ""
                                          }

@MastersThesis{,                          @Article{,
  author =      "",                         author =          "",
  title =       "",                         title =           "",
  school =      "",                         journal =         "",
  year =        "",                         year =            "",
  OPTaddress =  "",                         OPTvolume =       "",
  OPTmonth =    "",                         OPTnumber =       "",
  OPTnote =     ""                          OPTpages =        "",
}                                           OPTmonth =        "",
                                            OPTnote =         ""
                                           }\end{verbatim} }
    % a BibTeX reference
\chapter{Mathematics Examples}
This appendix provides an example of \LaTeX's typesetting
capabilities.  Most of text was obtained from the University of
Wisconsin-Madison Math Department's example thesis file.

\section{Matrices}
The equations for the {\em dq}-model of an induction machine in the
synchronous reference frame are
\begin{eqnarray}
 \left[\begin{array}{c} v_{qs}^e\\v_{ds}^e\\v_{qr}^e\\v_{dr}^e  \end{array}\right]                                                                                                                                                                                                                                                                                                                                                                                                                                                                                                              
 &=& \left[ \begin{array}{cccc}
 r_s + x_s\frac{\rho}{\omega_b} & \frac{\omega_e}{\omega_b}x_s & x_m\frac{\rho}{\omega_b} & \frac{\omega_e}{\omega_b}x_m \\
 -\frac{\omega_e}{\omega_b}x_s & r_s + x_s\frac{\rho}{\omega_b} & -\frac{\omega_e}{\omega_b}x_m & x_m\frac{\rho}{\omega_b} \\
 x_m\frac{\rho}{\omega_b} & \frac{\omega_e -\omega_r}{\omega_b}x_m & r_r'+x_r'\frac{\rho}{\omega_b} & \frac{\omega_e - \omega_r}{\omega_b}x_r' \\
 -\frac{\omega_e -\omega_r}{\omega_b}x_m & x_m\frac{\rho}{\omega_b} & -\frac{\omega_e - \omega_r}{\omega_b}x_r' & r_r' + x_r'\frac{\rho}{\omega_b}
 \end{array} \right]
 \left[\begin{array}{c} i_{qs}^e\\i_{ds}^e\\i_{qr}^e\\i_{dr}^e\end{array} \right] \label{volteq}\\
 T_e&=&\frac{3}{2}\frac{P}{2}\frac{x_m}{\omega_b}\left(i_{qs}^ei_{dr}^e - i_{ds}^ei_{qr}^e\right) \label{torqueeq}\\
 T_e-T_l&=&\frac{2J\omega_b}{P}\frac{d}{dt}\left(\frac{\omega_r}{\omega_b}\right) \label{mecheq}.
\end{eqnarray}

\section{Multi-line Equations}

\LaTeX{} has a built-in equation array feature, however the
equation numbers must be on the same line as an equation.  For example:
\begin{eqnarray}
\Delta u + \lambda e^u &= 0&u\in \Omega,  \nonumber \\
u&=0&u\in\partial\Omega.
\end{eqnarray}

Alternatively, the number can be centered in the equation using the
following method.
%
% The equation-array feature in LaTeX is a bad idea.  For centered
% numbers you should set your own equations and arrays as follows:
%
\def\dd{\displaystyle}
\begin{equation}\label{gelfand}
\begin{array}{rl}
\dd \Delta u + \lambda e^u = 0, &
\dd u\in \Omega,\\[8pt] % add 8pt extra vertical space. 1 line=23pt
\dd u=0, & \dd u\in\partial\Omega.
\end{array}
\end{equation}
The previous equation had a label.  It may be referenced as
equation~(\ref{gelfand}).


\section{More Complicated Equations}
\section*{Rellich's identity}\label{rellich.section}
\setcounter{theorem}{0}
%
%

Standard developments of Pohozaev's identity used an identity by
Rellich~\cite{rellich:der40}, reproduced here.

\begin{lemma}[Rellich]
Given $L$ in divergence form and $a,d$ defined above, $u\in C^2
(\Omega )$, we have
\begin{equation}\label{rellich}
\int_{\Omega}(-Lu)\nabla u\cdot (x-\overline{x})\, dx
= (1-\frac{n}{2}) \int_{\Omega} a(\nabla u,\nabla u) \, dx
-
\frac{1}{2} \int_{\Omega}
d(\nabla u, \nabla u) \, dx
\end{equation}
$$
+
\frac{1}{2} \int_{\partial\Omega} a(\nabla u,\nabla u)(x-\overline{x})
\cdot \nu  \, dS
-
\int_{\partial\Omega}
a(\nabla u,\nu )\nabla u\cdot (x-\overline{x}) \, dS.
$$
\end{lemma}
{\bf Proof:}\\
There is no loss in generality to take $\overline{x} = 0$. First
rewrite $L$:
$$Lu = \frac{1}{2}\left[ \sum_{i}\sum_{j}
\frac{\partial}{\partial x_i}
\left( a_{ij} \frac{\partial u}{\partial x_j} \right) +
\sum_{i}\sum_{j}
\frac{\partial}{\partial x_i}
\left( a_{ij} \frac{\partial u}{\partial x_j} \right)
\right]$$
Switching the order of summation on the second term and relabeling
subscripts, $j \rightarrow i$ and $i \rightarrow j$, then using the fact
that $a_{ij}(x)$ is a symmetric matrix,
gives the symmetric form needed to derive Rellich's identity.
\begin{equation}
Lu = \frac{1}{2} \sum_{i,j}\left[
\frac{\partial}{\partial x_i}
\left( a_{ij} \frac{\partial u}{\partial x_j} \right) +
\frac{\partial}{\partial x_j}
\left( a_{ij} \frac{\partial u}{\partial x_i} \right)
\right].
\end{equation}

Multiplying $-Lu$ by $\frac{\partial u}{\partial x_k} x_k$ and integrating
over $\Omega$, yields
$$\int_{\Omega}(-Lu)\frac{\partial u}{\partial x_k} x_k \, dx=
-\frac{1}{2} \int_{\Omega}
\sum_{i,j}\left[
\frac{\partial}{\partial x_i}
\left( a_{ij} \frac{\partial u}{\partial x_j} \right) +
\frac{\partial}{\partial x_j}
\left( a_{ij} \frac{\partial u}{\partial x_i} \right)
\right]
\frac{\partial u}{\partial x_k} x_k \, dx$$
Integrating by parts (for integral theorems see~\cite[p. 20]
{zeidler:nfa88IIa})
gives
$$= \frac{1}{2} \int_{\Omega}
\sum_{i,j} a_{ij} \left[
\frac{\partial u}{\partial x_j}
\frac{\partial^2 u}{\partial x_k\partial x_i} +
\frac{\partial u}{\partial x_i}
\frac{\partial^2 u}{\partial x_k\partial x_j}
\right] x_k \, dx
$$
$$
+
\frac{1}{2} \int_{\Omega}
\sum_{i,j} a_{ij} \left[
\frac{\partial u}{\partial x_j} \delta_{ik} +
\frac{\partial u}{\partial x_i} \delta_{jk}
\right] \frac{\partial u}{\partial x_k} \, dx
$$
$$- \frac{1}{2} \int_{\partial\Omega}
\sum_{i,j} a_{ij} \left[
\frac{\partial u}{\partial x_j} \nu_{i} +
\frac{\partial u}{\partial x_i} \nu_{j}
\right] \frac{\partial u}{\partial x_k} x_k \, dx
$$
= $I_1 + I_2 + I_3$, where the unit normal vector is $\nu$.
One may rewrite $I_1$ as
$$I_1 = \frac{1}{2} \int_{\Omega}
\sum_{i,j} a_{ij} \frac{\partial}{\partial x_k}\left(
\frac{\partial u}{\partial x_i}
\frac{\partial u}{\partial x_j}
\right) x_k \, dx
$$
Integrating the first term by parts again yields
$$I_1 = -\frac{1}{2} \int_{\Omega}
\sum_{i,j} a_{ij} \left(
\frac{\partial u}{\partial x_i}
\frac{\partial u}{\partial x_j}
\right) \, dx
+
\frac{1}{2} \int_{\partial\Omega}
\sum_{i,j} a_{ij} \left(
\frac{\partial u}{\partial x_i}
\frac{\partial u}{\partial x_j}
\right) x_k \nu_k \, dS
$$
$$
-
\frac{1}{2} \int_{\Omega}
\sum_{i,j} \left(
\frac{\partial u}{\partial x_i}
\frac{\partial u}{\partial x_j}
\right) x_k \frac{\partial a_{ij}}{\partial x_k}\, dx.
$$
Summing over $k$ gives
$$\int_{\Omega}(-Lu)(\nabla u\cdot x)\, dx =
-\frac{n}{2} \int_{\Omega}
\sum_{i,j} a_{ij} \left(
\frac{\partial u}{\partial x_i}
\frac{\partial u}{\partial x_j}
\right) \, dx
$$
$$
+
\frac{1}{2} \int_{\partial\Omega}
\sum_{i,j} a_{ij} \left(
\frac{\partial u}{\partial x_i}
\frac{\partial u}{\partial x_j}
\right) (x\cdot \nu ) \, dS
-
\frac{1}{2} \int_{\Omega}
\sum_{i,j} \left(
\frac{\partial u}{\partial x_i}
\frac{\partial u}{\partial x_j}
\right) (x\cdot  \nabla a_{ij}) \, dx
$$
$$
+
\frac{1}{2} \int_{\Omega}
\sum_{i,j,k} a_{ij} \left[
\frac{\partial u}{\partial x_j}
\frac{\partial u}{\partial x_k} \delta_{ik} +
\frac{\partial u}{\partial x_i}
\frac{\partial u}{\partial x_k} \delta_{jk}
\right] \, dx
$$
$$- \frac{1}{2} \int_{\partial\Omega}
\sum_{i,j} a_{ij} \left[
\frac{\partial u}{\partial x_j} \nu_{i} +
\frac{\partial u}{\partial x_i} \nu_{j}
\right] (\nabla u\cdot x) \, dS.
$$
Combining the first and fourth term on the right-hand side
simplifies the expression
$$\int_{\Omega}(-Lu)(\nabla u\cdot x)\, dx
=
(1-\frac{n}{2}) \int_{\Omega}
\sum_{i,j} a_{ij} \left(
\frac{\partial u}{\partial x_i}
\frac{\partial u}{\partial x_j}
\right) \, dx
$$
$$
+
\frac{1}{2} \int_{\partial\Omega}
\sum_{i,j} a_{ij} \left(
\frac{\partial u}{\partial x_i}
\frac{\partial u}{\partial x_j}
\right) (x\cdot \nu ) \, dS
-
\frac{1}{2} \int_{\Omega}
\sum_{i,j} \left(
\frac{\partial u}{\partial x_i}
\frac{\partial u}{\partial x_j}
\right) (x\cdot  \nabla a_{ij}) \, dx
$$
$$
-
\frac{1}{2} \int_{\partial\Omega}
\sum_{i,j} a_{ij} \left[
\frac{\partial u}{\partial x_j} \nu_{i} +
\frac{\partial u}{\partial x_i} \nu_{j}
\right] (\nabla u\cdot x) \, dS.
$$
Using the notation defined above, the result follows.


      % Complex Equations from the UW Math Department

\end{appendices}    % End of the Appendix Chapters. ibid on \end{appendix}
\end{verbatim}\end{quote}
The difference between these two environments is the way that the chapter header is
created and how this is listed in the table of contents.
          % Chapter 5 Strongly based on similar by J.D. McCauley
\bibliography{refs}      % Make the bibliography
\begin{appendices}       % Start of the Appendix Chapters.  If there is only
                         % one Appendix Chapter, then use \begin{appendix}
% code.tex
% this file is part of the example UW-Madison Thesis document
% It demonstrates one method for incorporating program listings
% into a document.

\chapter{Matlab Code} \label{matlab}
This is an example of a Matlab m-file.
\verbatimfile{derivs.m}
         % Including computer code listings
\chapter{Bib\TeX\ Entries}
\label{bibrefs}
The following shows the fields required in all types of Bib\TeX\ entries.
Fields with {\tt OPT} prefixed are optional (the three letters {\tt OPT} should 
not be used).  If an optional field is not used, then the entire field can be deleted.

{\tt
\singlespace
\begin{verbatim}

@Unpublished{,                            @Manual{,
  author =      "",                         title =           "",
  title =       "",                         OPTauthor =       "",
  note =        "",                         OPTorganization = "",
  OPTyear =     "",                         OPTaddress =      "",
  OPTmonth =    ""                          OPTedition =      "",
}                                           OPTyear =         "",
                                            OPTmonth =        "",
@TechReport{,                               OPTnote =         "" 
  author =      "",                       }
  title =       "",
  institution = "",                       @InProceedings{,
  year =        "",                         author =          "",
  OPTtype =     "",                         title =           "",
  OPTnumber =   "",                         booktitle =       "",
  OPTaddress =  "",                         year =            "",
  OPTmonth =    "",                         OPTeditor =       "",
  OPTnote =     ""                          OPTpages =        "",
}                                           OPTorganization = "",
                                            OPTpublisher =    "",
@Proceedings{,                              OPTaddress =      "",
  title =           "",                     OPTmonth =        "",
  year =            "",                     OPTnote =         "" 
  OPTeditor =       "",                   }
  OPTpublisher =    "",
  OPTorganization = "",
  OPTaddress =      "",
  OPTmonth =        "",
  OPTnote =         "" 
}



@PhDThesis{,                              @InCollection{,
  author =      "",                         author =          "",
  title =       "",                         title =           "",
  school =      "",                         booktitle =       "",
  year =        "",                         publisher =       "",
  OPTaddress =  "",                         year =            "",
  OPTmonth =    "",                         OPTeditor =       "",
  OPTnote =     ""                          OPTchapter =      "",
}                                           OPTpages =        "",
                                            OPTaddress =      "",
                                            OPTmonth =        "",
                                            OPTnote =         ""
                                          }

 
@Misc{,                                   @InCollection{,
  OPTauthor =       "",                     author =          "",
  OPTtitle =        "",                     title =           "",
  OPThowpublished = "",                     chapter =         "",
  OPTyear =         "",                     publisher =       "",
  OPTmonth =        "",                     year =            "",
  OPTnote =         ""                      OPTeditor =       "",
}                                           OPTpages =        "",
}                                           OPTvolume =       "",
                                            OPTseries =       "",
                                            OPTaddress =      "",
                                            OPTedition =      "",
                                            OPTmonth =        "",
                                            OPTnote =         ""
                                          }

@MastersThesis{,                          @Article{,
  author =      "",                         author =          "",
  title =       "",                         title =           "",
  school =      "",                         journal =         "",
  year =        "",                         year =            "",
  OPTaddress =  "",                         OPTvolume =       "",
  OPTmonth =    "",                         OPTnumber =       "",
  OPTnote =     ""                          OPTpages =        "",
}                                           OPTmonth =        "",
                                            OPTnote =         ""
                                           }\end{verbatim} }
         % a BibTeX reference
\chapter{Mathematics Examples}
This appendix provides an example of \LaTeX's typesetting
capabilities.  Most of text was obtained from the University of
Wisconsin-Madison Math Department's example thesis file.

\section{Matrices}
The equations for the {\em dq}-model of an induction machine in the
synchronous reference frame are
\begin{eqnarray}
 \left[\begin{array}{c} v_{qs}^e\\v_{ds}^e\\v_{qr}^e\\v_{dr}^e  \end{array}\right]                                                                                                                                                                                                                                                                                                                                                                                                                                                                                                              
 &=& \left[ \begin{array}{cccc}
 r_s + x_s\frac{\rho}{\omega_b} & \frac{\omega_e}{\omega_b}x_s & x_m\frac{\rho}{\omega_b} & \frac{\omega_e}{\omega_b}x_m \\
 -\frac{\omega_e}{\omega_b}x_s & r_s + x_s\frac{\rho}{\omega_b} & -\frac{\omega_e}{\omega_b}x_m & x_m\frac{\rho}{\omega_b} \\
 x_m\frac{\rho}{\omega_b} & \frac{\omega_e -\omega_r}{\omega_b}x_m & r_r'+x_r'\frac{\rho}{\omega_b} & \frac{\omega_e - \omega_r}{\omega_b}x_r' \\
 -\frac{\omega_e -\omega_r}{\omega_b}x_m & x_m\frac{\rho}{\omega_b} & -\frac{\omega_e - \omega_r}{\omega_b}x_r' & r_r' + x_r'\frac{\rho}{\omega_b}
 \end{array} \right]
 \left[\begin{array}{c} i_{qs}^e\\i_{ds}^e\\i_{qr}^e\\i_{dr}^e\end{array} \right] \label{volteq}\\
 T_e&=&\frac{3}{2}\frac{P}{2}\frac{x_m}{\omega_b}\left(i_{qs}^ei_{dr}^e - i_{ds}^ei_{qr}^e\right) \label{torqueeq}\\
 T_e-T_l&=&\frac{2J\omega_b}{P}\frac{d}{dt}\left(\frac{\omega_r}{\omega_b}\right) \label{mecheq}.
\end{eqnarray}

\section{Multi-line Equations}

\LaTeX{} has a built-in equation array feature, however the
equation numbers must be on the same line as an equation.  For example:
\begin{eqnarray}
\Delta u + \lambda e^u &= 0&u\in \Omega,  \nonumber \\
u&=0&u\in\partial\Omega.
\end{eqnarray}

Alternatively, the number can be centered in the equation using the
following method.
%
% The equation-array feature in LaTeX is a bad idea.  For centered
% numbers you should set your own equations and arrays as follows:
%
\def\dd{\displaystyle}
\begin{equation}\label{gelfand}
\begin{array}{rl}
\dd \Delta u + \lambda e^u = 0, &
\dd u\in \Omega,\\[8pt] % add 8pt extra vertical space. 1 line=23pt
\dd u=0, & \dd u\in\partial\Omega.
\end{array}
\end{equation}
The previous equation had a label.  It may be referenced as
equation~(\ref{gelfand}).


\section{More Complicated Equations}
\section*{Rellich's identity}\label{rellich.section}
\setcounter{theorem}{0}
%
%

Standard developments of Pohozaev's identity used an identity by
Rellich~\cite{rellich:der40}, reproduced here.

\begin{lemma}[Rellich]
Given $L$ in divergence form and $a,d$ defined above, $u\in C^2
(\Omega )$, we have
\begin{equation}\label{rellich}
\int_{\Omega}(-Lu)\nabla u\cdot (x-\overline{x})\, dx
= (1-\frac{n}{2}) \int_{\Omega} a(\nabla u,\nabla u) \, dx
-
\frac{1}{2} \int_{\Omega}
d(\nabla u, \nabla u) \, dx
\end{equation}
$$
+
\frac{1}{2} \int_{\partial\Omega} a(\nabla u,\nabla u)(x-\overline{x})
\cdot \nu  \, dS
-
\int_{\partial\Omega}
a(\nabla u,\nu )\nabla u\cdot (x-\overline{x}) \, dS.
$$
\end{lemma}
{\bf Proof:}\\
There is no loss in generality to take $\overline{x} = 0$. First
rewrite $L$:
$$Lu = \frac{1}{2}\left[ \sum_{i}\sum_{j}
\frac{\partial}{\partial x_i}
\left( a_{ij} \frac{\partial u}{\partial x_j} \right) +
\sum_{i}\sum_{j}
\frac{\partial}{\partial x_i}
\left( a_{ij} \frac{\partial u}{\partial x_j} \right)
\right]$$
Switching the order of summation on the second term and relabeling
subscripts, $j \rightarrow i$ and $i \rightarrow j$, then using the fact
that $a_{ij}(x)$ is a symmetric matrix,
gives the symmetric form needed to derive Rellich's identity.
\begin{equation}
Lu = \frac{1}{2} \sum_{i,j}\left[
\frac{\partial}{\partial x_i}
\left( a_{ij} \frac{\partial u}{\partial x_j} \right) +
\frac{\partial}{\partial x_j}
\left( a_{ij} \frac{\partial u}{\partial x_i} \right)
\right].
\end{equation}

Multiplying $-Lu$ by $\frac{\partial u}{\partial x_k} x_k$ and integrating
over $\Omega$, yields
$$\int_{\Omega}(-Lu)\frac{\partial u}{\partial x_k} x_k \, dx=
-\frac{1}{2} \int_{\Omega}
\sum_{i,j}\left[
\frac{\partial}{\partial x_i}
\left( a_{ij} \frac{\partial u}{\partial x_j} \right) +
\frac{\partial}{\partial x_j}
\left( a_{ij} \frac{\partial u}{\partial x_i} \right)
\right]
\frac{\partial u}{\partial x_k} x_k \, dx$$
Integrating by parts (for integral theorems see~\cite[p. 20]
{zeidler:nfa88IIa})
gives
$$= \frac{1}{2} \int_{\Omega}
\sum_{i,j} a_{ij} \left[
\frac{\partial u}{\partial x_j}
\frac{\partial^2 u}{\partial x_k\partial x_i} +
\frac{\partial u}{\partial x_i}
\frac{\partial^2 u}{\partial x_k\partial x_j}
\right] x_k \, dx
$$
$$
+
\frac{1}{2} \int_{\Omega}
\sum_{i,j} a_{ij} \left[
\frac{\partial u}{\partial x_j} \delta_{ik} +
\frac{\partial u}{\partial x_i} \delta_{jk}
\right] \frac{\partial u}{\partial x_k} \, dx
$$
$$- \frac{1}{2} \int_{\partial\Omega}
\sum_{i,j} a_{ij} \left[
\frac{\partial u}{\partial x_j} \nu_{i} +
\frac{\partial u}{\partial x_i} \nu_{j}
\right] \frac{\partial u}{\partial x_k} x_k \, dx
$$
= $I_1 + I_2 + I_3$, where the unit normal vector is $\nu$.
One may rewrite $I_1$ as
$$I_1 = \frac{1}{2} \int_{\Omega}
\sum_{i,j} a_{ij} \frac{\partial}{\partial x_k}\left(
\frac{\partial u}{\partial x_i}
\frac{\partial u}{\partial x_j}
\right) x_k \, dx
$$
Integrating the first term by parts again yields
$$I_1 = -\frac{1}{2} \int_{\Omega}
\sum_{i,j} a_{ij} \left(
\frac{\partial u}{\partial x_i}
\frac{\partial u}{\partial x_j}
\right) \, dx
+
\frac{1}{2} \int_{\partial\Omega}
\sum_{i,j} a_{ij} \left(
\frac{\partial u}{\partial x_i}
\frac{\partial u}{\partial x_j}
\right) x_k \nu_k \, dS
$$
$$
-
\frac{1}{2} \int_{\Omega}
\sum_{i,j} \left(
\frac{\partial u}{\partial x_i}
\frac{\partial u}{\partial x_j}
\right) x_k \frac{\partial a_{ij}}{\partial x_k}\, dx.
$$
Summing over $k$ gives
$$\int_{\Omega}(-Lu)(\nabla u\cdot x)\, dx =
-\frac{n}{2} \int_{\Omega}
\sum_{i,j} a_{ij} \left(
\frac{\partial u}{\partial x_i}
\frac{\partial u}{\partial x_j}
\right) \, dx
$$
$$
+
\frac{1}{2} \int_{\partial\Omega}
\sum_{i,j} a_{ij} \left(
\frac{\partial u}{\partial x_i}
\frac{\partial u}{\partial x_j}
\right) (x\cdot \nu ) \, dS
-
\frac{1}{2} \int_{\Omega}
\sum_{i,j} \left(
\frac{\partial u}{\partial x_i}
\frac{\partial u}{\partial x_j}
\right) (x\cdot  \nabla a_{ij}) \, dx
$$
$$
+
\frac{1}{2} \int_{\Omega}
\sum_{i,j,k} a_{ij} \left[
\frac{\partial u}{\partial x_j}
\frac{\partial u}{\partial x_k} \delta_{ik} +
\frac{\partial u}{\partial x_i}
\frac{\partial u}{\partial x_k} \delta_{jk}
\right] \, dx
$$
$$- \frac{1}{2} \int_{\partial\Omega}
\sum_{i,j} a_{ij} \left[
\frac{\partial u}{\partial x_j} \nu_{i} +
\frac{\partial u}{\partial x_i} \nu_{j}
\right] (\nabla u\cdot x) \, dS.
$$
Combining the first and fourth term on the right-hand side
simplifies the expression
$$\int_{\Omega}(-Lu)(\nabla u\cdot x)\, dx
=
(1-\frac{n}{2}) \int_{\Omega}
\sum_{i,j} a_{ij} \left(
\frac{\partial u}{\partial x_i}
\frac{\partial u}{\partial x_j}
\right) \, dx
$$
$$
+
\frac{1}{2} \int_{\partial\Omega}
\sum_{i,j} a_{ij} \left(
\frac{\partial u}{\partial x_i}
\frac{\partial u}{\partial x_j}
\right) (x\cdot \nu ) \, dS
-
\frac{1}{2} \int_{\Omega}
\sum_{i,j} \left(
\frac{\partial u}{\partial x_i}
\frac{\partial u}{\partial x_j}
\right) (x\cdot  \nabla a_{ij}) \, dx
$$
$$
-
\frac{1}{2} \int_{\partial\Omega}
\sum_{i,j} a_{ij} \left[
\frac{\partial u}{\partial x_j} \nu_{i} +
\frac{\partial u}{\partial x_i} \nu_{j}
\right] (\nabla u\cdot x) \, dS.
$$
Using the notation defined above, the result follows.


           % Complex Equations from the UW Math Department
% acrobat.tex
%
% This file explains how to generate Adobe Acrobat files
%
% Eric Benedict, July 2000
%
% It is provided without warranty on an AS IS basis.

\newcommand{\pdf}{\mbox{\tt *.pdf}}

\chapter{Adobe Acrobat (\pdf ) Files}
The Adobe Acrobat file format has pretty much become the {\em de facto}
standard for document sharing.  As such, some faculty members and/or
departments may be requiring a final copy of the thesis in Acrobat format
(\pdf ).

There are several different methods of obtaining a \pdf\ file from a \LaTeX{}
thesis; however, they are all very site specific.  A couple of different
methods which have been found to work are mentioned as suggested ideas to try
as a starting point.  Depending on what is installed at your site/location some
of these may be applicable.

\section{Converting from {\tt *.ps} to \pdf}
One option to obtain the \pdf\ file would be to generate the thesis in a normal
manner and then use the Acrobat\ {\tt Distiller}\ to convert the postscript file into a
\pdf\ file.

If the\ {\tt Distiller}\ program is available and convenient to use, then this is quite easy to do.

Depending on the choice of document fonts, the results may not be satisfactory since
some of the fonts may end up as bit-mapped fonts and will display poorly at any resolution
other than what they were sampled on.  Also, since the\ {\tt Distiller}\ program is an expensive
program to obtain, it is not always available.

An alternative to the Adobe\ {\tt Distiller}\ program is the Alladin\ {\tt Ghostscript}\ program.  This is
available for free from

{\tt \verb|   http://www.cs.wisc.edu/~ghost/index.html|}

This program is available for most common operating systems as a compiled binary, but the source code
is available for other systems.  One drawback is that this conversion must be performed as a
command line invocation and isn't very user friendly.  This may be addressed in a future version of\
{\tt Ghostview}, the program which provides a nice user interface to\ {\tt Ghostscript}.

\section{Converting from {\tt *.dvi} to \pdf}
There are two programs available which will convert from {\tt *.dvi} to \pdf,\ {\tt dvipdf}\ and\
{\tt dvipdfm}.  The\ {\tt dvipdfm}\ program  will be discussed here.  In version 0.12, it can generate
bookmarks, thumbnails (with assistance from\ {\tt Ghostscript}), scaling and rotation, JPEG and
PNG bitmaps and font encoding and re-encoding (to support fonts which aren't fully supported by
the Acrobat suite).  When\ {\tt Ghostscript}\ is properly installed,\ {\tt dvipdfm}\ will automatically
convert any encapsulated PostScript figures into the required \pdf{} format.  This program behaved in a
similar manner to the {\tt dvips} program and was used to produce the \pdf{} format of this document.



\section{Generating \pdf{} Initially}
There are now some programs which are similar to \TeX{} but instead of producing a\ {\tt .dvi}\ output,
they produce \pdf as a native output.  One such program, {\sc pdf}\TeX{} / {\sc pdf}\LaTeX{},
is available from

{\tt \verb|   http://www.tug.org/applications/pdftex|}

Note that as of this date, July 2000, {\sc pdf}\TeX{} / {\sc pdf}\LaTeX{} while currently quite usable, it
is still in a beta version.  Look at the web site for more current information.

The present version was able to produce a \pdf{} file of this document without any required
changes, except for the Postscript figure inclusion  (Figure~\ref{vwcontr}).  To properly include
this figure, requires the conversion of the postscript figure into a \pdf{} figure.  The procedure
is described in the manual for {\sc pdf}\TeX{} / {\sc pdf}\LaTeX{}.  Note that the figure conversion will
require either\ {\tt Distiller}\ or\ {\tt Ghostscript}.
           % A discussion on generating PDF files.
\end{appendices}         % End of the Appendix Chapters.  ibid on \end{appendix}
%\include{vita}          % Optional Vita, use \begin{vita} vita text \end{vita}
\end{document}
\end{verbatim}
\end{quote}

\section{Prelude}
After the {\tt \verb|\begin{document}|} comes the preliminary information found in
theses.  In this manual, the information is kept in the file {\tt prelude.tex} (see
above).  These pages will need to be numbered with roman numerals, so use
\begin{quote}\tt\singlespace\begin{verbatim}
\clearpage\pagenumbering{roman}
\end{verbatim}\end{quote}

Next, comes your thesis or dissertation title, your name, date of graduation, department
and degree.
\begin{quote}\tt\singlespace\begin{verbatim}
\title{How to \LaTeX\ a Thesis}
\author{Eric R. Benedict}
\date{2000}
%   - The default degree is ``Doctor of Philosophy''
%     Degree can be changed using the command \degree{}
%\degree{New Degree}
%   - for a PhD dissertation (default), specify \dissertation
%\dissertation
%   - for a masters project report, specify \project
%\project
%   - for a preliminary report, specify \prelim
%\prelim
%   - for a masters thesis, specify \thesis
%\thesis
%   - The default department is ``Electrical Engineering''
%     The department can be changed using the command \department{}
%\department{New Department}
\end{verbatim}\end{quote}

If you specified the class option {\tt msthesis}, then the degree is changed to
{\em Master \break of Science} and the {\tt \verb|\thesis|} option is specified.  If you
want to have the masters margins with another document, then the {\tt \verb|\degree|}
and {\tt \verb|\dissertation|},  {\tt \verb|\project|}, {\em etc.\/} can be specified
as needed.

Once the
above are all defined, use  {\tt \verb|\maketitle|} to generate the title page.
\begin{quote}\tt\singlespace\begin{verbatim}
\maketitle
\end{verbatim}\end{quote}

If you wish to include a copyright page (see Section~\ref{copyright} for
information on registering the copyright.), then add the command
\begin{quote}\tt\singlespace\begin{verbatim}
\copyrightpage
\end{verbatim}\end{quote}
This will generate the proper copyright page and will use the name and date specified
in {\tt \verb|\author{}|} and {\tt \verb|\date{}|}.

Next are the dedications and acknowledgements:
\begin{quote}\tt\singlespace\begin{verbatim}
\begin{dedication}
To my pet rock, Skippy.
\end{dedication}

\begin{acknowledgments}
I thank the many people who have done lots of nice things for me.
\end{acknowledgments}
\end{verbatim}\end{quote}

You must tell \LaTeX{} to generate a table of contents, a list of tables and a list of
figures:
\begin{quote}\tt\singlespace\begin{verbatim}
\tableofcontents
\listoftables
\listoffigures
\end{verbatim}\end{quote}

If you wish to have a nomenclature, list of symbols or glossary it can go here.
\begin{quote}\tt\singlespace\begin{verbatim}
\begin{nomenclature}
%\begin{listofsymbols}
%\begin{glossary}
\begin{tabular}{ll}
$C_1$ & Constant 1\\
\ldots
\end{tabular}
%\end{glossary}
%\end{listofsymbols}
\end{nomenclature}
\end{verbatim}\end{quote}

If your abstract will be microfilmed by Bell and Howell (formerly UMI), then you
will need to generate an abstract of less than 350 words.  This abstract can be created
using the {\tt umiabstract} environment.  This environment requires that you define your
advisor and your advisor's title using {\tt \verb|\advisorname{}|} and
{\tt \verb|\advisortitle{}|}.
\begin{quote}\tt\singlespace\begin{verbatim}
\advisorname{Bucky J. Badger}
\advisortitle{Assistant Professor}
% ABSTRACT
\begin{umiabstract}
\noindent       % Don't indent first paragraph.
This explains the basics for using \LaTeX\ to typeset a
dissertation, thesis or project report for the University of
Wisconsin-Madison.

...

\end{umiabstract}
\end{verbatim}\end{quote}
This will place your name, title and required text at the top of the page and follow the
abstract text with your advisor's name at the bottom for your advisor's signature.  This
page is not numbered and would be submitted separately.

If you will have an abstract as part of your document, then the {\tt abstract} environment
should be used.
\begin{quote}\tt\singlespace\begin{verbatim}
\begin{abstract}
\noindent       % Don't indent first paragraph.
This explains the basics for using \LaTeX\ to typeset a
dissertation, thesis or project report for the University of
Wisconsin-Madison.

...

\end{abstract}
\end{verbatim}\end{quote}
This will generate a page number and it will be included in the Table
of Contents.  

If you will have both the UMI and regular abstracts like this document, then
you will probably want to write the abstract once and save it in a seperate
file such as {\tt abstract.tex}.  Then, you can use the same abstract for
both purposes.

\begin{quote}\begin{verbatim}
\begin{umiabstract}
  % abstract.tex
%
% This file has the abstract for the withesis style documentation
%
% Eric Benedict, Aug 2000
%
% It is provided without warranty on an AS IS basis.

\noindent       % Don't indent this paragraph.
This is not a thesis or dissertation and Master \TeX nician is not a
degree granted at the University of Wisconsin-Madison.

\vspace*{0.5em}
\noindent       % Don't indent this paragraph.
This explains the basics for using \LaTeX\ to typeset a dissertation,
thesis or masters project or preliminary report for the University of 
Wisconsin-Madison. Chapter
1 talks briefly about the thesis formatting at UW-Madison.  Chapter 2 gives
an overview of the ``essentials'' of \LaTeX{} and was written by Jon Warbrick.
Chapter 3 talks about figures and tables and what a {\em float} is.  Chapter 4
briefly introduces the {\sc Bib}\TeX{} program.  And finally, Chapter 5 discusses
some of the details for using the {\tt withesis} style file.  The material in
Chapters 2-4 basically are a review of fundamental \LaTeX{} usage and form
a reasonable basic tutorial.

\vspace*{0.5em}
\noindent       % Don't indent this paragraph.
The style discussed in this manual was originally written by Dave Kraynie and
edited by James Darrell McCauley as the {\tt puthesis} style for Purdue
University's theses.  This style was modified to form the {\tt withesis} style. This
manual is largely based on a similar manual by James Darrell McCauley and Scott Hucker.
Permission to use, copy, modify and distribute this software and its documentation
for any purpose and without fee is here by granted.  This software and its documentation
is provided ``as is'' without any express or implied warranty.

\end{umiabstract}

\begin{abstract}
  % abstract.tex
%
% This file has the abstract for the withesis style documentation
%
% Eric Benedict, Aug 2000
%
% It is provided without warranty on an AS IS basis.

\noindent       % Don't indent this paragraph.
This is not a thesis or dissertation and Master \TeX nician is not a
degree granted at the University of Wisconsin-Madison.

\vspace*{0.5em}
\noindent       % Don't indent this paragraph.
This explains the basics for using \LaTeX\ to typeset a dissertation,
thesis or masters project or preliminary report for the University of 
Wisconsin-Madison. Chapter
1 talks briefly about the thesis formatting at UW-Madison.  Chapter 2 gives
an overview of the ``essentials'' of \LaTeX{} and was written by Jon Warbrick.
Chapter 3 talks about figures and tables and what a {\em float} is.  Chapter 4
briefly introduces the {\sc Bib}\TeX{} program.  And finally, Chapter 5 discusses
some of the details for using the {\tt withesis} style file.  The material in
Chapters 2-4 basically are a review of fundamental \LaTeX{} usage and form
a reasonable basic tutorial.

\vspace*{0.5em}
\noindent       % Don't indent this paragraph.
The style discussed in this manual was originally written by Dave Kraynie and
edited by James Darrell McCauley as the {\tt puthesis} style for Purdue
University's theses.  This style was modified to form the {\tt withesis} style. This
manual is largely based on a similar manual by James Darrell McCauley and Scott Hucker.
Permission to use, copy, modify and distribute this software and its documentation
for any purpose and without fee is here by granted.  This software and its documentation
is provided ``as is'' without any express or implied warranty.

\end{abstract}
\end{verbatim}\end{quote}

Finally, the page numbers must be changed to arabic numbers to conclude the preliminary
portion of the document.
\begin{quote}\tt\singlespace\begin{verbatim}
\clearpage\pagenumbering{arabic}
\end{verbatim}\end{quote}

\section{The Body}
At the beginning of {\tt intro.tex} there is the following command:
\begin{quote}\tt\singlespace\begin{verbatim}
\chapter{Introducing the {\tt withesis} \LaTeX{} Style Guide}
\end{verbatim}\end{quote}
Following that is the text of the chapter.  The body of your thesis is separated by
sectioning commands like {\tt \verb|\chapter{}|}.  For more information on the sectioning
commands, see Section~\ref{ess:sectioning}.

Remember the basic rule of outlining you learned in grammar school:
\begin{quote}
You cannot have an `A' if you do not have a `B'
\end{quote}
Take care to have at least two {\tt \verb|\section|}s if you use the command; have
two {\tt \verb|\subsection|}s, {\em etc}.



\section{Additional Theorem Like Environments}
The {\tt withesis} style adds numerous additional theorem like environments.  These
environments were included to allow compatibility with the University of Wisconsin's
Math Department's style file.  These environments are
{\tt theorem}, {\tt assertion}, {\tt claim}, {\tt conjecture}, {\tt corollary},
{\tt definition}, {\tt example}, {\tt figger}, {\tt lemma}, {\tt prop} and {\tt remark}.

As an example, consider the following.
\begin{lemma}
Assuming that $\partial\Omega_2 = \emptyset$ and that $h(t) = 1$, we
have $$
\begin{array}{lr}
\Delta u = f, &  x\in\Omega ,\\[2pt]
u =  g_1, &  x\in\partial\Omega .
\end{array}
$$
\end{lemma}
which was produced with the following:
\begin{quote}\tt\singlespace\begin{verbatim}
\begin{lemma}
Assuming that $\partial\Omega_2 = \emptyset$ and that $h(t) = 1$, we
have $$
\begin{array}{lr}
\Delta u = f, &  x\in\Omega ,\\[2pt]
 u =  g_1, &  x\in\partial\Omega .
\end{array}
$$
\end{lemma}
\end{verbatim}\end{quote}

\section{Bibliography or References}
As a final note, the default title for the references chapter is ``LIST OF REFERENCES.''
Since some people may prefer ``BIBLIOGRAPHY'', the command
\break{\tt \verb|\altbibtitle|}
has been added to change the chapter title.

\section{Appendices}
There are two commands which are available to suppress the writing of the auxiliary information
(to the {\tt .lot} and {\tt .lof} files).  They are:
\begin{quote}\tt\singlespace\begin{verbatim}
\noappendixtables                % Don't have appendix tables
\noappendixfigures               % Don't have appendix figures
\end{verbatim}\end{quote}
These commands should be in the preamble.  See Section~\ref{usage:noapp}.

There are two environments for doing the appendix chapter: {\tt appendix} and
\break {\tt appendices}.  If you have only one chapter in the appendix, use the {\tt appendix}
environment.  If you have more than one chapter, like this manual, use the
{\tt appendices} environment.
\begin{quote}\tt\singlespace\footnotesize\begin{verbatim}
\begin{appendices}  % Start of the Appendix Chapters.  If there is only
                    % one Appendix Chapter, then use \begin{appendix}
% code.tex
% this file is part of the example UW-Madison Thesis document
% It demonstrates one method for incorporating program listings
% into a document.

\chapter{Matlab Code} \label{matlab}
This is an example of a Matlab m-file.
\verbatimfile{derivs.m}
      % Including computer code listings
\chapter{Bib\TeX\ Entries}
\label{bibrefs}
The following shows the fields required in all types of Bib\TeX\ entries.
Fields with {\tt OPT} prefixed are optional (the three letters {\tt OPT} should 
not be used).  If an optional field is not used, then the entire field can be deleted.

{\tt
\singlespace
\begin{verbatim}

@Unpublished{,                            @Manual{,
  author =      "",                         title =           "",
  title =       "",                         OPTauthor =       "",
  note =        "",                         OPTorganization = "",
  OPTyear =     "",                         OPTaddress =      "",
  OPTmonth =    ""                          OPTedition =      "",
}                                           OPTyear =         "",
                                            OPTmonth =        "",
@TechReport{,                               OPTnote =         "" 
  author =      "",                       }
  title =       "",
  institution = "",                       @InProceedings{,
  year =        "",                         author =          "",
  OPTtype =     "",                         title =           "",
  OPTnumber =   "",                         booktitle =       "",
  OPTaddress =  "",                         year =            "",
  OPTmonth =    "",                         OPTeditor =       "",
  OPTnote =     ""                          OPTpages =        "",
}                                           OPTorganization = "",
                                            OPTpublisher =    "",
@Proceedings{,                              OPTaddress =      "",
  title =           "",                     OPTmonth =        "",
  year =            "",                     OPTnote =         "" 
  OPTeditor =       "",                   }
  OPTpublisher =    "",
  OPTorganization = "",
  OPTaddress =      "",
  OPTmonth =        "",
  OPTnote =         "" 
}



@PhDThesis{,                              @InCollection{,
  author =      "",                         author =          "",
  title =       "",                         title =           "",
  school =      "",                         booktitle =       "",
  year =        "",                         publisher =       "",
  OPTaddress =  "",                         year =            "",
  OPTmonth =    "",                         OPTeditor =       "",
  OPTnote =     ""                          OPTchapter =      "",
}                                           OPTpages =        "",
                                            OPTaddress =      "",
                                            OPTmonth =        "",
                                            OPTnote =         ""
                                          }

 
@Misc{,                                   @InCollection{,
  OPTauthor =       "",                     author =          "",
  OPTtitle =        "",                     title =           "",
  OPThowpublished = "",                     chapter =         "",
  OPTyear =         "",                     publisher =       "",
  OPTmonth =        "",                     year =            "",
  OPTnote =         ""                      OPTeditor =       "",
}                                           OPTpages =        "",
}                                           OPTvolume =       "",
                                            OPTseries =       "",
                                            OPTaddress =      "",
                                            OPTedition =      "",
                                            OPTmonth =        "",
                                            OPTnote =         ""
                                          }

@MastersThesis{,                          @Article{,
  author =      "",                         author =          "",
  title =       "",                         title =           "",
  school =      "",                         journal =         "",
  year =        "",                         year =            "",
  OPTaddress =  "",                         OPTvolume =       "",
  OPTmonth =    "",                         OPTnumber =       "",
  OPTnote =     ""                          OPTpages =        "",
}                                           OPTmonth =        "",
                                            OPTnote =         ""
                                           }\end{verbatim} }
    % a BibTeX reference
\chapter{Mathematics Examples}
This appendix provides an example of \LaTeX's typesetting
capabilities.  Most of text was obtained from the University of
Wisconsin-Madison Math Department's example thesis file.

\section{Matrices}
The equations for the {\em dq}-model of an induction machine in the
synchronous reference frame are
\begin{eqnarray}
 \left[\begin{array}{c} v_{qs}^e\\v_{ds}^e\\v_{qr}^e\\v_{dr}^e  \end{array}\right]                                                                                                                                                                                                                                                                                                                                                                                                                                                                                                              
 &=& \left[ \begin{array}{cccc}
 r_s + x_s\frac{\rho}{\omega_b} & \frac{\omega_e}{\omega_b}x_s & x_m\frac{\rho}{\omega_b} & \frac{\omega_e}{\omega_b}x_m \\
 -\frac{\omega_e}{\omega_b}x_s & r_s + x_s\frac{\rho}{\omega_b} & -\frac{\omega_e}{\omega_b}x_m & x_m\frac{\rho}{\omega_b} \\
 x_m\frac{\rho}{\omega_b} & \frac{\omega_e -\omega_r}{\omega_b}x_m & r_r'+x_r'\frac{\rho}{\omega_b} & \frac{\omega_e - \omega_r}{\omega_b}x_r' \\
 -\frac{\omega_e -\omega_r}{\omega_b}x_m & x_m\frac{\rho}{\omega_b} & -\frac{\omega_e - \omega_r}{\omega_b}x_r' & r_r' + x_r'\frac{\rho}{\omega_b}
 \end{array} \right]
 \left[\begin{array}{c} i_{qs}^e\\i_{ds}^e\\i_{qr}^e\\i_{dr}^e\end{array} \right] \label{volteq}\\
 T_e&=&\frac{3}{2}\frac{P}{2}\frac{x_m}{\omega_b}\left(i_{qs}^ei_{dr}^e - i_{ds}^ei_{qr}^e\right) \label{torqueeq}\\
 T_e-T_l&=&\frac{2J\omega_b}{P}\frac{d}{dt}\left(\frac{\omega_r}{\omega_b}\right) \label{mecheq}.
\end{eqnarray}

\section{Multi-line Equations}

\LaTeX{} has a built-in equation array feature, however the
equation numbers must be on the same line as an equation.  For example:
\begin{eqnarray}
\Delta u + \lambda e^u &= 0&u\in \Omega,  \nonumber \\
u&=0&u\in\partial\Omega.
\end{eqnarray}

Alternatively, the number can be centered in the equation using the
following method.
%
% The equation-array feature in LaTeX is a bad idea.  For centered
% numbers you should set your own equations and arrays as follows:
%
\def\dd{\displaystyle}
\begin{equation}\label{gelfand}
\begin{array}{rl}
\dd \Delta u + \lambda e^u = 0, &
\dd u\in \Omega,\\[8pt] % add 8pt extra vertical space. 1 line=23pt
\dd u=0, & \dd u\in\partial\Omega.
\end{array}
\end{equation}
The previous equation had a label.  It may be referenced as
equation~(\ref{gelfand}).


\section{More Complicated Equations}
\section*{Rellich's identity}\label{rellich.section}
\setcounter{theorem}{0}
%
%

Standard developments of Pohozaev's identity used an identity by
Rellich~\cite{rellich:der40}, reproduced here.

\begin{lemma}[Rellich]
Given $L$ in divergence form and $a,d$ defined above, $u\in C^2
(\Omega )$, we have
\begin{equation}\label{rellich}
\int_{\Omega}(-Lu)\nabla u\cdot (x-\overline{x})\, dx
= (1-\frac{n}{2}) \int_{\Omega} a(\nabla u,\nabla u) \, dx
-
\frac{1}{2} \int_{\Omega}
d(\nabla u, \nabla u) \, dx
\end{equation}
$$
+
\frac{1}{2} \int_{\partial\Omega} a(\nabla u,\nabla u)(x-\overline{x})
\cdot \nu  \, dS
-
\int_{\partial\Omega}
a(\nabla u,\nu )\nabla u\cdot (x-\overline{x}) \, dS.
$$
\end{lemma}
{\bf Proof:}\\
There is no loss in generality to take $\overline{x} = 0$. First
rewrite $L$:
$$Lu = \frac{1}{2}\left[ \sum_{i}\sum_{j}
\frac{\partial}{\partial x_i}
\left( a_{ij} \frac{\partial u}{\partial x_j} \right) +
\sum_{i}\sum_{j}
\frac{\partial}{\partial x_i}
\left( a_{ij} \frac{\partial u}{\partial x_j} \right)
\right]$$
Switching the order of summation on the second term and relabeling
subscripts, $j \rightarrow i$ and $i \rightarrow j$, then using the fact
that $a_{ij}(x)$ is a symmetric matrix,
gives the symmetric form needed to derive Rellich's identity.
\begin{equation}
Lu = \frac{1}{2} \sum_{i,j}\left[
\frac{\partial}{\partial x_i}
\left( a_{ij} \frac{\partial u}{\partial x_j} \right) +
\frac{\partial}{\partial x_j}
\left( a_{ij} \frac{\partial u}{\partial x_i} \right)
\right].
\end{equation}

Multiplying $-Lu$ by $\frac{\partial u}{\partial x_k} x_k$ and integrating
over $\Omega$, yields
$$\int_{\Omega}(-Lu)\frac{\partial u}{\partial x_k} x_k \, dx=
-\frac{1}{2} \int_{\Omega}
\sum_{i,j}\left[
\frac{\partial}{\partial x_i}
\left( a_{ij} \frac{\partial u}{\partial x_j} \right) +
\frac{\partial}{\partial x_j}
\left( a_{ij} \frac{\partial u}{\partial x_i} \right)
\right]
\frac{\partial u}{\partial x_k} x_k \, dx$$
Integrating by parts (for integral theorems see~\cite[p. 20]
{zeidler:nfa88IIa})
gives
$$= \frac{1}{2} \int_{\Omega}
\sum_{i,j} a_{ij} \left[
\frac{\partial u}{\partial x_j}
\frac{\partial^2 u}{\partial x_k\partial x_i} +
\frac{\partial u}{\partial x_i}
\frac{\partial^2 u}{\partial x_k\partial x_j}
\right] x_k \, dx
$$
$$
+
\frac{1}{2} \int_{\Omega}
\sum_{i,j} a_{ij} \left[
\frac{\partial u}{\partial x_j} \delta_{ik} +
\frac{\partial u}{\partial x_i} \delta_{jk}
\right] \frac{\partial u}{\partial x_k} \, dx
$$
$$- \frac{1}{2} \int_{\partial\Omega}
\sum_{i,j} a_{ij} \left[
\frac{\partial u}{\partial x_j} \nu_{i} +
\frac{\partial u}{\partial x_i} \nu_{j}
\right] \frac{\partial u}{\partial x_k} x_k \, dx
$$
= $I_1 + I_2 + I_3$, where the unit normal vector is $\nu$.
One may rewrite $I_1$ as
$$I_1 = \frac{1}{2} \int_{\Omega}
\sum_{i,j} a_{ij} \frac{\partial}{\partial x_k}\left(
\frac{\partial u}{\partial x_i}
\frac{\partial u}{\partial x_j}
\right) x_k \, dx
$$
Integrating the first term by parts again yields
$$I_1 = -\frac{1}{2} \int_{\Omega}
\sum_{i,j} a_{ij} \left(
\frac{\partial u}{\partial x_i}
\frac{\partial u}{\partial x_j}
\right) \, dx
+
\frac{1}{2} \int_{\partial\Omega}
\sum_{i,j} a_{ij} \left(
\frac{\partial u}{\partial x_i}
\frac{\partial u}{\partial x_j}
\right) x_k \nu_k \, dS
$$
$$
-
\frac{1}{2} \int_{\Omega}
\sum_{i,j} \left(
\frac{\partial u}{\partial x_i}
\frac{\partial u}{\partial x_j}
\right) x_k \frac{\partial a_{ij}}{\partial x_k}\, dx.
$$
Summing over $k$ gives
$$\int_{\Omega}(-Lu)(\nabla u\cdot x)\, dx =
-\frac{n}{2} \int_{\Omega}
\sum_{i,j} a_{ij} \left(
\frac{\partial u}{\partial x_i}
\frac{\partial u}{\partial x_j}
\right) \, dx
$$
$$
+
\frac{1}{2} \int_{\partial\Omega}
\sum_{i,j} a_{ij} \left(
\frac{\partial u}{\partial x_i}
\frac{\partial u}{\partial x_j}
\right) (x\cdot \nu ) \, dS
-
\frac{1}{2} \int_{\Omega}
\sum_{i,j} \left(
\frac{\partial u}{\partial x_i}
\frac{\partial u}{\partial x_j}
\right) (x\cdot  \nabla a_{ij}) \, dx
$$
$$
+
\frac{1}{2} \int_{\Omega}
\sum_{i,j,k} a_{ij} \left[
\frac{\partial u}{\partial x_j}
\frac{\partial u}{\partial x_k} \delta_{ik} +
\frac{\partial u}{\partial x_i}
\frac{\partial u}{\partial x_k} \delta_{jk}
\right] \, dx
$$
$$- \frac{1}{2} \int_{\partial\Omega}
\sum_{i,j} a_{ij} \left[
\frac{\partial u}{\partial x_j} \nu_{i} +
\frac{\partial u}{\partial x_i} \nu_{j}
\right] (\nabla u\cdot x) \, dS.
$$
Combining the first and fourth term on the right-hand side
simplifies the expression
$$\int_{\Omega}(-Lu)(\nabla u\cdot x)\, dx
=
(1-\frac{n}{2}) \int_{\Omega}
\sum_{i,j} a_{ij} \left(
\frac{\partial u}{\partial x_i}
\frac{\partial u}{\partial x_j}
\right) \, dx
$$
$$
+
\frac{1}{2} \int_{\partial\Omega}
\sum_{i,j} a_{ij} \left(
\frac{\partial u}{\partial x_i}
\frac{\partial u}{\partial x_j}
\right) (x\cdot \nu ) \, dS
-
\frac{1}{2} \int_{\Omega}
\sum_{i,j} \left(
\frac{\partial u}{\partial x_i}
\frac{\partial u}{\partial x_j}
\right) (x\cdot  \nabla a_{ij}) \, dx
$$
$$
-
\frac{1}{2} \int_{\partial\Omega}
\sum_{i,j} a_{ij} \left[
\frac{\partial u}{\partial x_j} \nu_{i} +
\frac{\partial u}{\partial x_i} \nu_{j}
\right] (\nabla u\cdot x) \, dS.
$$
Using the notation defined above, the result follows.


      % Complex Equations from the UW Math Department

\end{appendices}    % End of the Appendix Chapters. ibid on \end{appendix}
\end{verbatim}\end{quote}
The difference between these two environments is the way that the chapter header is
created and how this is listed in the table of contents.
          % Chapter 5 Strongly based on similar by J.D. McCauley
\bibliography{refs}      % Make the bibliography
\begin{appendices}       % Start of the Appendix Chapters.  If there is only
                         % one Appendix Chapter, then use \begin{appendix}
% code.tex
% this file is part of the example UW-Madison Thesis document
% It demonstrates one method for incorporating program listings
% into a document.

\chapter{Matlab Code} \label{matlab}
This is an example of a Matlab m-file.
\verbatimfile{derivs.m}
         % Including computer code listings
\chapter{Bib\TeX\ Entries}
\label{bibrefs}
The following shows the fields required in all types of Bib\TeX\ entries.
Fields with {\tt OPT} prefixed are optional (the three letters {\tt OPT} should 
not be used).  If an optional field is not used, then the entire field can be deleted.

{\tt
\singlespace
\begin{verbatim}

@Unpublished{,                            @Manual{,
  author =      "",                         title =           "",
  title =       "",                         OPTauthor =       "",
  note =        "",                         OPTorganization = "",
  OPTyear =     "",                         OPTaddress =      "",
  OPTmonth =    ""                          OPTedition =      "",
}                                           OPTyear =         "",
                                            OPTmonth =        "",
@TechReport{,                               OPTnote =         "" 
  author =      "",                       }
  title =       "",
  institution = "",                       @InProceedings{,
  year =        "",                         author =          "",
  OPTtype =     "",                         title =           "",
  OPTnumber =   "",                         booktitle =       "",
  OPTaddress =  "",                         year =            "",
  OPTmonth =    "",                         OPTeditor =       "",
  OPTnote =     ""                          OPTpages =        "",
}                                           OPTorganization = "",
                                            OPTpublisher =    "",
@Proceedings{,                              OPTaddress =      "",
  title =           "",                     OPTmonth =        "",
  year =            "",                     OPTnote =         "" 
  OPTeditor =       "",                   }
  OPTpublisher =    "",
  OPTorganization = "",
  OPTaddress =      "",
  OPTmonth =        "",
  OPTnote =         "" 
}



@PhDThesis{,                              @InCollection{,
  author =      "",                         author =          "",
  title =       "",                         title =           "",
  school =      "",                         booktitle =       "",
  year =        "",                         publisher =       "",
  OPTaddress =  "",                         year =            "",
  OPTmonth =    "",                         OPTeditor =       "",
  OPTnote =     ""                          OPTchapter =      "",
}                                           OPTpages =        "",
                                            OPTaddress =      "",
                                            OPTmonth =        "",
                                            OPTnote =         ""
                                          }

 
@Misc{,                                   @InCollection{,
  OPTauthor =       "",                     author =          "",
  OPTtitle =        "",                     title =           "",
  OPThowpublished = "",                     chapter =         "",
  OPTyear =         "",                     publisher =       "",
  OPTmonth =        "",                     year =            "",
  OPTnote =         ""                      OPTeditor =       "",
}                                           OPTpages =        "",
}                                           OPTvolume =       "",
                                            OPTseries =       "",
                                            OPTaddress =      "",
                                            OPTedition =      "",
                                            OPTmonth =        "",
                                            OPTnote =         ""
                                          }

@MastersThesis{,                          @Article{,
  author =      "",                         author =          "",
  title =       "",                         title =           "",
  school =      "",                         journal =         "",
  year =        "",                         year =            "",
  OPTaddress =  "",                         OPTvolume =       "",
  OPTmonth =    "",                         OPTnumber =       "",
  OPTnote =     ""                          OPTpages =        "",
}                                           OPTmonth =        "",
                                            OPTnote =         ""
                                           }\end{verbatim} }
         % a BibTeX reference
\chapter{Mathematics Examples}
This appendix provides an example of \LaTeX's typesetting
capabilities.  Most of text was obtained from the University of
Wisconsin-Madison Math Department's example thesis file.

\section{Matrices}
The equations for the {\em dq}-model of an induction machine in the
synchronous reference frame are
\begin{eqnarray}
 \left[\begin{array}{c} v_{qs}^e\\v_{ds}^e\\v_{qr}^e\\v_{dr}^e  \end{array}\right]                                                                                                                                                                                                                                                                                                                                                                                                                                                                                                              
 &=& \left[ \begin{array}{cccc}
 r_s + x_s\frac{\rho}{\omega_b} & \frac{\omega_e}{\omega_b}x_s & x_m\frac{\rho}{\omega_b} & \frac{\omega_e}{\omega_b}x_m \\
 -\frac{\omega_e}{\omega_b}x_s & r_s + x_s\frac{\rho}{\omega_b} & -\frac{\omega_e}{\omega_b}x_m & x_m\frac{\rho}{\omega_b} \\
 x_m\frac{\rho}{\omega_b} & \frac{\omega_e -\omega_r}{\omega_b}x_m & r_r'+x_r'\frac{\rho}{\omega_b} & \frac{\omega_e - \omega_r}{\omega_b}x_r' \\
 -\frac{\omega_e -\omega_r}{\omega_b}x_m & x_m\frac{\rho}{\omega_b} & -\frac{\omega_e - \omega_r}{\omega_b}x_r' & r_r' + x_r'\frac{\rho}{\omega_b}
 \end{array} \right]
 \left[\begin{array}{c} i_{qs}^e\\i_{ds}^e\\i_{qr}^e\\i_{dr}^e\end{array} \right] \label{volteq}\\
 T_e&=&\frac{3}{2}\frac{P}{2}\frac{x_m}{\omega_b}\left(i_{qs}^ei_{dr}^e - i_{ds}^ei_{qr}^e\right) \label{torqueeq}\\
 T_e-T_l&=&\frac{2J\omega_b}{P}\frac{d}{dt}\left(\frac{\omega_r}{\omega_b}\right) \label{mecheq}.
\end{eqnarray}

\section{Multi-line Equations}

\LaTeX{} has a built-in equation array feature, however the
equation numbers must be on the same line as an equation.  For example:
\begin{eqnarray}
\Delta u + \lambda e^u &= 0&u\in \Omega,  \nonumber \\
u&=0&u\in\partial\Omega.
\end{eqnarray}

Alternatively, the number can be centered in the equation using the
following method.
%
% The equation-array feature in LaTeX is a bad idea.  For centered
% numbers you should set your own equations and arrays as follows:
%
\def\dd{\displaystyle}
\begin{equation}\label{gelfand}
\begin{array}{rl}
\dd \Delta u + \lambda e^u = 0, &
\dd u\in \Omega,\\[8pt] % add 8pt extra vertical space. 1 line=23pt
\dd u=0, & \dd u\in\partial\Omega.
\end{array}
\end{equation}
The previous equation had a label.  It may be referenced as
equation~(\ref{gelfand}).


\section{More Complicated Equations}
\section*{Rellich's identity}\label{rellich.section}
\setcounter{theorem}{0}
%
%

Standard developments of Pohozaev's identity used an identity by
Rellich~\cite{rellich:der40}, reproduced here.

\begin{lemma}[Rellich]
Given $L$ in divergence form and $a,d$ defined above, $u\in C^2
(\Omega )$, we have
\begin{equation}\label{rellich}
\int_{\Omega}(-Lu)\nabla u\cdot (x-\overline{x})\, dx
= (1-\frac{n}{2}) \int_{\Omega} a(\nabla u,\nabla u) \, dx
-
\frac{1}{2} \int_{\Omega}
d(\nabla u, \nabla u) \, dx
\end{equation}
$$
+
\frac{1}{2} \int_{\partial\Omega} a(\nabla u,\nabla u)(x-\overline{x})
\cdot \nu  \, dS
-
\int_{\partial\Omega}
a(\nabla u,\nu )\nabla u\cdot (x-\overline{x}) \, dS.
$$
\end{lemma}
{\bf Proof:}\\
There is no loss in generality to take $\overline{x} = 0$. First
rewrite $L$:
$$Lu = \frac{1}{2}\left[ \sum_{i}\sum_{j}
\frac{\partial}{\partial x_i}
\left( a_{ij} \frac{\partial u}{\partial x_j} \right) +
\sum_{i}\sum_{j}
\frac{\partial}{\partial x_i}
\left( a_{ij} \frac{\partial u}{\partial x_j} \right)
\right]$$
Switching the order of summation on the second term and relabeling
subscripts, $j \rightarrow i$ and $i \rightarrow j$, then using the fact
that $a_{ij}(x)$ is a symmetric matrix,
gives the symmetric form needed to derive Rellich's identity.
\begin{equation}
Lu = \frac{1}{2} \sum_{i,j}\left[
\frac{\partial}{\partial x_i}
\left( a_{ij} \frac{\partial u}{\partial x_j} \right) +
\frac{\partial}{\partial x_j}
\left( a_{ij} \frac{\partial u}{\partial x_i} \right)
\right].
\end{equation}

Multiplying $-Lu$ by $\frac{\partial u}{\partial x_k} x_k$ and integrating
over $\Omega$, yields
$$\int_{\Omega}(-Lu)\frac{\partial u}{\partial x_k} x_k \, dx=
-\frac{1}{2} \int_{\Omega}
\sum_{i,j}\left[
\frac{\partial}{\partial x_i}
\left( a_{ij} \frac{\partial u}{\partial x_j} \right) +
\frac{\partial}{\partial x_j}
\left( a_{ij} \frac{\partial u}{\partial x_i} \right)
\right]
\frac{\partial u}{\partial x_k} x_k \, dx$$
Integrating by parts (for integral theorems see~\cite[p. 20]
{zeidler:nfa88IIa})
gives
$$= \frac{1}{2} \int_{\Omega}
\sum_{i,j} a_{ij} \left[
\frac{\partial u}{\partial x_j}
\frac{\partial^2 u}{\partial x_k\partial x_i} +
\frac{\partial u}{\partial x_i}
\frac{\partial^2 u}{\partial x_k\partial x_j}
\right] x_k \, dx
$$
$$
+
\frac{1}{2} \int_{\Omega}
\sum_{i,j} a_{ij} \left[
\frac{\partial u}{\partial x_j} \delta_{ik} +
\frac{\partial u}{\partial x_i} \delta_{jk}
\right] \frac{\partial u}{\partial x_k} \, dx
$$
$$- \frac{1}{2} \int_{\partial\Omega}
\sum_{i,j} a_{ij} \left[
\frac{\partial u}{\partial x_j} \nu_{i} +
\frac{\partial u}{\partial x_i} \nu_{j}
\right] \frac{\partial u}{\partial x_k} x_k \, dx
$$
= $I_1 + I_2 + I_3$, where the unit normal vector is $\nu$.
One may rewrite $I_1$ as
$$I_1 = \frac{1}{2} \int_{\Omega}
\sum_{i,j} a_{ij} \frac{\partial}{\partial x_k}\left(
\frac{\partial u}{\partial x_i}
\frac{\partial u}{\partial x_j}
\right) x_k \, dx
$$
Integrating the first term by parts again yields
$$I_1 = -\frac{1}{2} \int_{\Omega}
\sum_{i,j} a_{ij} \left(
\frac{\partial u}{\partial x_i}
\frac{\partial u}{\partial x_j}
\right) \, dx
+
\frac{1}{2} \int_{\partial\Omega}
\sum_{i,j} a_{ij} \left(
\frac{\partial u}{\partial x_i}
\frac{\partial u}{\partial x_j}
\right) x_k \nu_k \, dS
$$
$$
-
\frac{1}{2} \int_{\Omega}
\sum_{i,j} \left(
\frac{\partial u}{\partial x_i}
\frac{\partial u}{\partial x_j}
\right) x_k \frac{\partial a_{ij}}{\partial x_k}\, dx.
$$
Summing over $k$ gives
$$\int_{\Omega}(-Lu)(\nabla u\cdot x)\, dx =
-\frac{n}{2} \int_{\Omega}
\sum_{i,j} a_{ij} \left(
\frac{\partial u}{\partial x_i}
\frac{\partial u}{\partial x_j}
\right) \, dx
$$
$$
+
\frac{1}{2} \int_{\partial\Omega}
\sum_{i,j} a_{ij} \left(
\frac{\partial u}{\partial x_i}
\frac{\partial u}{\partial x_j}
\right) (x\cdot \nu ) \, dS
-
\frac{1}{2} \int_{\Omega}
\sum_{i,j} \left(
\frac{\partial u}{\partial x_i}
\frac{\partial u}{\partial x_j}
\right) (x\cdot  \nabla a_{ij}) \, dx
$$
$$
+
\frac{1}{2} \int_{\Omega}
\sum_{i,j,k} a_{ij} \left[
\frac{\partial u}{\partial x_j}
\frac{\partial u}{\partial x_k} \delta_{ik} +
\frac{\partial u}{\partial x_i}
\frac{\partial u}{\partial x_k} \delta_{jk}
\right] \, dx
$$
$$- \frac{1}{2} \int_{\partial\Omega}
\sum_{i,j} a_{ij} \left[
\frac{\partial u}{\partial x_j} \nu_{i} +
\frac{\partial u}{\partial x_i} \nu_{j}
\right] (\nabla u\cdot x) \, dS.
$$
Combining the first and fourth term on the right-hand side
simplifies the expression
$$\int_{\Omega}(-Lu)(\nabla u\cdot x)\, dx
=
(1-\frac{n}{2}) \int_{\Omega}
\sum_{i,j} a_{ij} \left(
\frac{\partial u}{\partial x_i}
\frac{\partial u}{\partial x_j}
\right) \, dx
$$
$$
+
\frac{1}{2} \int_{\partial\Omega}
\sum_{i,j} a_{ij} \left(
\frac{\partial u}{\partial x_i}
\frac{\partial u}{\partial x_j}
\right) (x\cdot \nu ) \, dS
-
\frac{1}{2} \int_{\Omega}
\sum_{i,j} \left(
\frac{\partial u}{\partial x_i}
\frac{\partial u}{\partial x_j}
\right) (x\cdot  \nabla a_{ij}) \, dx
$$
$$
-
\frac{1}{2} \int_{\partial\Omega}
\sum_{i,j} a_{ij} \left[
\frac{\partial u}{\partial x_j} \nu_{i} +
\frac{\partial u}{\partial x_i} \nu_{j}
\right] (\nabla u\cdot x) \, dS.
$$
Using the notation defined above, the result follows.


           % Complex Equations from the UW Math Department
% acrobat.tex
%
% This file explains how to generate Adobe Acrobat files
%
% Eric Benedict, July 2000
%
% It is provided without warranty on an AS IS basis.

\newcommand{\pdf}{\mbox{\tt *.pdf}}

\chapter{Adobe Acrobat (\pdf ) Files}
The Adobe Acrobat file format has pretty much become the {\em de facto}
standard for document sharing.  As such, some faculty members and/or
departments may be requiring a final copy of the thesis in Acrobat format
(\pdf ).

There are several different methods of obtaining a \pdf\ file from a \LaTeX{}
thesis; however, they are all very site specific.  A couple of different
methods which have been found to work are mentioned as suggested ideas to try
as a starting point.  Depending on what is installed at your site/location some
of these may be applicable.

\section{Converting from {\tt *.ps} to \pdf}
One option to obtain the \pdf\ file would be to generate the thesis in a normal
manner and then use the Acrobat\ {\tt Distiller}\ to convert the postscript file into a
\pdf\ file.

If the\ {\tt Distiller}\ program is available and convenient to use, then this is quite easy to do.

Depending on the choice of document fonts, the results may not be satisfactory since
some of the fonts may end up as bit-mapped fonts and will display poorly at any resolution
other than what they were sampled on.  Also, since the\ {\tt Distiller}\ program is an expensive
program to obtain, it is not always available.

An alternative to the Adobe\ {\tt Distiller}\ program is the Alladin\ {\tt Ghostscript}\ program.  This is
available for free from

{\tt \verb|   http://www.cs.wisc.edu/~ghost/index.html|}

This program is available for most common operating systems as a compiled binary, but the source code
is available for other systems.  One drawback is that this conversion must be performed as a
command line invocation and isn't very user friendly.  This may be addressed in a future version of\
{\tt Ghostview}, the program which provides a nice user interface to\ {\tt Ghostscript}.

\section{Converting from {\tt *.dvi} to \pdf}
There are two programs available which will convert from {\tt *.dvi} to \pdf,\ {\tt dvipdf}\ and\
{\tt dvipdfm}.  The\ {\tt dvipdfm}\ program  will be discussed here.  In version 0.12, it can generate
bookmarks, thumbnails (with assistance from\ {\tt Ghostscript}), scaling and rotation, JPEG and
PNG bitmaps and font encoding and re-encoding (to support fonts which aren't fully supported by
the Acrobat suite).  When\ {\tt Ghostscript}\ is properly installed,\ {\tt dvipdfm}\ will automatically
convert any encapsulated PostScript figures into the required \pdf{} format.  This program behaved in a
similar manner to the {\tt dvips} program and was used to produce the \pdf{} format of this document.



\section{Generating \pdf{} Initially}
There are now some programs which are similar to \TeX{} but instead of producing a\ {\tt .dvi}\ output,
they produce \pdf as a native output.  One such program, {\sc pdf}\TeX{} / {\sc pdf}\LaTeX{},
is available from

{\tt \verb|   http://www.tug.org/applications/pdftex|}

Note that as of this date, July 2000, {\sc pdf}\TeX{} / {\sc pdf}\LaTeX{} while currently quite usable, it
is still in a beta version.  Look at the web site for more current information.

The present version was able to produce a \pdf{} file of this document without any required
changes, except for the Postscript figure inclusion  (Figure~\ref{vwcontr}).  To properly include
this figure, requires the conversion of the postscript figure into a \pdf{} figure.  The procedure
is described in the manual for {\sc pdf}\TeX{} / {\sc pdf}\LaTeX{}.  Note that the figure conversion will
require either\ {\tt Distiller}\ or\ {\tt Ghostscript}.
           % A discussion on generating PDF files.
\end{appendices}         % End of the Appendix Chapters.  ibid on \end{appendix}
%\include{vita}          % Optional Vita, use \begin{vita} vita text \end{vita}
\end{document}
\end{verbatim}
\end{quote}

\section{Prelude}
After the {\tt \verb|\begin{document}|} comes the preliminary information found in
theses.  In this manual, the information is kept in the file {\tt prelude.tex} (see
above).  These pages will need to be numbered with roman numerals, so use
\begin{quote}\tt\singlespace\begin{verbatim}
\clearpage\pagenumbering{roman}
\end{verbatim}\end{quote}

Next, comes your thesis or dissertation title, your name, date of graduation, department
and degree.
\begin{quote}\tt\singlespace\begin{verbatim}
\title{How to \LaTeX\ a Thesis}
\author{Eric R. Benedict}
\date{2000}
%   - The default degree is ``Doctor of Philosophy''
%     Degree can be changed using the command \degree{}
%\degree{New Degree}
%   - for a PhD dissertation (default), specify \dissertation
%\dissertation
%   - for a masters project report, specify \project
%\project
%   - for a preliminary report, specify \prelim
%\prelim
%   - for a masters thesis, specify \thesis
%\thesis
%   - The default department is ``Electrical Engineering''
%     The department can be changed using the command \department{}
%\department{New Department}
\end{verbatim}\end{quote}

If you specified the class option {\tt msthesis}, then the degree is changed to
{\em Master \break of Science} and the {\tt \verb|\thesis|} option is specified.  If you
want to have the masters margins with another document, then the {\tt \verb|\degree|}
and {\tt \verb|\dissertation|},  {\tt \verb|\project|}, {\em etc.\/} can be specified
as needed.

Once the
above are all defined, use  {\tt \verb|\maketitle|} to generate the title page.
\begin{quote}\tt\singlespace\begin{verbatim}
\maketitle
\end{verbatim}\end{quote}

If you wish to include a copyright page (see Section~\ref{copyright} for
information on registering the copyright.), then add the command
\begin{quote}\tt\singlespace\begin{verbatim}
\copyrightpage
\end{verbatim}\end{quote}
This will generate the proper copyright page and will use the name and date specified
in {\tt \verb|\author{}|} and {\tt \verb|\date{}|}.

Next are the dedications and acknowledgements:
\begin{quote}\tt\singlespace\begin{verbatim}
\begin{dedication}
To my pet rock, Skippy.
\end{dedication}

\begin{acknowledgments}
I thank the many people who have done lots of nice things for me.
\end{acknowledgments}
\end{verbatim}\end{quote}

You must tell \LaTeX{} to generate a table of contents, a list of tables and a list of
figures:
\begin{quote}\tt\singlespace\begin{verbatim}
\tableofcontents
\listoftables
\listoffigures
\end{verbatim}\end{quote}

If you wish to have a nomenclature, list of symbols or glossary it can go here.
\begin{quote}\tt\singlespace\begin{verbatim}
\begin{nomenclature}
%\begin{listofsymbols}
%\begin{glossary}
\begin{tabular}{ll}
$C_1$ & Constant 1\\
\ldots
\end{tabular}
%\end{glossary}
%\end{listofsymbols}
\end{nomenclature}
\end{verbatim}\end{quote}

If your abstract will be microfilmed by Bell and Howell (formerly UMI), then you
will need to generate an abstract of less than 350 words.  This abstract can be created
using the {\tt umiabstract} environment.  This environment requires that you define your
advisor and your advisor's title using {\tt \verb|\advisorname{}|} and
{\tt \verb|\advisortitle{}|}.
\begin{quote}\tt\singlespace\begin{verbatim}
\advisorname{Bucky J. Badger}
\advisortitle{Assistant Professor}
% ABSTRACT
\begin{umiabstract}
\noindent       % Don't indent first paragraph.
This explains the basics for using \LaTeX\ to typeset a
dissertation, thesis or project report for the University of
Wisconsin-Madison.

...

\end{umiabstract}
\end{verbatim}\end{quote}
This will place your name, title and required text at the top of the page and follow the
abstract text with your advisor's name at the bottom for your advisor's signature.  This
page is not numbered and would be submitted separately.

If you will have an abstract as part of your document, then the {\tt abstract} environment
should be used.
\begin{quote}\tt\singlespace\begin{verbatim}
\begin{abstract}
\noindent       % Don't indent first paragraph.
This explains the basics for using \LaTeX\ to typeset a
dissertation, thesis or project report for the University of
Wisconsin-Madison.

...

\end{abstract}
\end{verbatim}\end{quote}
This will generate a page number and it will be included in the Table
of Contents.  

If you will have both the UMI and regular abstracts like this document, then
you will probably want to write the abstract once and save it in a seperate
file such as {\tt abstract.tex}.  Then, you can use the same abstract for
both purposes.

\begin{quote}\begin{verbatim}
\begin{umiabstract}
  % abstract.tex
%
% This file has the abstract for the withesis style documentation
%
% Eric Benedict, Aug 2000
%
% It is provided without warranty on an AS IS basis.

\noindent       % Don't indent this paragraph.
This is not a thesis or dissertation and Master \TeX nician is not a
degree granted at the University of Wisconsin-Madison.

\vspace*{0.5em}
\noindent       % Don't indent this paragraph.
This explains the basics for using \LaTeX\ to typeset a dissertation,
thesis or masters project or preliminary report for the University of 
Wisconsin-Madison. Chapter
1 talks briefly about the thesis formatting at UW-Madison.  Chapter 2 gives
an overview of the ``essentials'' of \LaTeX{} and was written by Jon Warbrick.
Chapter 3 talks about figures and tables and what a {\em float} is.  Chapter 4
briefly introduces the {\sc Bib}\TeX{} program.  And finally, Chapter 5 discusses
some of the details for using the {\tt withesis} style file.  The material in
Chapters 2-4 basically are a review of fundamental \LaTeX{} usage and form
a reasonable basic tutorial.

\vspace*{0.5em}
\noindent       % Don't indent this paragraph.
The style discussed in this manual was originally written by Dave Kraynie and
edited by James Darrell McCauley as the {\tt puthesis} style for Purdue
University's theses.  This style was modified to form the {\tt withesis} style. This
manual is largely based on a similar manual by James Darrell McCauley and Scott Hucker.
Permission to use, copy, modify and distribute this software and its documentation
for any purpose and without fee is here by granted.  This software and its documentation
is provided ``as is'' without any express or implied warranty.

\end{umiabstract}

\begin{abstract}
  % abstract.tex
%
% This file has the abstract for the withesis style documentation
%
% Eric Benedict, Aug 2000
%
% It is provided without warranty on an AS IS basis.

\noindent       % Don't indent this paragraph.
This is not a thesis or dissertation and Master \TeX nician is not a
degree granted at the University of Wisconsin-Madison.

\vspace*{0.5em}
\noindent       % Don't indent this paragraph.
This explains the basics for using \LaTeX\ to typeset a dissertation,
thesis or masters project or preliminary report for the University of 
Wisconsin-Madison. Chapter
1 talks briefly about the thesis formatting at UW-Madison.  Chapter 2 gives
an overview of the ``essentials'' of \LaTeX{} and was written by Jon Warbrick.
Chapter 3 talks about figures and tables and what a {\em float} is.  Chapter 4
briefly introduces the {\sc Bib}\TeX{} program.  And finally, Chapter 5 discusses
some of the details for using the {\tt withesis} style file.  The material in
Chapters 2-4 basically are a review of fundamental \LaTeX{} usage and form
a reasonable basic tutorial.

\vspace*{0.5em}
\noindent       % Don't indent this paragraph.
The style discussed in this manual was originally written by Dave Kraynie and
edited by James Darrell McCauley as the {\tt puthesis} style for Purdue
University's theses.  This style was modified to form the {\tt withesis} style. This
manual is largely based on a similar manual by James Darrell McCauley and Scott Hucker.
Permission to use, copy, modify and distribute this software and its documentation
for any purpose and without fee is here by granted.  This software and its documentation
is provided ``as is'' without any express or implied warranty.

\end{abstract}
\end{verbatim}\end{quote}

Finally, the page numbers must be changed to arabic numbers to conclude the preliminary
portion of the document.
\begin{quote}\tt\singlespace\begin{verbatim}
\clearpage\pagenumbering{arabic}
\end{verbatim}\end{quote}

\section{The Body}
At the beginning of {\tt intro.tex} there is the following command:
\begin{quote}\tt\singlespace\begin{verbatim}
\chapter{Introducing the {\tt withesis} \LaTeX{} Style Guide}
\end{verbatim}\end{quote}
Following that is the text of the chapter.  The body of your thesis is separated by
sectioning commands like {\tt \verb|\chapter{}|}.  For more information on the sectioning
commands, see Section~\ref{ess:sectioning}.

Remember the basic rule of outlining you learned in grammar school:
\begin{quote}
You cannot have an `A' if you do not have a `B'
\end{quote}
Take care to have at least two {\tt \verb|\section|}s if you use the command; have
two {\tt \verb|\subsection|}s, {\em etc}.



\section{Additional Theorem Like Environments}
The {\tt withesis} style adds numerous additional theorem like environments.  These
environments were included to allow compatibility with the University of Wisconsin's
Math Department's style file.  These environments are
{\tt theorem}, {\tt assertion}, {\tt claim}, {\tt conjecture}, {\tt corollary},
{\tt definition}, {\tt example}, {\tt figger}, {\tt lemma}, {\tt prop} and {\tt remark}.

As an example, consider the following.
\begin{lemma}
Assuming that $\partial\Omega_2 = \emptyset$ and that $h(t) = 1$, we
have $$
\begin{array}{lr}
\Delta u = f, &  x\in\Omega ,\\[2pt]
u =  g_1, &  x\in\partial\Omega .
\end{array}
$$
\end{lemma}
which was produced with the following:
\begin{quote}\tt\singlespace\begin{verbatim}
\begin{lemma}
Assuming that $\partial\Omega_2 = \emptyset$ and that $h(t) = 1$, we
have $$
\begin{array}{lr}
\Delta u = f, &  x\in\Omega ,\\[2pt]
 u =  g_1, &  x\in\partial\Omega .
\end{array}
$$
\end{lemma}
\end{verbatim}\end{quote}

\section{Bibliography or References}
As a final note, the default title for the references chapter is ``LIST OF REFERENCES.''
Since some people may prefer ``BIBLIOGRAPHY'', the command
\break{\tt \verb|\altbibtitle|}
has been added to change the chapter title.

\section{Appendices}
There are two commands which are available to suppress the writing of the auxiliary information
(to the {\tt .lot} and {\tt .lof} files).  They are:
\begin{quote}\tt\singlespace\begin{verbatim}
\noappendixtables                % Don't have appendix tables
\noappendixfigures               % Don't have appendix figures
\end{verbatim}\end{quote}
These commands should be in the preamble.  See Section~\ref{usage:noapp}.

There are two environments for doing the appendix chapter: {\tt appendix} and
\break {\tt appendices}.  If you have only one chapter in the appendix, use the {\tt appendix}
environment.  If you have more than one chapter, like this manual, use the
{\tt appendices} environment.
\begin{quote}\tt\singlespace\footnotesize\begin{verbatim}
\begin{appendices}  % Start of the Appendix Chapters.  If there is only
                    % one Appendix Chapter, then use \begin{appendix}
% code.tex
% this file is part of the example UW-Madison Thesis document
% It demonstrates one method for incorporating program listings
% into a document.

\chapter{Matlab Code} \label{matlab}
This is an example of a Matlab m-file.
\verbatimfile{derivs.m}
      % Including computer code listings
\chapter{Bib\TeX\ Entries}
\label{bibrefs}
The following shows the fields required in all types of Bib\TeX\ entries.
Fields with {\tt OPT} prefixed are optional (the three letters {\tt OPT} should 
not be used).  If an optional field is not used, then the entire field can be deleted.

{\tt
\singlespace
\begin{verbatim}

@Unpublished{,                            @Manual{,
  author =      "",                         title =           "",
  title =       "",                         OPTauthor =       "",
  note =        "",                         OPTorganization = "",
  OPTyear =     "",                         OPTaddress =      "",
  OPTmonth =    ""                          OPTedition =      "",
}                                           OPTyear =         "",
                                            OPTmonth =        "",
@TechReport{,                               OPTnote =         "" 
  author =      "",                       }
  title =       "",
  institution = "",                       @InProceedings{,
  year =        "",                         author =          "",
  OPTtype =     "",                         title =           "",
  OPTnumber =   "",                         booktitle =       "",
  OPTaddress =  "",                         year =            "",
  OPTmonth =    "",                         OPTeditor =       "",
  OPTnote =     ""                          OPTpages =        "",
}                                           OPTorganization = "",
                                            OPTpublisher =    "",
@Proceedings{,                              OPTaddress =      "",
  title =           "",                     OPTmonth =        "",
  year =            "",                     OPTnote =         "" 
  OPTeditor =       "",                   }
  OPTpublisher =    "",
  OPTorganization = "",
  OPTaddress =      "",
  OPTmonth =        "",
  OPTnote =         "" 
}



@PhDThesis{,                              @InCollection{,
  author =      "",                         author =          "",
  title =       "",                         title =           "",
  school =      "",                         booktitle =       "",
  year =        "",                         publisher =       "",
  OPTaddress =  "",                         year =            "",
  OPTmonth =    "",                         OPTeditor =       "",
  OPTnote =     ""                          OPTchapter =      "",
}                                           OPTpages =        "",
                                            OPTaddress =      "",
                                            OPTmonth =        "",
                                            OPTnote =         ""
                                          }

 
@Misc{,                                   @InCollection{,
  OPTauthor =       "",                     author =          "",
  OPTtitle =        "",                     title =           "",
  OPThowpublished = "",                     chapter =         "",
  OPTyear =         "",                     publisher =       "",
  OPTmonth =        "",                     year =            "",
  OPTnote =         ""                      OPTeditor =       "",
}                                           OPTpages =        "",
}                                           OPTvolume =       "",
                                            OPTseries =       "",
                                            OPTaddress =      "",
                                            OPTedition =      "",
                                            OPTmonth =        "",
                                            OPTnote =         ""
                                          }

@MastersThesis{,                          @Article{,
  author =      "",                         author =          "",
  title =       "",                         title =           "",
  school =      "",                         journal =         "",
  year =        "",                         year =            "",
  OPTaddress =  "",                         OPTvolume =       "",
  OPTmonth =    "",                         OPTnumber =       "",
  OPTnote =     ""                          OPTpages =        "",
}                                           OPTmonth =        "",
                                            OPTnote =         ""
                                           }\end{verbatim} }
    % a BibTeX reference
\chapter{Mathematics Examples}
This appendix provides an example of \LaTeX's typesetting
capabilities.  Most of text was obtained from the University of
Wisconsin-Madison Math Department's example thesis file.

\section{Matrices}
The equations for the {\em dq}-model of an induction machine in the
synchronous reference frame are
\begin{eqnarray}
 \left[\begin{array}{c} v_{qs}^e\\v_{ds}^e\\v_{qr}^e\\v_{dr}^e  \end{array}\right]                                                                                                                                                                                                                                                                                                                                                                                                                                                                                                              
 &=& \left[ \begin{array}{cccc}
 r_s + x_s\frac{\rho}{\omega_b} & \frac{\omega_e}{\omega_b}x_s & x_m\frac{\rho}{\omega_b} & \frac{\omega_e}{\omega_b}x_m \\
 -\frac{\omega_e}{\omega_b}x_s & r_s + x_s\frac{\rho}{\omega_b} & -\frac{\omega_e}{\omega_b}x_m & x_m\frac{\rho}{\omega_b} \\
 x_m\frac{\rho}{\omega_b} & \frac{\omega_e -\omega_r}{\omega_b}x_m & r_r'+x_r'\frac{\rho}{\omega_b} & \frac{\omega_e - \omega_r}{\omega_b}x_r' \\
 -\frac{\omega_e -\omega_r}{\omega_b}x_m & x_m\frac{\rho}{\omega_b} & -\frac{\omega_e - \omega_r}{\omega_b}x_r' & r_r' + x_r'\frac{\rho}{\omega_b}
 \end{array} \right]
 \left[\begin{array}{c} i_{qs}^e\\i_{ds}^e\\i_{qr}^e\\i_{dr}^e\end{array} \right] \label{volteq}\\
 T_e&=&\frac{3}{2}\frac{P}{2}\frac{x_m}{\omega_b}\left(i_{qs}^ei_{dr}^e - i_{ds}^ei_{qr}^e\right) \label{torqueeq}\\
 T_e-T_l&=&\frac{2J\omega_b}{P}\frac{d}{dt}\left(\frac{\omega_r}{\omega_b}\right) \label{mecheq}.
\end{eqnarray}

\section{Multi-line Equations}

\LaTeX{} has a built-in equation array feature, however the
equation numbers must be on the same line as an equation.  For example:
\begin{eqnarray}
\Delta u + \lambda e^u &= 0&u\in \Omega,  \nonumber \\
u&=0&u\in\partial\Omega.
\end{eqnarray}

Alternatively, the number can be centered in the equation using the
following method.
%
% The equation-array feature in LaTeX is a bad idea.  For centered
% numbers you should set your own equations and arrays as follows:
%
\def\dd{\displaystyle}
\begin{equation}\label{gelfand}
\begin{array}{rl}
\dd \Delta u + \lambda e^u = 0, &
\dd u\in \Omega,\\[8pt] % add 8pt extra vertical space. 1 line=23pt
\dd u=0, & \dd u\in\partial\Omega.
\end{array}
\end{equation}
The previous equation had a label.  It may be referenced as
equation~(\ref{gelfand}).


\section{More Complicated Equations}
\section*{Rellich's identity}\label{rellich.section}
\setcounter{theorem}{0}
%
%

Standard developments of Pohozaev's identity used an identity by
Rellich~\cite{rellich:der40}, reproduced here.

\begin{lemma}[Rellich]
Given $L$ in divergence form and $a,d$ defined above, $u\in C^2
(\Omega )$, we have
\begin{equation}\label{rellich}
\int_{\Omega}(-Lu)\nabla u\cdot (x-\overline{x})\, dx
= (1-\frac{n}{2}) \int_{\Omega} a(\nabla u,\nabla u) \, dx
-
\frac{1}{2} \int_{\Omega}
d(\nabla u, \nabla u) \, dx
\end{equation}
$$
+
\frac{1}{2} \int_{\partial\Omega} a(\nabla u,\nabla u)(x-\overline{x})
\cdot \nu  \, dS
-
\int_{\partial\Omega}
a(\nabla u,\nu )\nabla u\cdot (x-\overline{x}) \, dS.
$$
\end{lemma}
{\bf Proof:}\\
There is no loss in generality to take $\overline{x} = 0$. First
rewrite $L$:
$$Lu = \frac{1}{2}\left[ \sum_{i}\sum_{j}
\frac{\partial}{\partial x_i}
\left( a_{ij} \frac{\partial u}{\partial x_j} \right) +
\sum_{i}\sum_{j}
\frac{\partial}{\partial x_i}
\left( a_{ij} \frac{\partial u}{\partial x_j} \right)
\right]$$
Switching the order of summation on the second term and relabeling
subscripts, $j \rightarrow i$ and $i \rightarrow j$, then using the fact
that $a_{ij}(x)$ is a symmetric matrix,
gives the symmetric form needed to derive Rellich's identity.
\begin{equation}
Lu = \frac{1}{2} \sum_{i,j}\left[
\frac{\partial}{\partial x_i}
\left( a_{ij} \frac{\partial u}{\partial x_j} \right) +
\frac{\partial}{\partial x_j}
\left( a_{ij} \frac{\partial u}{\partial x_i} \right)
\right].
\end{equation}

Multiplying $-Lu$ by $\frac{\partial u}{\partial x_k} x_k$ and integrating
over $\Omega$, yields
$$\int_{\Omega}(-Lu)\frac{\partial u}{\partial x_k} x_k \, dx=
-\frac{1}{2} \int_{\Omega}
\sum_{i,j}\left[
\frac{\partial}{\partial x_i}
\left( a_{ij} \frac{\partial u}{\partial x_j} \right) +
\frac{\partial}{\partial x_j}
\left( a_{ij} \frac{\partial u}{\partial x_i} \right)
\right]
\frac{\partial u}{\partial x_k} x_k \, dx$$
Integrating by parts (for integral theorems see~\cite[p. 20]
{zeidler:nfa88IIa})
gives
$$= \frac{1}{2} \int_{\Omega}
\sum_{i,j} a_{ij} \left[
\frac{\partial u}{\partial x_j}
\frac{\partial^2 u}{\partial x_k\partial x_i} +
\frac{\partial u}{\partial x_i}
\frac{\partial^2 u}{\partial x_k\partial x_j}
\right] x_k \, dx
$$
$$
+
\frac{1}{2} \int_{\Omega}
\sum_{i,j} a_{ij} \left[
\frac{\partial u}{\partial x_j} \delta_{ik} +
\frac{\partial u}{\partial x_i} \delta_{jk}
\right] \frac{\partial u}{\partial x_k} \, dx
$$
$$- \frac{1}{2} \int_{\partial\Omega}
\sum_{i,j} a_{ij} \left[
\frac{\partial u}{\partial x_j} \nu_{i} +
\frac{\partial u}{\partial x_i} \nu_{j}
\right] \frac{\partial u}{\partial x_k} x_k \, dx
$$
= $I_1 + I_2 + I_3$, where the unit normal vector is $\nu$.
One may rewrite $I_1$ as
$$I_1 = \frac{1}{2} \int_{\Omega}
\sum_{i,j} a_{ij} \frac{\partial}{\partial x_k}\left(
\frac{\partial u}{\partial x_i}
\frac{\partial u}{\partial x_j}
\right) x_k \, dx
$$
Integrating the first term by parts again yields
$$I_1 = -\frac{1}{2} \int_{\Omega}
\sum_{i,j} a_{ij} \left(
\frac{\partial u}{\partial x_i}
\frac{\partial u}{\partial x_j}
\right) \, dx
+
\frac{1}{2} \int_{\partial\Omega}
\sum_{i,j} a_{ij} \left(
\frac{\partial u}{\partial x_i}
\frac{\partial u}{\partial x_j}
\right) x_k \nu_k \, dS
$$
$$
-
\frac{1}{2} \int_{\Omega}
\sum_{i,j} \left(
\frac{\partial u}{\partial x_i}
\frac{\partial u}{\partial x_j}
\right) x_k \frac{\partial a_{ij}}{\partial x_k}\, dx.
$$
Summing over $k$ gives
$$\int_{\Omega}(-Lu)(\nabla u\cdot x)\, dx =
-\frac{n}{2} \int_{\Omega}
\sum_{i,j} a_{ij} \left(
\frac{\partial u}{\partial x_i}
\frac{\partial u}{\partial x_j}
\right) \, dx
$$
$$
+
\frac{1}{2} \int_{\partial\Omega}
\sum_{i,j} a_{ij} \left(
\frac{\partial u}{\partial x_i}
\frac{\partial u}{\partial x_j}
\right) (x\cdot \nu ) \, dS
-
\frac{1}{2} \int_{\Omega}
\sum_{i,j} \left(
\frac{\partial u}{\partial x_i}
\frac{\partial u}{\partial x_j}
\right) (x\cdot  \nabla a_{ij}) \, dx
$$
$$
+
\frac{1}{2} \int_{\Omega}
\sum_{i,j,k} a_{ij} \left[
\frac{\partial u}{\partial x_j}
\frac{\partial u}{\partial x_k} \delta_{ik} +
\frac{\partial u}{\partial x_i}
\frac{\partial u}{\partial x_k} \delta_{jk}
\right] \, dx
$$
$$- \frac{1}{2} \int_{\partial\Omega}
\sum_{i,j} a_{ij} \left[
\frac{\partial u}{\partial x_j} \nu_{i} +
\frac{\partial u}{\partial x_i} \nu_{j}
\right] (\nabla u\cdot x) \, dS.
$$
Combining the first and fourth term on the right-hand side
simplifies the expression
$$\int_{\Omega}(-Lu)(\nabla u\cdot x)\, dx
=
(1-\frac{n}{2}) \int_{\Omega}
\sum_{i,j} a_{ij} \left(
\frac{\partial u}{\partial x_i}
\frac{\partial u}{\partial x_j}
\right) \, dx
$$
$$
+
\frac{1}{2} \int_{\partial\Omega}
\sum_{i,j} a_{ij} \left(
\frac{\partial u}{\partial x_i}
\frac{\partial u}{\partial x_j}
\right) (x\cdot \nu ) \, dS
-
\frac{1}{2} \int_{\Omega}
\sum_{i,j} \left(
\frac{\partial u}{\partial x_i}
\frac{\partial u}{\partial x_j}
\right) (x\cdot  \nabla a_{ij}) \, dx
$$
$$
-
\frac{1}{2} \int_{\partial\Omega}
\sum_{i,j} a_{ij} \left[
\frac{\partial u}{\partial x_j} \nu_{i} +
\frac{\partial u}{\partial x_i} \nu_{j}
\right] (\nabla u\cdot x) \, dS.
$$
Using the notation defined above, the result follows.


      % Complex Equations from the UW Math Department

\end{appendices}    % End of the Appendix Chapters. ibid on \end{appendix}
\end{verbatim}\end{quote}
The difference between these two environments is the way that the chapter header is
created and how this is listed in the table of contents.
